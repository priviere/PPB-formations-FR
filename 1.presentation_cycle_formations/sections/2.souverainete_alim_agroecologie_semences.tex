\section{Souveraineté alimentaire, \agec~et semences}
\subsection{Du concept de sécurité alimentaire à celui de souveraineté alimentaire}

Lors du sommet mondial de l'alimentation à la FAO en novembre 1996, la Déclaration de Rome pose les bases d'une concertation mondiale sur les problèmes de l'alimentation dans le monde \cite{fao_declaration_1996}.

Les parties contractantes s'engagent à mettre en oeuvre et soutenir le Plan d’Action du Sommet Mondial de l’Alimentation et les délégations s'engagent à éradiquer \guill{\textit{la faim dans tous les pays}}.

La Déclaration du forum des Organisations Non Gouvernementales et paysannes à Rome en 1996 rejoint le Plan d’Action du Sommet Mondial de l’Alimentation et pointe \guill{\textit{certaines causes fondamentales de l'insécurité alimentaire}} : 
\begin{enumerate}
\item La globalisation de l'économie mondiale livrée aux multinationales et au modèle de surconsommation qui amène à la \guill{\textit{destruction des économies rurales et à la faillite de l'agriculture familiale}}.
\item L'agriculture industrialisée qui détruit \guill{\textit{l'exploitation traditionnelle et empoisonne la planète et tous les êtres vivants}}
\item Les exportations subventionnées qui %provoquent la réduction du stock mondial de céréales et 
%accroissent \guill{\textit{l'instabilité du marché au détriment des agriculteurs familiaux}}.
\item Le rôle du FMI et de la Banque Mondiale qui obligent \guill{\textit{les agriculteurs familiaux et les personnes vulnérables}} à payer \guill{\textit{le prix de l'ajustement structurel et du remboursement de la dette}}.
\end{enumerate}

Pour répondre à ces difficultés, ces organisations proposent un modèle alternatif \guill{\textit{basé sur la décentralisation}} qui redistribuerait la richesse et le pouvoir et permettrait de parvenir à la sécurité alimentaire et à la protection des écosystèmes qui rendent la vie possible sur terre.

La Déclaration souligne \guill{\textit{six éléments clés de ce modèle alternatif}} : 
aider la capacité à produire des petits producteurs, 
inverser la concentration des richesses et du pouvoir,
développer l'\agec,
améliorer la capacité des états et des gouvernements à garantir la sécurité alimentaire,
renforcer la participation des organisations populaires et des ONG à tous les niveaux
et faire garantir le droit à l'alimentation par le droit international.

C'est de cette Déclaration que le concept de souveraineté alimentaire émerge.
Les questions liées à la gouvernance et à l'autonomie s'ajoutent aux questions liées à la sécurité alimentaire.
La dimension politique du contrôle des richesses et du pouvoir par les populations est au centre du concept de souveraineté.
En d'autres termes, comme le souligne la coordination européenne Via Campesina\footnote{
La Via Campesina est, d'après leur site internet, \guill{\textit{un mouvement international qui rassemble des millions de paysannes et de paysans, de petits et de moyens producteurs, de sans terre, de femmes et de jeunes du monde rural, d'indigènes, de migrants et de travailleurs agricoles ...  Elle défend l'agriculture durable de petite échelle comme moyen de promouvoir la justice sociale et la dignité.}}.} dans un communiqué de 2009 lors du Sommet Mondial sur la Sécurité Alimentaire à la FAO :
\guill{\textit{La construction de la souveraineté alimentaire repose sur la démocratisation de la prise de décision.}}.


\subsection{Etat des lieux et menaces pour la sécurité alimentaire dans le monde}
Selon la FAO \cite{fao_letat_2012}, en 2012, 870 millions de personnes souffrent de la faim dans le monde dont 850 millions dans les pays en voie de développement.

Dans les pays du nord, le système agricole intensif mis en place n'est pas durable.
L'impact sur la sécurité alimentaire est direct et négatif \cite{fao_biodiversity_2010,mea_ecosystems_2005,pimentel_environmental_2005,fao_statistical_2013,iaastd_agriculture_2008}.
Sont dangereusement réduits, les bienfaits rendus par les agrosystèmes à l'humanité : 
\begin{enumerate}
\item équilibre des cycles des nutriments, de la formation des sols, de la pollinisation, du contrôle biologique,
\item régulation du climat et des maladies,
\item approvisionnement en nourriture, eau, bois,
\item et par là même, accès à la culture, à l'esthétisme, à l'éducation. 
\end{enumerate}
Ces services éco-systémiques sont malmenés et la biodiversité recule \cite{mea_ecosystems_2005}.
La production de nourriture est basée sur une dépendance accrue aux intrants tels que les fertilisants chimiques, les produits phyto-sanitaires (herbicides, pesticides, fongicides), à l'irrigation qui provoque salinisation, érosion et abaissement des nappes phréatiques; et également aux antibiotiques utilisés pour les élevages \cite{fao_biodiversity_2010}.

Par ailleurs, pour la société, ce modèle agricole a un coût élevé pour sauvegarder la santé des hommes, des végétaux, des animaux et plus généralement de tout l'écosystème \cite{pimentel_environmental_2005,bommelear_couts_2011}.\\

A ces problèmes liés à l'agriculture non durable s'ajoute le changement climatique qui accentue l'insécurité alimentaire \cite{smith_agriculture_2007}.
La biodiversité diminue et les systèmes biologiques sont déstabilisés \cite{suarez_les_2002}.
Le changement climatique engendre une irrégularité et généralement une baisse des rendements agricoles \cite{smith_agriculture_2007,lobell_climate_2011}.

\subsection{Les alternatives : agroécologie et participation des acteurs}

Il y a un consensus au niveau des rapports internationaux : 
pour atteindre un système agricole qui permette, à la fois, de produire assez de nourriture saine et nutritive, qui s'adapte au changement climatique, qui respecte l'environnement, qui soit durable et fasse participer les différents acteurs aux processus de décisions : l'agriculture doit changer de paradigme \cite{fao_biodiversity_2010,iaastd_agriculture_2008,mea_ecosystems_2005,fao_declaration_2009,unctad_wake_2013}.

Pour réaliser un tel objectif, l'IAASTD - un consortium de 400 experts d'origines et de disciplines très différentes qui ont travaillé pendant quatre ans - appelle à une réorientation des sciences agronomiques vers une approche plus holistique des systèmes \cite{iaastd_agriculture_2008,even_liaastd_2009}.
Il s'agit de prendre en compte la complexité, la diversité, l'aspect pluri-factoriel et multi-fonctionnel des systèmes agricoles.
Selon le rapport de l'IAASTD, les dimensions sociales, culturelles, économiques et écologiques liées aux services éco-systémiques et aussi la diversité des processus d’innovation technologique doivent associer connaissances et savoirs locaux avec les savoirs scientifiques formels. \cite{iaastd_agriculture_2008,fao_report_2010,mea_ecosystems_2005}.

L'IAASTD souligne l'importance de l'accès aux ressources, notamment l'eau et la terre, du rapprochement entre science et société et souligne la nécessité de faire évoluer les modes de gouvernance, afin de renforcer la participation des différents acteurs du monde agricole aux processus de décision et d’évaluation.

Le rapport promeut le développement de l'\agec~et des recherches interdisciplinaires \cite{even_liaastd_2009}.
Ce point est corroboré par d'autres rapports de la FAO \cite{fao_international_2007,fao_biodiversity_2010}.
