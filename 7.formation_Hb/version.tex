
%%%%%%%%%%%%%%%%%%%%%%%%%%%%%%%%%%%%%%%%%%%%%%%%%%%%%%%%%%%%%%%%%%%%%%%%%%%%%%%%%%%%%%%%%%%%%%%%
%\newcommand{\versionFFb}{1}
%\newcommand{\dateversionFFb}{13 mai 2015}
% Fiche version 1
%%%%%%%%%%%%%%%%%%%%%%%%%%%%%%%%%%%%%%%%%%%%%%%%%%%%%%%%%%%%%%%%%%%%%%%%%%%%%%%%%%%%%%%%%%%%%%%%


%%%%%%%%%%%%%%%%%%%%%%%%%%%%%%%%%%%%%%%%%%%%%%%%%%%%%%%%%%%%%%%%%%%%%%%%%%%%%%%%%%%%%%%%%%%%%%%%
% \newcommand{\versionFFb}{2}
% \newcommand{\dateversionFFb}{21 mars 2016}
% Présentation: Version de départ
% Fiche: mise à jour
%%%%%%%%%%%%%%%%%%%%%%%%%%%%%%%%%%%%%%%%%%%%%%%%%%%%%%%%%%%%%%%%%%%%%%%%%%%%%%%%%%%%%%%%%%%%%%%%

%%%%%%%%%%%%%%%%%%%%%%%%%%%%%%%%%%%%%%%%%%%%%%%%%%%%%%%%%%%%%%%%%%%%%%%%%%%%%%%%%%%%%%%%%%%%%%%%
\newcommand{\versionFFb}{3}
\newcommand{\dateversionFFb}{8 mars 2017}
% to do
% la CNIL + cf règlement intérieur SP céréales
% protection des données, cf ici:
% https://info.securityinabox.org/fr/handsonguides
% https://info.securityinabox.org/fr/howtobooklet
% https://info.securityinabox.org/fr
%%%%%%%%%%%%%%%%%%%%%%%%%%%%%%%%%%%%%%%%%%%%%%%%%%%%%%%%%%%%%%%%%%%%%%%%%%%%%%%%%%%%%%%%%%%%%%%%

comme variétés de conservation/variétés sans valeur intrinsèque)
=> Cela demande une inscription sur ces deux listes spécifiques donc de répondre à des critères précis et de suivre la procédure administrative d'inscription au catalogue. Je précise simplement cela car ce n'est pas le cas pour les semences vendues à des amateurs (le "en vue d'une exploitation non commerciale")

à la confiscation d’une variété paysanne et/ou de sa dénomination par
un obtenteur qui la développe pour pouvoir l’inscrire au catalogue et
en commercialiser les semences.
=> Tout est dans le "qui la développe" mais pour ne pas effrayer trop de monde, il me semble que important de préciser quand même que c'est avant tout le COV qui confisque la variété en tant que telle. S'il n'y a qu'une "simple" inscription au catalogue (sans dépôt de COV) c'est avant tout la dénomination qui va alors être associée à la description faite pour la catalogue qui va être confisquée (pas d'enjeu type semence de ferme s'il n'y a pas de COV). Il ne sera alors plus possible en théorie de nommer de la même manière une population sensiblement différente à la variété inscrite au catalogue. Par ailleurs, normalement s'il y a eu une description, par exemple à travers le Spicilège, d'une population et qu'on lui associe un nom, alors logiquement il doit y avoir moyen de remettre en cause l'inscription au catalogue (et le dépôt éventuel du COV) sur cette variété, car on prouve avec le Spicilège que ladite  variété ne répond pas au critère de nouveauté

COV, quel accès ?
=> Deux choses sur l'exception de sélection 
1 - on estime au sein du RSP que les paysans faisant de la sélection doivent pouvoir utiliser cette exception et qu'elle n'est pas réserver aux "sélectionneurs" cad entreprise de sélection. Le petit Hic souvent est que pour faire de la sélection paysanne en générale tu le fais en condition réelle de culture (car c'est important pour les critères de sélection paysans) donc c'est difficile de différencier la partie de ton champ qui est fait pour la sélection , de celle que tu utilises pour la culture. On se met toujours en avant ces éléments dans les discussions politique en demandant une reconnaissance de la sélection paysanne et de sa spécificité par rapport à de la sélection en station d'expé
2- peut-être citer le fait que l'UPOV 2021 est en cours de réflexion : il s'agit d'évolution de la convention UPOV est on voit aujourd'hui clairement une demande de la part des industriels de suspendre l'exception de sélection pour les 5 1eres années de mise sur le marché d'une variété. Ils défendent cela ils argumentent qu'aujourd'hui la valeur économique et la rentabilité d'une variété est plus courte et que dans un contexte de concurrence accrue il est nécessaire de protéger plus strictement la variété. En pratique cela signifie que la protection du COV se rapproche de celle du brevet.

=> rappeler que les petits agriculteurs (cad cultivant une surface équivalent à la production de 32T de céréales par an) n'ont pas à payer de royalties lorsqu'ils font des semences de ferme

=> Depuis la loi biodiversité, il semble important de faire une différence entre brevet déposé au niveau français et brevet déposé au niveau européen cad à l'OEB. Je fais un topo dans le corps de mail de réponse car se sera trop long en commentaire.

=> le terme cotisation GNIS est assez ambigu pour moi, tu mets quoi dessous ?
Si un paysan ne fait que la sélection pour lui même et ne vend pas de semence il n'a pas du tout d'obligation de se référencer comme "producteur de semences". 
Par ailleurs, si le paysan vend des semences, il peut être contrôler par contre pour le respect des règles de la production de semences et également pour les règles de la commercialisation de semences mais je ne suis pas sure que cela ait un impact par rapport à la gestion des données. 
Pour qu'il y ai un contrôle sur la COV c'est vraiment rare car il faut que le détenteur du COV fasse valoir son droit. C'est à lui de le faire respecter, ce qui signifie en pratique les pouvoirs publics ne vont pas spontanément faire des compagnes de contrôle sur ce point.

- règles usages: accès aux données, pas de pbs pour les brevets qui peuvent être fait pas ailleur mais pb si récupèrent les graines et 
leur 'fond génétique'

- http://opendatacommons.org/
- inra et données open source: cf tendance à mettre en ligne toutes les données
- http://internetactu.blog.lemonde.fr/2016/10/15/big-data-de-la-prediction-a-lintervention/
- protection des données, cf ici:
	- https://info.securityinabox.org/fr/handsonguides
	- https://info.securityinabox.org/fr/howtobooklet
	- https://info.securityinabox.org/fr
