\subsection{Au sein des collectifs}  

\begin{frame}
\frametitle{La gestion des données au sein des collectifs}

La gestion des données au sein des collectifs est envisagée à deux niveaux:

\begin{itemize}
\item les avantages
\item les inconvenients
\end{itemize}

\end{frame}

\subsubsection{avantages}

\begin{frame} 
\frametitle{La gestion des données au sein des collectifs: avantages}

\yo{Objectif:} Faciliter l'animation des projets de sélection et apporter des  informations pour accompagner la sélection 

\vfil

\yo{Données stockées:}
\begin{enumerate}
\item Données brutes plutôt quantitatives sur les caractéristiques des variétés dans différents lieux. 
\item Données brutes plutôt qualitatives sur les caractéristiques des variétés dans différents lieux, les lieux, les personnes, etc.
\item Données analysées et synthétisées, à partir des données brutes (avec ou sans statistique), sous forme de tableaux, de graphiques, de rapports.
\end{enumerate}

\vfill

\yo{Accès aux données:}
Cela dépend des MSPs: par internet, sur un ordinateur non connecté à internet, par rapports papiers, etc.

\vfill

\yo{Public visé:}
Cela dépend des MSPs: les participants des programmes de sélection, les adhérents de la MSP, le plus grand nombre, etc.


\end{frame}


\subsubsection{inconvenients}

\begin{frame}
\frametitle{La gestion des données au sein des collectifs: inconvenients}

La numérisation des données peut avoir une forte influence sur l'organisation et le fonctionnement des
collectifs.

\vfill

\textit{A priori}, ils peuvent engendrer  :
\begin{itemize}
\item moins de contacts humains
\item moins d'échanges oraux
\item une prise de pouvoir des personnes qui savent utiliser l'outil, c'est à dire qui savent retrouver les informations
\item une entrave à la vie privée/professionnelle des membres du collectifs
\item des échanges anonymes donc sans garanties issues de la confiance et de la connaissance mutuelle entre les membres d'un collectif
\end{itemize}

\end{frame}

\begin{frame}
\frametitle{La gestion des données au sein des collectifs: inconvenients}

\begin{itemize}
\item Les données stockées ne sont pas exhaustives et ne peuvent pas représenter la totalité de ce qui se passe sur le terrain !!!

\item Olivier Rey : « les moyens techniques lorsqu'ils dépassent une certaine échelle, au lieu d'être libérateurs, sont aliénants »
\end{itemize}


\end{frame}


