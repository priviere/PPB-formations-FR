\subsection{Les relations avec le monde extérieur} 

\begin{frame}
\frametitle{L'analyse des données et les relations avec le monde extérieur}

L'analyse des données et les relations avec le monde extérieur est envisagée à deux niveaux:

\begin{itemize}
\item les avantages
\item les inconvenients
\end{itemize}

\end{frame}

\subsubsection{avantages}

\begin{frame}
\frametitle{L'analyse des données et les relations avec le monde extérieur: avantages}

\begin{itemize}
\item Communication, reconnaissance et validité auprès des scientifiques, institutionnels, politiques, agriculteurs, ...
\end{itemize}

\end{frame}


\subsubsection{inconvenients}

\begin{frame}
\frametitle{L'analyse des données et les relations avec le monde extérieur: inconvenients}

\begin{itemize}
\item Attention à bien discuter des résultats avec tous les acteurs du groupe pour éviter que certains prennent les données pour dire n'importe quoi sans consultation en faisant leur propre analyse
\item Les analyses sont forcément réductionnistes
\item Olivier Rey: « La statistique devait refléter l’état du monde, le monde est devenu un reflet de la statistique. »

\end{itemize}

\end{frame}


