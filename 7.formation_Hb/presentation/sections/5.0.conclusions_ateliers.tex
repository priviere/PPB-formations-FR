\section{Conlusions et perspectives} \begin{frame}\small\tableofcontents[currentsection,currentsubsection,subsectionstyle=show/show/hide]\end{frame}

\begin{frame}
\frametitle{Conclusions et perspectives}

\yo{Conclusions}:
\begin{itemize}
\item Des avantages et des inconvénients concernant la gestion et l'analyse des données
\item Nécessité des collectifs et des réseaux de collectifs de définir leurs règles d'usage quant à l'utilisation de ces méthodes et de ces outils
\end{itemize}

\yo{Perspectives}:
\begin{itemize}
\item mieux comprendre le type de données qui permettent le dépot de brevet
\item engranger de la connaissance sur les accords entre laboratoires de recherche et organisations paysannes sur la gestion des données (cf plan de gestion des données)
\end{itemize}

\end{frame}

\section{Ateliers} 

\begin{frame}
\frametitle{Ateliers}

\begin{enumerate}
\item Ateliers en parallèle : 
	\begin{itemize}
	\item quelles gestions des données au sein des MSPs ?
	\item quelles règles d'usage ?	
	\end{itemize}
\item Restitution et discussion sur chaque thème
\end{enumerate}

\vfill

Différents thèmes à discuter:

\begin{itemize}
\item Données
\item Analyses
\item Vie du groupe « outils pour les MSPs »
\end{itemize}

\end{frame}


\begin{frame}
\frametitle{Ateliers : Données}

\begin{itemize}

\item Jusqu'où dématérialiser l'information ? Impact des outils dans les relations, dans les décisions.

\vfill

\item Quelles données stocker ?
Faut-il limiter les informations sur le comportement et les caractéristiques de nos variétés? 
	\begin{itemize}
	\item personnes
	\item histoire des lots de semences
	\item variétés (cf cas 1 et cas 2)
	\item variables		
	\end{itemize}

\end{itemize}

\end{frame}

\begin{frame}
\frametitle{Ateliers : Données}

\begin{itemize}

\item Quels accès ?
Qui ? Quoi ? Comment ? Quand ? Quelle place de l’anonymat ? Règles différentes selon le temps passé dans le groupe ? Quel équilibre entre communication et protection ?
Comment ne pas freiner le partage des connaissances indispensable à la circulation des semences? 
	\begin{itemize}
	\item Au sein des collectifs
	\item Au sein de réseaux de collectifs
	\item Le monde extérieur aux collectifs
		\begin{itemize}
		\item Met on nos données (et nos semences) dans le TIRPAA ?
		\item Quelles données rendre publiques dans le cadre des gestionnaires de RG ?
		\end{itemize}
	\end{itemize}
%	\item Quels partenariats recherche ?
%	\item Quelles contraintes administratives pour être gestionnaire RG ?
\end{itemize}

\end{frame}


\begin{frame}
\frametitle{Ateliers : Analyses et Vie du groupe}

\begin{itemize}

\item Analyses
	\begin{itemize}
		\item Jusqu'où dématérialiser l'information ? Impact des outils dans les relations, dans les décisions.
	\item comment gérer et organiser les analyses ?	 Tout le monde peut il analyser les données ?
	\end{itemize}

\vfill

\item Vie du groupe outils pour les MSPs
	\begin{itemize}
	\item Comment continuer à échanger sur les méthodes, les outils et les règles usages ?	
	\end{itemize}

\end{itemize}

\end{frame}


\begin{frame}
\frametitle{Ateliers : Réstitutions}

\end{frame}



