\subsection{Types de données} 

\begin{frame} 
\frametitle{Différents type de données}

Différents type d'informations sur :

\begin{itemize}
\item les personnes et indirectement les groupes de personnes
\item l'histoire des lots de semences
\item les variétés qui regroupent des lots de semences ayant une histoire commune
\item les variables associées aux lots de semences ou au relations entre ces lots
	\begin{itemize}
	\item savoirs et savoir-faire
	\item phénotypiques (agronomiques, nutritionnelles, organoleptiques)
	\item moléculaires
	\item contexte pédo-climatique
	\end{itemize}
\end{itemize}

\end{frame}


\begin{frame} 
\frametitle{Différents type de données}

Ces informations peuvent avoir plusieurs supports de stockage :
\begin{itemize}
\item nos cerveaux !
\item des feuilles papiers
\item des tableurs informatiques
\item des bases de données
\end{itemize}

\vfill

Ces informations peuvent ensuite être transmises à l'oral ou à l'écrit par papier ou par informatique.

\begin{block}{}
Dans la suite, il n'est question que des supports de stockage informatique,
qu'ils soient sous forme de tableurs informatiques ou de bases de données.
\end{block}

\end{frame}



\subsubsection{personnes}

\begin{frame}
\frametitle{Types de données : les personnes}

Ces données concernent le sexe, l'age, l'adresse, les numéros de téléphones, le statut professionel, ... 

\vfill

\includegraphics[width=.7\textwidth]{Logo_CNIL}

Une déclaration auprès de la CNIL n'est pas obligatoire pour les bases de données internes à des organismes à but non lucratif. 
Un adhérent peut à tout moment ordonner la suppression de ses informations personnelles d’un système de base de données.

\end{frame}

\subsubsection{histoire des lots de semences}

\begin{frame}
\frametitle{Types de données : l'histoire des lots de semences}

\begin{columns}

\begin{column}{.4\textwidth}

L'histoire d'un lot de semence peut etre lié à :
\begin{itemize}
\item un lieu,
\item une personne,
\item une année,
\item une variété
\item des relations à d'autres lots de semences :
	\begin{itemize}
	\item diffusion
	\item mélange
	\item reproduction
	\item croisement
	\item sélection
	\end{itemize}
\end{itemize}

\end{column}

\begin{column}{.6\textwidth}

\begin{center}
\includegraphics[width=.95\textwidth]{relation_SL} \tiny \cite{relation_SL}
\end{center}

\end{column}

\end{columns}

\end{frame}


\subsubsection{variétés}


\begin{frame}
\frametitle{Types de données : les variétés}

Il existe au moins deux types de statut pour les variétés :
\begin{enumerate}
\item \yo{cas 1}: Les variétés notoirement connues.
C'est le cas de la majorité des variétés anciennes et de pays conservées dans les CRB. 
Elles peuvent, sous certaines conditions, faire partie du système multilatéral du TIRPAA. 
Leurs semences peuvent également être commercialisées (« en vue d'une exploitation non commerciale » ou comme variétés de conservation/variétés sans valeur intrinsèque).

\item \yo{cas 2}: Les variétés en développement, c'est à dire en cours de sélection paysannes et/ou en gestion dynamique collective et qui ne sont pas notoirement connues . 

\end{enumerate}


\vfil

\begin{block}{}
\centering
A priori, les stratégies seront différentes selon le statut de la variété
\end{block}

\end{frame}


\subsubsection{variables}

\begin{frame}
\frametitle{Types de données : les variables associées aux lots de semences ou au relations entre ces
lots}

Il existe différents type de données :

\begin{itemize}
\item liées aux savoirs et savoir-faire
\item phénotypiques 
	\begin{itemize}
	\item agronomiques
	\item nutritionnelles
	\item organoleptiques
	\end{itemize}
\item moléculaires
\item liées contexte pédo-climatique
\end{itemize}

\end{frame}


%\paragraph{savoirs et savoir-faire}
\begin{frame}
\frametitle{Types de données : les variables concernant les savoirs et savoir-faire}

\begin{itemize}
\item les pratiques (conduite de culture, processus de transformation, etc)
\item les savoirs et les savoir-faire paysans associés à des variétés
\end{itemize}

\vfill

Ces données peuvent être recueillies avec ou sans un partenariat avec la recherche

\end{frame}



%\paragraph{phénotypiques}
\begin{frame}
\frametitle{Types de données : les variables phénotypiques}

\begin{itemize}
\item plus ou moins précises, de type qualitative en général
\item très précises grâce aux collaborations avec la recherche, quantiative en général
	\begin{itemize}
	\item longueur des épis de blé, 
	\item nombre d'épillets des épis de blé, 
	\item description des indices de nutrition azoté (INN), 
	\item type de gluten, 
	\item type de micro-nutriments, 
	\item etc. 
	\end{itemize}

\end{itemize}

\end{frame}


%\paragraph{moléculaires}
\begin{frame}
\frametitle{Types de données : les variables moléculaires}

Lors d'une collaboration avec la recherche, il est possible d'avoir accès à données moléculaires qui décrivent le génome des plantes ou des micro-organismes (cas du levain par exemple) à l'aide de marqueurs.
\end{frame}


\begin{frame}
\frametitle{Types de données : les variables moléculaires}
\framesubtitle{Principes de biologie moléculaire : la cellule}

\begin{columns}

\begin{column}{.4\textwidth}

\begin{itemize}
\item La cellule est l'unité fonctionelle des êtres vivants.
\item Elle effectue diverses missions à partir de l'information génétique (ADN). Celles-ci sont modulées par les conditions environmentales par l'intermédiaire des hormones.
\end{itemize}


\end{column}

\begin{column}{.6\textwidth}

\begin{center}
\begin{tabular}{cc}
\includegraphics[width=.9\textwidth]{cellule_vegetal} &  \rotatebox{90}{
\tiny l'Eprouvette © UNIL} \\
\small Structure simplifiée d'une cellule végétale & \\
\end{tabular}
\end{center}

\end{column}

\end{columns}

\end{frame}


\begin{frame}
\frametitle{Types de données : les variables moléculaires}
\framesubtitle{Principes de biologie moléculaire: l'ADN, sa localisation dans la cellule}

L'information génétique des cellules est inscrite dans l'ADN: l'Acide désoxyribonucléique.

Toutes les cellules d'un même organisme ont le même ADN. L'ADN se situe dans 
le  noyau (sous forme de chromosomes),
les mitochondries et
les chloroplastes.

\begin{center}
\begin{tabular}{cc}
\includegraphics[width=.8\textwidth]{cellule_chromosome_adn} &  \rotatebox{90}{
\href{http://www.bbc.co.uk/schools/gcsebitesize/science/add_aqa_pre_2011/celldivision/celldivision1.shtml}{\tiny BBC}
} \\
\small L'ADN au sein des chromosomes dans le noyau de la cellule & \\
\end{tabular}
\end{center}


\end{frame}


\begin{frame}
\frametitle{Types de données : les variables moléculaires}
\framesubtitle{Principes de biologie moléculaire: l'ADN, sa structure}

L'ADN est composé de 4 nucléotides: 
\textbf{\Huge A}, 
\textbf{\Huge C}, 
\textbf{\Huge G} et 
\textbf{\Huge T}.

\begin{columns}

\begin{column}{.6\textwidth}
\tiny
ACAAGATGCCATTGTCCCCCGGCCTCCTGCTGCTGCTGCTCTCCGGGG\\
CCACGGCCACCGCTGCCCTGCCCCTGGAGGGTGGCCCCACCGGCCGAG\\
ACAGCGAGCATATGCAGGAAGCGGCAGGAATAAGGAAAAGCAGCCTCC\\
TGACTTTCCTCGCTTGGTGGTTTGAGTGGACCTCCCAGGCCAGTGCCG\\
GGCCCCTCATAGGAGAGGAAGCTCGGGAGGTGGCCAGGCGGCAGGAAG\\
GCGCACCCCCCCAGCAATCCGCGCGCCGGGACAGAATGCCCTGCAGGA\\
ACTTCTTCTGGAAGACCTTCTCCTCCTGCAAATAAAACCTCACCCATG\\
AATGCTCACGCAAGTTTAATTACAGACCTGAAACAAGATGCCATTGTC\\
CCCCGGCCTCCTGCTGCTGCTGCTCTCCGGGGCCACGGCCACCGCTGC\\
CCTGGAGGGTGGCCCCACCGGCCGAGACAGCGAGCATATGCAGGAAGC\\
GGCAGGAATAAGGAAAAGCAGCCTCCTGACTTTCCTCGCTTGGTGGTT\\
TGAGTGGACCTCCCAGGCCAGTGCCGGGCCCCTCATAGGAGAGGAAGC\\
TCGGGAGGTGGCCAGGCGGCAGGAAGGCGCACCCCCCCAGCAATCCGC\\
CTGCAGGAACTTCTTCTGGAAGACCTTCTCCTCCTGCAAATAAAACCT\\
CACCCATGAATGCTCACGCAAGTTTAATTACAGACCTGAAACAAGATG\\
GGCAGGAATAAGGAAAAGCAGCCTCCTGACTTTCCTCGCTTGGTGGTT\\
TGAGTGGACCTCCCAGGCCAGTGCCGGGCCCCTCATAGGAGAGGAAGC\\
TCGGGAGGTGGCCAGGCGGCAGGAAGGCGCACCCCCCCAGCAATCCGC\\

\end{column}

\begin{column}{.4\textwidth}
\begin{center}
\begin{tabular}{cc}
\includegraphics[width=.8\textwidth]{611px-DNA_structure_and_bases_FR} &  \rotatebox{90}{
\href{
http://commons.wikimedia.org/wiki/File:DNA_structure_and_bases_FR.svg?uselang=fr}{\tiny Dosto (d): MesserWoland}
} \\
\small La structure de l'ADN & \\
\end{tabular}
\end{center}
\end{column}

\end{columns}

\end{frame}


\begin{frame}
\frametitle{Types de données : les variables moléculaires}
\framesubtitle{Techniques de biologie moléculaire: les marqueurs moléculaires}

\begin{columns}
\begin{column}{.5\textwidth}
\yo{Marqueurs de l’ADN} : sites dans le génome qui présentent une variation de séquence d’un individu à l’autre.  \\

~\\\vfill

Par exemple: changement d’une base ou insertion ou délétion de quelques bases. \\

~\\\vfill

Il existe plusieurs types de marqueurs par exemple les SNP (Single Nucleotide Polymorphism)
\end{column}

\begin{column}{.5\textwidth}

\small

\begin{tabular}{ccccc}
\hline
Ind 1 & Ind 2 & Ind 3 & Ind 4 & Ind 5 \\
\hline
A & A & A & A & A \\
C & C & C & C & C \\
C & C & C & C & C \\
G & G & G & G & G \\ 
T &T & T & T & T \\
G & G & G & G & G \\
A & A & A & A & A \\
\cellcolor{mln-green} G & \cellcolor{mln-green} A & \cellcolor{mln-green} T & \cellcolor{mln-green} A & \cellcolor{mln-green} G \\ 
T &T & T & T & T \\
C & C & C & C & C \\
T &T & T & T & T \\
A & A & A & A & A \\
\hline
\end{tabular}
\centering\small Marqueurs SNP
\end{column}

\end{columns}

\end{frame}


\begin{frame}
\frametitle{Types de données : les variables moléculaires}
\framesubtitle{Techniques de biologie moléculaire: les marqueurs moléculaires}

Trois familles de marqueurs:

\begin{itemize}[<+->]
\item \yo{Marqueurs \guill{neutres}}: localisés dans  une zone non codante (qui ne sera pas  transcrite et traduite en protéine).  Cette zone peut cependant influencer la  régulation de l’expression d’autres gènes.
Des marqueurs neutres répartis sur l’ensemble des chromosomes renseignent sur la diversité et le comportement global du génome et donc sur l’histoire des populations étudiées. 

\item \yo{Marqueurs localisés dans des gènes}: associés la variation de caractères.
Par exemple la précocité de floraison.

\item (\yo{Marqueurs épigénétiques}: informations non transmises par l'ADN mais par des méthylations sur l'ADN. Celles-ci sont issues d'une réponse à un environnement donné.)

\end{itemize} 


\end{frame}


