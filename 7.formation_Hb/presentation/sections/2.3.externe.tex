\subsection{Relation avec le monde extérieur} 

\begin{frame}
\frametitle{La gestion des données et les relations avec le monde extérieur}

La gestion des données et les relations avec le monde extérieur est envisagée à deux niveaux:

\begin{itemize}
\item les avantages
\item les inconvenients
\end{itemize}

\end{frame}

\subsubsection{avantages}
\begin{frame}
\frametitle{La gestion des données et les relations avec le monde extérieur: avantages}

\yo{Objectif:} Rendre public une description générale de certaines variétés dans certains terroirs et modes de culture pour communiquer vers l'extérieur. Ex: le Spicilège

\begin{center}
\includegraphics[width=\textwidth]{spicilege_capture_ecran}
\end{center}

\end{frame}

\begin{frame}
\frametitle{La gestion des données et les relations avec le monde extérieur: avantages}

\yo{Données stockées:}
Données brutes plutôt qualitatives:
description botanique et texte sur le comportement général de la variété dans différents environnements

\vfill

\yo{Accès aux données:} 
Par internet

\vfill

\yo{Public visé:}
Le plus grand nombre

\end{frame}



\subsubsection{inconvenients}

\begin{frame}
\frametitle{La gestion des données et les relations avec le monde extérieur: inconvenients}

Deux grands aspects:
\begin{itemize}
\item la biopiraterie
\item les contrôles
\end{itemize}

\end{frame}

\begin{frame}
\frametitle{La biopiraterie}

La biopiraterie peut faire référence
\begin{itemize}
\item à l'utilisation non autorisée de ressources biologiques ou des connaissances sur les ressources biologiques des communautés qui les conservent et les renouvellent.
\item au non respect des obligations de partage équitable des bénéfices issus de l'exploitation
commerciale de ressources génétiques et/ou de connaissances associées avec ceux qui les
ont sélectionnées et conservées
\item à la confiscation d'une variété paysanne et/ou de sa dénomination par un obtenteur qui la
développe pour pouvoir l'inscrire au catalogue et en commercialiser les semences. 
%Dans ce nouveau cadre qui leur est imposé de l'extérieur, les paysans ne peuvent plus commercialiser les semences et les produits de leur variété paysanne originelle sous sa dénomination et sont obligés de reproduire la nouvelle variété commerciale enregistrée seule à pouvoir revendiquer cette dénomination.
\item à la confiscation d'une variété paysanne ou de sa dénomination par un titre de propriété
intellectuel tel que le COV ou par un brevet déposé sur un de ses traits « natifs ».
%Dans ce nouveau cadre qui leur est imposé de l'extérieur, les paysans qui ont sélectionné et conservé cette variété paysanne doivent acheter les semences du nouveau propriétaire ou lui payer des droits de licence pour pouvoir continuer de la cultiver.
\end{itemize}

\end{frame}




\begin{frame} 
\frametitle{Les risques de biopiraterie}

Les principaux droits de propriété intelectuelle qui engendrent de la biopiraterie :
\begin{itemize}
\item le \COV~(COV)
\item le brevet
\end{itemize}

\vfill

\begin{block}{}
\begin{itemize}
\item Notre travail peut être vu comme du 'prebreeding' pour les industriels
\item Les risques ne sont pas les mêmes selon le type de variétés décrites (cf~cas~1~et~2) et le type de droit de propriété intelectuelle déposé
\end{itemize}
\end{block}

\end{frame}

\begin{frame} 
\frametitle{Les risques de biopiraterie}
\framesubtitle{Le COV}

Il est possible de déposer un COV quand la variété est 

\begin{itemize}
\item nouvelle: non pas ce qui n'existait pas auparavant, mais ce qui n'est pas notoirement connue
\item distincte, homogène et stable (critères DHS) 
\end{itemize}

\vfil

\yo{Quels accès?}

\begin{itemize}
\item Les sélectionneurs peuvent utiliser gratuitement et sans autorisation une variété ayant un COV comme ressource pour de nouveaux programmes de sélection. 
\item Pour 34 espèces, les paysans ne peuvent ressemer qu'en payant des royalties aux obtenteurs, pour les
autres espèces, c'est interdit.
\end{itemize}

% rappeler que les petits agriculteurs (cad cultivant une surface équivalent à la production de 32T de céréales par an) n'ont pas à payer de royalties lorsqu'ils font des semences de ferme

\end{frame}


\begin{frame} 
\frametitle{Les risques de biopiraterie}
\framesubtitle{Le COV}

\yo{Quelle stratégie pour se protéger ?}

Prouver que la variété est déjà notoirement connue : si la description antérieure à la revendication du COV est suffisamment précise et correspond aux critères nécessaires au dépôt d'un COV.
%, notamment sur les caractères phénotypiques imposés par l'union de protection des obtentions végétale (UPOV) et l'office communautaire des variétés végétales (OCVV) 

~\\

Néanmoins, la description d'une population ne peut être identique à la description d'un de ses individus qui, une fois « homogénéisé et stabilisé » deviendrait une variété DHS protégée par un COV. 

\end{frame}


\begin{frame}
\frametitle{Les risques de biopiraterie}
\framesubtitle{Le brevet}

Un brevet est possible s'il
\begin{itemize}
\item résulte d'une activité inventive et pas uniquement d'une découverte
\item porte sur quelque chose de nouveau, c'est à dire dont la connaissance n'a jamais été
accessible au public (i.e. publiée) même si elle existe depuis longtemps
\item a une application industrielle. La culture agricole, ou la sélection pour la culture agricole,
sont considérées comme des applications industrielles.
\end{itemize}

\end{frame}


\begin{frame}
\frametitle{Les risques de biopiraterie}
\framesubtitle{Le brevet}

Il est possible de déposer un brevet
\begin{itemize}
\item sur un \yo{produit} c'est à dire une matière biologique ou une information génétique, qui lorsqu'elles sont isolées de leur environnement naturel et associée à une « fonction » (i.e. un trait), qu'elle soit issue de techniques OGM ou préexistante à l'état naturel %issus PEB non en FR et ok en UE
\item sur un \yo{procédé} inventif d'obtention ou de sélection « microbiologiques » ou « techniques », sauf s'ils sont « essentiellement biologiques »
%Un procédé d'obtention de
%végétaux ou d'animaux est essentiellement biologique s'il consiste intégralement en des
%phénomènes naturels tels que le croisement ou la sélection (article 2 dir 2001/18). Un
%procédé technique ou microbiologique non essentiellement biologique correspond à toutes
%les autres situations.
\end{itemize}

\begin{block}{}
La propriété intellectuelle (ou industrielle) du brevet s’étend à toutes les plantes porteuses du trait breveté ou issues du procédé breveté.
\end{block}

\end{frame}


\begin{frame} 
\frametitle{Les risques de biopiraterie}
\framesubtitle{Le brevet}

Il y a risque de brevet si une personne ou une organisation mal intentionnée :

\begin{enumerate}
\item détient des données sur un caractère intéressant (arôme, résistance à une maladie, etc)

\item et les lie à des informations génétiques (SNP, ...) ou à un procédé technique non essentiellement biologique permettant de le reproduire.
\begin{itemize}
\item issues des programmes de recherche
\item issues des graines récupérées dans le cadre d'échanges dans une association, sur des sites d'échanges « libres » et anonymes par internet, ou alors par le système multilatéral du TIRPAA si la variété y est versée
\end{itemize}

\end{enumerate}

\begin{block}{}
La propriété intellectuelle (ou industrielle) du brevet s’étend à toutes les plantes porteuses du trait breveté ou issues du procédé breveté.
\end{block}

\end{frame}



\begin{frame}
\frametitle{Les risques de biopiraterie}
\framesubtitle{Le brevet : un exemple}

\begin{columns}

\begin{column}{.8\textwidth}
\begin{itemize}
\item L’entreprise néerlandaise Rijk Zwaan a déposé une demande de brevet portant sur toute laitue résistante à un puceron 
\item  Ce brevet a été accordé par l’OEB : Rijk Zwaan détient un monopole d’exploitation de toute laitue cultivée exprimant une résistance à ce puceron
\end{itemize}
\end{column}

\begin{column}{.2\textwidth}
\includegraphics[width=\textwidth]{salade}
\end{column}

\end{columns}

\vfill

Une organisation qui « utilise » cette résistance ne peut pas prouver qu'elle n'a pas utilisé l’invention brevetée et est contrainte de négocier un droit de licence pour pouvoir continuer à commercialiser ses semences de salade. 

\vfill

%L’article 9 de la Loi biodiversité annulerait ce brevet de l’entreprise Rik Zwann s’il était français : malheureusement, il est européen. 

\end{frame}

\begin{frame}
\frametitle{Les risques de biopiraterie}
\framesubtitle{Le brevet}

\yo{Quels accès ?}

Si nos variétés paysannes sont contaminées ou contiennent naturellement des plantes avec le trait breveté, alors il faut obtenir un droit de licence et, s'il est accordé, payer des royalties au détenteur du brevet.

Au bout de 20 ans, le brevet tombe dans le domaine public.

% exception recherche et exception d'exploitation. Ds brevet FR et unitaire UE

% plante paysans: si a un trait brevet avec ses PEB en france et s'implique aux brevet FR et UE
% mais s'applique pas sur brevet qui porte sur information génétique

% dans LAAF: si présence fortuite : ok

\vfill

\yo{Quelle stratégie pour se protéger ?}

\begin{itemize}
\item Il faut être vigilant et ne pas faciliter le travail de biopiraterie en offrant des informations sur le comportement des variétés dans les champs qui peuvent donner la « puce à l'oreille » et faciliter ensuite le dépôt de brevets. 

\item La nouvelle loi biodiversité et la LAAF ouvre des portes pour protéger les paysans. 
Mais le problème n'est pas totalement réglé.

\end{itemize}

\end{frame}



\begin{frame}
\frametitle{Les risques de biopiraterie}
\framesubtitle{Le brevet}

Actuellement, 99.99\% (?) des gènes sont potentiellement déjà brevetables à cause 
du couple phénomics + génomics.

\begin{itemize}

\item Phénomics : données haut débit

	\begin{itemize}
	\item De plus en plus d'investissements dans les plateformes de phénotypages
	\end{itemize}

	\includegraphics[width=.45\textwidth]{plateforme_phenotypage_agropolis}
	\hfill 
	\includegraphics[width=.45\textwidth]{plateforme_phenotypage_inra_toulouse}

\end{itemize}

\end{frame}


\begin{frame}
\frametitle{Les risques de biopiraterie}
\framesubtitle{Le brevet}

\begin{itemize}

\item Génomics : données haut débit
	\begin{itemize}
	\item Le prix du séquençage diminue d'année en année
\item Les espèces d'intérêt agronomique publiquement séquencées: maïs, soja, riz, pomme de terre, sorgho ...
	\end{itemize}

\end{itemize}

\vfill

En 2014, la coalition No Patent on Seeds a recensé, en Europe, au moins 2400 brevets sur les plantes et 1400 sur les animaux depuis 1980. 
Plus d'une centaine d'entre eux portent sur des « traits natifs ». 
Plus de 7500 brevets sur des plantes et 5000 sur des animaux sont en attente d'autorisation!

                                                                                                                                                                                                                                                                                                                 
\end{frame}




\begin{frame}
\frametitle{Les contrôles}

Les contrôles peuvent faire référence
\begin{itemize}
\item à la vie privée et professionnelle (concurence sur des marchés de vente, sur des demandes de subventions)
\item aux normes sanitaires, environnementales, de biosécurité
\item aux métiers de paysan, notamment liés à
\begin{itemize}
\item la sélection
\item la transformation à la ferme, menacée par l'industrie % (cf dossier meunerie)
\end{itemize}
\end{itemize}

\end{frame}
