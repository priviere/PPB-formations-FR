\section{Environnement extérieur} \begin{frame}\small\tableofcontents[currentsection,currentsubsection,subsectionstyle=show/show/hide]\end{frame}

\subsection{Projets de recherche}

\begin{frame}
\frametitle{Projets de recherche}

Travailler avec la recherche pour
\begin{itemize}
\item mieux comprendre le comportement de nos populations dans les champs, réfléchir à nos organisations, formaliser des méthodes, etc
\item faire reconnaître notre travail au niveau des institutions et des décideurs
\item permettre de faire évoluer les équipes de recherche vers une science plus collaborative et plus citoyenne, qui prend mieux en compte la complexité
\end{itemize}

\begin{block}{}
Travailler avec la recherche apporte des données inédites et des analyses statistiques pointues.
\end{block}

\end{frame}


\begin{frame}
\frametitle{Les projets de recherche}
\framesubtitle{Mettre en place une recherche collaborative}

Trois étapes importantes :

\begin{enumerate}
\item se mettre d'accord sur la construction et le déroulement du projet
\item rédiger un accord de consortium pour protéger les outils, les semences et les informations liées aux semences, etc
\begin{block}{}
\centering
Cela est possible si nous sommes partenaires.

!!! la confiance est nécessaire quoi qu'il arrive !!!
\end{block}

Même si la confiance s'installe entre les acteurs, les contraintes institutionnelles ne sont pas éradiquées pour autant (cf personne morale différente de personne physique)

\item veiller au bon déroulement du projet

\end{enumerate}

\end{frame}

%\begin{frame}
%\frametitle{La recherche participative}
%\framesubtitle{Interactions entre les acteurs}
%
%\begin{center}
%\begin{overprint}
%\onslide<1>\includegraphics[width=\textwidth,page=1]{interactions-SP.pdf}
%\onslide<2>\includegraphics[width=\textwidth,page=2]{interactions-SP.pdf}
%\onslide<3>\includegraphics[width=\textwidth,page=3]{interactions-SP.pdf}
%\onslide<4>\includegraphics[width=\textwidth,page=4]{interactions-SP.pdf}
%\end{overprint}
%\end{center}
%\end{frame}
%
%
%
%\begin{frame}
%\frametitle{La recherche participative}
%
%\framesubtitle{Co-construction du programme \\ quelques étapes à construire dans un programme ... liste non exhaustive !}
%
%\begin{columns}
%
%\begin{column}{.75\textwidth}
%\begin{center}
%\includegraphics[width=\textwidth]{etapes_co_construction}
%\end{center}
%\end{column}
%
%\begin{column}{.25\textwidth}
%Fond gris : étapes discutées au cours du temps.\\
%Fond blanc : étapes qui suivent le cycle de la plante.
%\end{column}
%
%\end{columns}
%
%\end{frame}
%
%
%
%\begin{frame}
%\frametitle{Les règles d'usage au sein d'un groupe}
%\framesubtitle{Co-construction du programme}
%
%Concrétement c'est se poser ensemble les questions suivantes: \\
%
%\begin{center}
%
%\begin{tabular}{
%p{.4\textwidth}
%p{.4\textwidth}
%}
%
%\hline
%Pour chaque étape & Globalement \\
%\hline
%
%qui?
%quoi?
%comment?
%quand?
%pourquoi?
%où?
%...
%
%&
%
%quels financements?
%quelle place des statistiques?
%quelle gestion des données?
%...
%\\
%
%\hline
%\end{tabular}
%
%\end{center}
%
%~\\
%
%Ce qui peut amener à rédiger une \yo{charte} ... pour mettre en place des règles de fonctionnement collectif
%
%\vfill
%
%\begin{block}{Le programme est dynamique !!!}
%La co-construction est au coeur du programme : la sélection, les méthodes et l'organisation peuvent être modifiées à chaque étape. Ces évolutions sont basées sur des décisions collectives entre paysans, jardiniers, artisans semenciers, animateurs et chercheurs.
%\end{block}
%
%\end{frame}
%
%\subsection{Mise en place d'une charte}
%
%\begin{frame}
%\frametitle{Les règles d'usage au sein d'un groupe}
%\framesubtitle{Mise en place d'un règlement intérieur et d'une charte}
%Exemple de la charte du programme de sélection collaborative sur les céréales
%
%\begin{itemize}
%\item \yo{Article 1.} Présentation des acteurs et des objectifs
%\item \yo{Article 2.} Co-construction du programme et prise de décision
%\item \yo{Article 3.} Arrivée de nouvelles personnes dans le groupe
%\item \yo{Article 4.} Accès aux populations non issues du travail collectif et aux populations
%issues du travail collectif
%\item \yo{Article 5. \textbf{Gestion et accès des données} }
%\item \yo{Article 6.} Publication des résultats du projet
%\item \yo{Article 7.} Cadre juridique
%\item \yo{Article 8.} Départ d’adhérents du groupe
%\item \yo{Article 9.} Exclusion de personnes du groupe
%\end{itemize}
%
%
%\end{frame}
%

%
%\subsection{Les projets de recherche}
%
%\begin{frame}
%\frametitle{Les règles d'usage au sein d'un groupe}
%\framesubtitle{Les projets de recherche : mettre en place une recherche collaborative}
%
%\yo{Premère étape: se mettre d'accord sur la construction et le déroulement du projet}
%
%\vfill
%
%\begin{itemize}
%\item Etre impliqué dès le début du projet pour :
%
%\begin{itemize}
%\item Choisir les partenaires avec qui on souhaite travailler
%\item Orienter le projet pour répondre aux besoins des membres du RSP
%\end{itemize}
%
%\item Mettre en place des espaces de discussion afin de 
%\begin{itemize}
%\item participer à la mise en place des protocoles (transparence) et au choix des méthodes et
%\item participer à l'analyse des résultats, valider ce qui sort du projet concernant les membres du RSP
%\end{itemize}
%
%\end{itemize}
%
%\end{frame}
%
%
%\begin{frame}
%\frametitle{Les règles d'usage au sein d'un groupe}
%\framesubtitle{Les projets de recherche : mettre en place une recherche collaborative}
%
%\yo{Deuxième étape: l'accord de consortium pour protéger les outils, les semences et les informations liées aux semences}
%
%\vfill
%Il est important dans ce document de demander: 
%
%\begin{itemize}
%\item à ce que tous les résultats de la recherche soient communs (ainsi il est possible d'avoir un droit de regard sur ce qui sort des études),
%\item à ce qu'il n'y ait pas de titre de propriété intellectuelle déposé sur les résultats (brevet, COV notamment) et que les résultats ne facilitent pas le dépôt de titre de propriété.
%\item des ATM sur les échanges de semences (connaissances propres). Cet ATM prévient du risque de biopiraterie.
%\item une sauvegarde de confidentialité concernant les réponses aux questionnaires et aux entretiens (connaissances propres)
%\item à ce que les développements informatiques fournissent des logiciels libres
%\item Plan de Gestion des Données (PGD)
%\end{itemize} 
%
%\end{frame}
%
%
%\begin{frame}
%\frametitle{Les règles d'usage au sein d'un groupe}
%\framesubtitle{Les projets de recherche : mettre en place une recherche collaborative}
%
%\yo{Deuxième étape: l'accord de consortium pour protéger les outils, les semences et les informations liées aux semences}
%
%\vfill
%
%
%\begin{block}{}
%\centering
%Cela est possible si nous sommes partenaires.
%
%!!! la confiance est nécessaire quoi qu'il arrive !!!
%\end{block}
%
%\vfill
%
%
%\begin{block}{}
%\centering
%Prestataire = tiers au projet = aucun droit
%\end{block}
%
%
%\end{frame}
%
%
%\begin{frame}
%\frametitle{Les règles d'usage au sein d'un groupe}
%\framesubtitle{Les projets de recherche : mettre en place une recherche collaborative}
%
%\yo{Troisième étape: le déroulé du projet}
%
%\vfill
%Veiller à ce que des des espaces de discussion soient ouverts afin de 
%\begin{itemize}
%\item participer à la mise en place des protocoles (transparence) et au choix des méthodes et
%\item participer à l'analyse des résultats, valider ce qui sort du projet concernant les membres du RSP
%\end{itemize}
%
%\end{frame}



\subsection{TIRPAA}
\begin{frame}
\frametitle{TIRPAA}

Le Traité International sur les Ressources Phytogénétiques pour l’Alimentation et l’Agriculture
(TIRPAA) installe un système d’échange multilatéral entre les parties prenantes. 

\vfill

%Ce système est censé permettre l’échange des ressources et aussi le partage des avantages résultant de l'exploitation de ces ressources, de manière juste, transparente et équitable. 

%Les échanges se font entre les pays et relèvent à l'intérieur de chaque pays des législations nationales. 

Les ressources phytogénétiques versées dans le système multilatéral (les variétés dites patrimoniales) ne sont pas soumises, en théorie, à un droit de propriété intellectuelle et sont accessibles à tous. 

\vfill

% cf aussi tout aux USA

%C'est bien évidement la théorie car, en pratique, aucun partage des
%avantages n'a eu lieu alors que de très nombreuses variétés commerciales, protégées par des titres
%de propriété intellectuelles, sont issues de ressources du TIRPAA.


%Les variétés inscrites au catalogue officiel ne peuvent pas être reconnues en France comme des
%ressources phytogénétiques (sauf pour les 11 variétés de conservation depuis le 12 mars 2015!). Les
%ressources phytogénétiques françaises sont reconnues « patrimoniales » lorsqu'elles sont
%notoirement connues et font partie de l'histoire agricole et alimentaire sur le territoire national.

%Elles peuvent être versées dans la collection nationale à l'initiative des personnes publiques ou
%privées qui en assurent la gestion, à condition qu'elles aient « obtenu l'accord du fournisseur initial
%de chaque ressource pour l'utiliser et l'échanger en vue d’une libre utilisation ». Elles sont alors
%automatiquement versées au système multilatéral du TIRPAA. Ces ressources sont conservées dans
%des Centres de Ressources Biologiques (CRB), des collections privées d'organismes publics (INRA,
%Universités) ou privés.

%En France, à ce jour, seules quatre collections privées de l'INRA ont été versées : maïs, blés,
%fourragères et pommes de terre entre 2009 et 2013. 

%Tout gestionnaire de ressources
%phytogénétiques patrimoniales ou non patrimoniales peut solliciter une reconnaissance officielle
%qui lui octroie de fait le droit d'échanger des ressources phytogénétiques. La reconnaissance de la
%conservation « in situ y compris sur l'exploitation agricole » et de ses « gestionnaires » peut ainsi
%faciliter les échanges entre agriculteurs et/ou jardiniers de semences n'appartenant pas à une
%variété inscrite au catalogue.

Le système multilatéral du TIRPAA peut faciliter la biopiraterie. 

\vfill

\begin{block}{}
\centering
Qu'est ce qui est considéré comme variété patrimoniale ? Cf cas 1 et cas 2.
\end{block}

%De plus ces graines sont obligatoirement accompagnées de données passeport qui précisent au
%minimum le lieu et la date de collecte ainsi que quelques caractéristiques pouvant être associées à
%un trait brevetable (résistance à un bioagresseur, adaptation à tel type de sol et de climat...).
%
%Un point de discussion est également ce que l'on considère comme « patrimonial ». En effet
%dans le cas 2, les populations en cours de sélections paysannes ne peuvent pas être considérées
%comme « patrimoniales » car elles sont nouvelles et pas (encore!?) notoirement connues. Il en est
%de même pour les populations dont le nom est notoirement connu mais qui évoluent, s'adaptent, se
%mélangent parfois, et ne ressemblent plus forcément à ce qui est dans les banques.
%Des fonds publics destinés à la conservation peuvent être octroyés pour la conservation des variétés
%patrimoniales. Considérer que nos variétés ne sont que dans le cadre du cas 2 nous exclut de ces
%fonds, mais nous ouvre les portes des fonds destinés aux programmes de sélection à la ferme.
%Reste la question non encore résolue de la reconnaissance de la conservation de la diversité
%cultivée, et non de variétés stables, sous forme de gestion dynamique à la ferme. Elle ne peut
%rentrer actuellement que dans le cadre de la sélection qui n'est pensé que pour la « création » de
%nouvelles variétés stables. Néanmoins, ne serait-ce pas tomber dans un piège, tendu par les
%biopirates qui souhaitent alimenter leur pratiques grâce à ces aides financières, que de considérer
%nos nouvelles variétés paysannes comme patrimoniales tant qu'elles ne sont pas largement
%diffusées en vue d'une exploitation non commerciale ou grâce à une éventuelle évolution du
%catalogue de conservation (cf partie II.a.) ? C'est pourquoi il est important de différencier la
%reconnaissance de gestionnaire de collection ouvrant droit aux échanges de semences
%aussi pour des ressources phytogénétiques non patrimoniales, ou en développement, de la
%reconnaissance de collections patrimoniales de ressources patrimoniales fixées, stabilisées
%et inventoriées.
%

\end{frame}



\subsection{Futurs gestionnaires de ressources génétiques}
\begin{frame}
\frametitle{Futurs gestionnaires de ressources génétiques}

Le décret 2015-1731, du 22 décembre 2015 de la loi de 11 décembre 2011, propose la mise en place de gestionnaires de ressources génétiques.

\vfill

Tout est en cours de discussion au sein de la section RG du CTPS où le RSP est représenté.

\vfill


Ces gestionnaires 

\begin{itemize}
\item devront pouvoir fournir aux personnes qui le souhaitent les variétés patrimoniales et ne pourront que facturer les frais postaux et non pas les frais de conservation.
Ces dernières seront versées dans le MLS du TIRPAA.

\item s’engagent à tenir à jour une base de données leur permettant d’enregistrer les ressources phytogénétiques qu’elles gèrent et d’identifier, en particulier, les ressources phytogénétiques patrimoniales. 
\end{itemize}

\end{frame}

\begin{frame}
\frametitle{Futurs gestionnaires de ressources génétiques}

\begin{itemize}
\item devront rendre publiques les informations concernant les RG patrimoniales. Pour les autres types de RG, pas d'obligation de rendre public mais par contre obligation de tenir une liste à la dispo du ministère de l'agriculture. 
Les infos disponibles doivent notamment permettre de
\begin{itemize}
\item définir le matériel entrant et sortant de sa collection
\item assurer une traçabilité amont et aval : cad identifier ceux qui fournissent des RG  au gestionnaire et ceux qui en sont des utilisateurs (le tout hors du réseau interne du gestionnaire )  
\item conserver les informations sur le statut juridique des ressources phytogénétiques, notamment en ce qui concerne l’existence ou l’absence de titres de propriété intellectuelle et de clauses relatives à leur distribution et à leur utilisation
\end{itemize}
\item auraient des moyens financiers pour fonctionner
\item pourront échanger en leur sein leurs RG
\end{itemize}

\end{frame}

%
%\subsection{Protection des données}
%
%\begin{frame}
%\frametitle{Protection des données}
%
%Les niveaux d'accessibilité sont différents selon les cas. Les données peuvent être en ligne, comme
%dans le cas du Spicilège par exemple, sur un ordinateur avec accès par mot de passe, sur un
%ordinateur sans accès par mot de passe, etc.
%Une technique comme la cryptographie asymétrique 23 permet d'assurer un certain niveau de
%protection des données, même si cela n'est pas fiable à 100 %. Certains logiciels informatiques
%libres proposent de hauts niveaux de protection relativement accessibles pour des non initiés (True
%Crypt par exemple 24 ). Ces différentes techniques sont abordées lors des formations sur la gestion
%technique des données.
%Le meilleur moyen serait d'avoir un ordinateur non connecté à Internet et d'être vigilant aux
%données qui pourraient être transmises par clés usb.
%
%
%\end{frame}
%
