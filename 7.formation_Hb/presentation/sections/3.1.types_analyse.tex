\subsection{Types d'analyse} 

\begin{frame}
\frametitle{Différents types d'analyse de données}

Les différents types d'analyse (non exhaustif):

\begin{itemize}
\item sur les données phénotypiques pour
	\begin{itemize}
	\item estimer les potentiels génétiques des variétés, environnements, intéractions (principe de génétiques quantitatives)
	\end{itemize}
	
\item sur les données moléculaires pour étudier
	\begin{itemize}
	\item la diversité
	\item l'adaptation et d'évolution
	\item les déterminismes génétiques
	\end{itemize}

\item sur les données phénotypiques et moléculaires pour étudier
	\begin{itemize}
	\item l'association entre des données phénotypiques et génétiques
	\end{itemize}

\end{itemize}

\begin{block}{}
\centering
Ce sont les statistiques qui permettent de réaliser les analyses
\end{block}

\end{frame}


\begin{frame}
\frametitle{Les statistiques}

\begin{itemize}
\item Les statistiques tentent de répondre à une question à partir de modèles mathématiques qui sont censés représenter la réalité. 

\item Un modèle est toujours faux ! Ce n'est qu'une interprétation de la réalité qui est soumis à de nombreuses hypothèses.

\item un modèle répond à une question avec une probabilité
\end{itemize}

\end{frame}




\begin{frame}
\frametitle{Données phénotypiques}
\framesubtitle{Principes de génétique quantitative}

\begin{overprint}
\onslide<1>\centering\includegraphics[width=.8\textwidth,page=10]{gxe}
\onslide<2>\centering\includegraphics[width=.8\textwidth,page=11]{gxe}
\onslide<3>\centering\includegraphics[width=.8\textwidth,page=12]{gxe}
\onslide<4>\centering\includegraphics[width=.8\textwidth,page=13]{gxe} 
\onslide<5>\centering\includegraphics[width=.8\textwidth,page=14]{gxe} 
\onslide<6>\centering\includegraphics[width=.8\textwidth,page=15]{gxe} 
\onslide<7>\centering\includegraphics[width=.8\textwidth,page=16]{gxe} 
\onslide<8>\centering\includegraphics[width=.8\textwidth,page=17]{gxe} 
\onslide<9>\centering\includegraphics[width=.8\textwidth,page=18]{gxe} 
\onslide<10>\centering\includegraphics[width=.8\textwidth,page=19]{gxe} 
\onslide<11>\centering\includegraphics[width=.8\textwidth,page=20]{gxe} 
\onslide<12>\centering\includegraphics[width=.8\textwidth,page=21]{gxe} 
\onslide<13>\centering\includegraphics[width=.8\textwidth,page=22]{gxe} 
\end{overprint}

\end{frame}


\begin{frame}
\frametitle{Données phénotypiques}
\framesubtitle{Exemples}

Comparaisons de moyennes dans les fermes : les populations sont elles différentes entre elles ?

\begin{columns}

\begin{column}{.6\textwidth}
\begin{center}
\includegraphics[page=1,width=.9\textwidth]{PPBstats_unnamed-chunk-25-1} \tiny \cite{PPBstats_unnamed-chunk-25-1}
\end{center}
\end{column}

\begin{column}{.4\textwidth}
Quand on donne une information, il y a un risque de se tromper : c'est alpha.\\
Ici, alpha = 0.1 : j'ai 10\% de risque de me tromper.\\
~\\
Les variétés qui partagent la même ligne \textbf{NE sont PAS} différentes.
\end{column}

\end{columns}

\end{frame}


\begin{frame}
\frametitle{Données phénotypiques}
\framesubtitle{Exemples}

Analyse des interactions $G \times E $ dans le réseau d'essais :  quelles populations puis-je tester dans ma ferme ?

\begin{columns}

\begin{column}{.4\textwidth}
\begin{itemize}
\item les fermes qui sont les plus proches des vôtres (c'est à dire où les populations que vous avez chez vous ont eu un comportement proche dans une autre ferme)
\item Il est également possible de prédire le passé ...
\end{itemize}
\end{column}

\begin{column}{.6\textwidth}
\begin{center} \includegraphics[width=.8\textwidth]{PPBstats_unnamed-chunk-46-1} \tiny \cite{PPBstats_unnamed-chunk-46-1}\end{center}
\end{column}

\end{columns}


\end{frame}


\begin{frame}
\frametitle{Données moléculaires}

\yo{Objectif}: Mieux comprendre:

\begin{itemize}
\item quelle est la structure de la diversité génétique au sein des variétés locales ou paysannes,
\item comment la diversité est maintenue au cours des processus de sélection,
\item comment la diversité est mobilisée pour permettre aux populations de répondre à la sélection.
\end{itemize}

Pour cela, il faut
\begin{itemize}
\item des données phénotypiques
\item des données moléculaires: 90 marqueurs SNP suffisent: par exemple 45 neutres et 45 proches des gènes de précocité de floraison.
\end{itemize}



\end{frame}


\begin{frame}
\frametitle{Données phénotypiques et moléculaires}

\yo{Objectif}: associer un marqueur à un caractère phénotypique. 
On parle de génétique d'association, de détéction de QTL (Quantitative Trait Loci). 

Pour cela, il faut:
\begin{itemize}
\item des données phénotypiques (\guill{phénomique})
\item des données moléculaires (\guill{génomique}: \guill{puces} de milliers de marqueurs SNPs (420 000 chez le blé tendre))
\end{itemize}

\vspace{.5cm}

\begin{overprint}
\onslide<2>
\begin{center}
\begin{tabular}{ccccccc}
\hline
Ind & pmg & m1 & m2 & m3 & m4 & m5\\
\hline
Ind 1 & 55 & \cellcolor{mln-green} A & C & \cellcolor{mln-green} G & G & T \\
Ind 2 & 50 & T & C & C & G & A \\
Ind 3 & 45 & C & G & A & A & T \\
Ind 4 & 40 & T & T & A & A & C \\
Ind 5 & 35 & G & A & C & G & C \\
\hline
\end{tabular}
\end{center}

\onslide<3>
Ces associations statistiques sont dépendantes de la structure des populations, du nombre d'individus, du nombre de marqueurs ...
\end{overprint}

\end{frame}


