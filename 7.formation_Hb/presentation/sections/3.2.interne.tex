\subsection{Au sein des collecifs}  

\begin{frame}
\frametitle{L'analyse des données au sein des collectifs}

L'analyse des données au sein des collectifs est envisagée à deux niveaux:

\begin{itemize}
\item les avantages
\item les inconvenients
\end{itemize}

\end{frame}


\subsubsection{avantages}

\begin{frame}
\frametitle{L'analyse des données au sein des collectifs: avantages}

\begin{itemize}
\item Les résultats sont discutés par tous les acteurs du groupe
\item Apporter de l'information au groupe de sélection
\item Avoir un support de discussions entre équipe de recherche et collectif paysans
\end{itemize}

\end{frame}


\subsubsection{inconvenients}

\begin{frame}
\frametitle{L'analyse des données au sein des collectifs: inconvenients}

L'analyse des données peut avoir une forte influence sur l’organisation et le fonctionnement des collectifs.

\vfill

\begin{itemize}

\item les analyses sont forcément réductionnistes

\item attention à bien discuter des résultats avec tous les acteurs du groupe pour éviter que certains prennent les données pour dire n'importe quoi sans consultation en faisant leur propre analyse

\item Olivier Rey : « les moyens techniques lorsqu'ils dépassent une certaine échelle, au lieu d'être libérateurs, sont aliénants »

\item Olivier Rey: « La statistique devait refléter l’état du monde, le monde est devenu un reflet de la statistique. »

\end{itemize}

\end{frame}



\begin{frame}
\frametitle{L'analyse des données au sein des collectifs: inconvenients}

\begin{itemize}
\item La décision du paysan de sélectionner repose sur différents éléments

\begin{itemize}
\item son experience dans sa ferme
\item les échanges informels avec les paysans, les chercheurs, les techniciens, ...
\end{itemize}

et aussi,

\begin{itemize}
\item les informations stockées dans la base de données
\item les analyses statistiques
\end{itemize}

\begin{block}{}
\centering
Quelle est la place de ces éléments dans la décision?
%De tels outils ne doivent pas se substituer au bon sens paysan !
\end{block}

%: les stats et la BDD sont une goutte d'eau dans un océan que représente la décision
\end{itemize}


\end{frame}


