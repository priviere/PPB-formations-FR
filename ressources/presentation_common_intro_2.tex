% Copyright Réseau Semences Paysannes.

% Ce code est sous licence creative commons BY-NC-SA. Vous êtes autorisé à partager et à
% adapter son contenu tant que vous citez les auteurs de ce document et indiquez si des changements
% ont eu lieu, que vous ne faites pas un usage commercial de ce code, tout ou partie du matériel
% le composant, que vous partagez dans les mêmes conditions votre code issu de ce code.

% Pour citer ce code: Cycle de formations sur la gestion dynamique de la biodiversité
% cultivée dans les Maisons des Semences Paysannes en réseau. Code tex de la présentation du RSP, commune à toutes les présentations. Réseau Semences Paysannes.
% Version 1 du 15 janvier 2106. Licence CC BY NC SA.

\subsection{Définitions}

\begin{frame}
\frametitle{Définitions}

Quelques définitions avant de commencer ...
\begin{itemize}
\item \yo{Lignée pure}:  Variété constituée d’individus homozygotes sur tout le génome et ayant le même génotype. Tous les individus sont identiques entre eux. Une lignée pure peut être obtenue par des autofécondations ou croisements consanguins répétés.

\item \yo{Hybride}: Première génération d'un croisement

\item \yo{Hybride F1}: Première génération d'un croisement ayant impliqué en amont deux lignées pures

\item \yo{Croisement}: Reproduction impliquant les gamètes (pollen et ovules) d’individus différents

\item \yo{Variété-population}: Variété constituée d'individus différents avec des caractéristiques permettant de la distinguer d'autres variétés. 

\item \yo{Population}: une variété-population dans un lieu donné une année donnée

\end{itemize}


\end{frame}

