% Copyright Réseau Semences Paysannes.

% Ce code est sous licence creative commons BY-NC-SA. Vous êtes autorisé à partager et à
% adapter son contenu tant que vous citez les auteurs de ce document et indiquez si des changements
% ont eu lieu, que vous ne faites pas un usage commercial de ce code, tout ou partie du matériel
% le composant, que vous partagez dans les mêmes conditions votre code issu de ce code.

% Pour citer ce code: Cycle de formations sur la gestion dynamique de la biodiversité
% cultivée dans les Maisons des Semences Paysannes en réseau. Code tex de la strucutre commune des fiches et des présentations. Réseau Semences Paysannes.
% Version 1 du 15 janvier 2106. Licence CC BY NC SA.

% FR %%%%%%%%%%%%%%%%%%%%%%%%%%%%%%%%%%%%%%%%
\usepackage[utf8]{inputenc}
\usepackage[francais]{babel}
\usepackage[cyr]{aeguill}
\usepackage[T1]{fontenc}
\usepackage{lmodern}
% policepar défaut : Computer Modern sans serif
% http://mcclinews.free.fr/latex/introbeamer/elements_contenu.html
% http://deic.uab.es/~iblanes/beamer_gallery/
% http://mcclinews.free.fr/latex/introbeamer/elements_diaporama.html#toc6
% \Huge (géant)
% \huge (énorme)
% \LARGE (très grand)
% \Large (plus grand)
% \large (grand)
% \normalsize (normal)
% \small (petit)
% \footnotesize (assez petit)
% \scriptsize (très petit)
% \tiny (minuscule)

% Maths %%%%%%%%%%%%%%%%%%%%%%%%%%%%%%%%%%%%%%%%%
\usepackage{amsmath}
\setcounter{MaxMatrixCols}{50}

% graphs %%%%%%%%%%%%%%%%%%%%%%%%%%%%%%%%%%%%%%%%
\usepackage{float} % [H] dans figure et ça bouge pas!
\usepackage{graphicx}
\usepackage{tikz} % pour dessiner sur des figures
\usepackage{pstricks} % pour dessiner sur des figures

\usetikzlibrary{shapes,arrows}

\usepackage{wasysym} % pour faire des smileys
\usepackage{pict2e} % pour faire des dessins
\setlength{\unitlength}{1mm} % pour faire des dessins, valeur de référence : 1mm
\usepackage{pdfpages} % pour rentrer des page.pdf


% tableaux %%%%%%%%%%%%%%%%%%%%%%%%%%%%%%%%%%%%%%%%
\usepackage{lscape}
\usepackage{supertabular}

% couleurs %%%%%%%%%%%%%%%%%%%%%%%%%%%%%%%%%%%%%%%%
\usepackage{colortbl}
\usepackage{xcolor}
\definecolor{mln-green}{RGB}{13,109,35}
\definecolor{mln-brown}{RGB}{141,162,69}
\definecolor{jaune}{RGB}{255,255,0}
\definecolor{rose}{RGB}{255,0,255}
\definecolor{marron}{RGB}{136,127,0}
\definecolor{bleu}{RGB}{0,255,255}
\definecolor{vert}{RGB}{0,255,0}

% autres
\usepackage{multicol}

\usepackage{enumitem} % Customize lists
\setlist{nolistsep} % Reduce spacing between bullet points and numbered lists
\usepackage{booktabs} % Required for nicer horizontal rules in tables
\usepackage{eso-pic} % Required for specifying an image background in the title page


% conditions pour les différentes espèces
\newif\ifcereales
\newif\iftomates
\newif\ifmais

% raccourcis %%%%%%%%%%%%%%%%%%%%%%%%%%%%%%%%%%%%%%%%
\newcommand\guill[1]{\og #1 \fg}
\renewcommand{\thefootnote}{\roman{footnote}} % pour ne pas confondre avec les notes de bas de page
\newcommand{\BDD}{\texttt{SHiNeMaS}}
\newcommand{\BDDfull}{\texttt{Seed History and Network Management System}}
\newcommand{\R}{\texttt{R}}
\newcommand{\RG}{ressources génétiques}
\newcommand{\agec}{agroécologie}
\newcommand{\SP}{sélection collaborative}
\newcommand{\FR}{fermes régionales}
\newcommand{\FS}{fermes satellites}
\newcommand{\CRB}{Centre de Ressources Biologiques}
\newcommand{\CRBs}{Centres de Ressources Biologiques}
\newcommand{\MSP}{Maison des Semences Paysannes}
\newcommand{\MSPs}{Maisons des Semences Paysannes}
\newcommand{\RSP}{Réseau Semences Paysannes}
\newcommand{\INRA}{Institut National de la Recherche Agronomique}
\newcommand{\ITAB}{Institut Technique de l'Agriculture Biologique}
\newcommand{\exsitu}{\textit{ex-situ}}
\newcommand{\insitu}{\textit{in-situ}}
\newcommand{\COV}{Certificat d'Obtention Végétal}
\newcommand{\yo}[1]{\textbf{\color{mln-green}#1}}



% Titre commun à toutes les fiches et présentation
\newcommand{\titre}{\textbf{La gestion dynamique de la biodiversité cultivée dans les \MSPs~en réseau}}


% Noms des formations
\newcommand{\formationA}{Formation A. Sélection décentralisée et collaborative en réseau}
\newcommand{\descriptifFA}{
Cette formation présente les grandes lignes de la méthodologie des programmes de sélection dynamique et récurrente dans les \MSPs en réseau. 

Elle vise à donner les bases, sans rentrer dans les détails, de grands axes qui pourront être approfondis lors de formations spécifiques. 
La formation prendra comme exemple un programme sur le blé tendre. %, la tomate, le mais ou le châtaignier. 

Points abordés lors de la journée :
\begin{itemize}
\item Les céréales (généalogie, évolution, contexte historique)
\item Mobilisation et le brassage de la biodiversité cultivée

\item Evaluation et sélection
\begin{itemize}
\item Evaluation et sélection agrononomique
\begin{itemize}
\item principes de génétique quantitative et des populations pour mieux comprendre comment
il est possible de sélectionner au niveau agronomique
\item sélection décentralisée et participative en réseau
\end{itemize}
\item Evaluation et sélection organoleptique et nutritionelle
\end{itemize} 

\item Production

\item Mise en réseau des acteurs

\item Evaluation du programme et les premiers résultats

\item Règles et droits d’usage collectifs
\end{itemize}

Une visite dans les champs ou dans une ferme pourra permettre de faire une petite pause après le
repas.

Un temps de discussion sera prévu à la fin de la journée afin de réfléchir ensemble à la mise en
place d’un programme de sélection collaborative.
}

\newcommand{\formationB}{Formation B. Gestion, mobilisation et brassage de la diversité}
\newcommand{\descriptifFB}{
Cette formation approfondit les méthodes et les techniques pour gérer, mobiliser et brasser la biodiversité cultivée du blé tendre à la ferme.
Une première partie fait l'état des lieux des méthodes et des techniques en lien avec les dernières 
publications sur le sujet.
Une deuxième partie pratique permettra de réaliser ses propres croisements.
}

\newcommand{\formationC}{Formation C. Evaluation et sélection agronomique}
\newcommand{\descriptifFC}{
En cours de rédaction
}

\newcommand{\formationD}{Formation D. Evaluation et sélection organoleptique}
\newcommand{\descriptifFD}{
En cours de rédaction
}

\newcommand{\formationE}{Formation E. Evaluation bilan}
\newcommand{\descriptifFE}{
En cours de rédaction
}

\newcommand{\formationFa}{Formation Fa. Outils de gestion des données: aspects techniques}
\newcommand{\descriptifFFa}{
En cours de rédaction
%Cette formation permet de maitriser au niveau technique la base de données \BDD~(\BDDfull).
%Des tutoriels sont proposés sur différentes espèces (céréales, maïs, tomates, oignons, fourragères, arbres).
%Un temps de tests sur les ordinateurs important est prévu.
%Les aspects d'accès à l'information et d'organisation d'un réseau de base de données est traité.
%Les notions de règles et de droits d'usage sont abordés en perspectives.
}

\newcommand{\formationFb}{Formation Fb. Outils de gestion des données: aspects règles et droits d'usage}
\newcommand{\descriptifFFb}{
En cours de rédaction
%Cette formation présente les règles et droits d'usage liés aux données eu égard de l'organisation d'un collectif et des risques de biopiraterie.
%Une première partie présente les méthodes et les outils utilisés dans les programmes de sélection.
%Ensuite, les risques de biopiraterie associés aux données est abordé.
%Enfin des éléments sont données pour se poser des questions au sein d'un collectif eu égard de la gestion des données. 
%Un temps important de travail de groupe est prévu.
}

\newcommand{\formationG}{Formation G. Outils d'analyse des données}
\newcommand{\descriptifFG}{
En cours de rédaction
}

\newcommand{\formationH}{Formation H. Règles et droits d'usage collectifs}
\newcommand{\descriptifFH}{
En cours de rédaction
}

