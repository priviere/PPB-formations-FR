% Copyright Réseau Semences Paysannes.

% Ce code est sous licence creative commons BY-NC-SA. Vous êtes autorisé à partager et à
% adapter son contenu tant que vous citez les auteurs de ce document et indiquez si des changements
% ont eu lieu, que vous ne faites pas un usage commercial de ce code, tout ou partie du matériel
% le composant, que vous partagez dans les mêmes conditions votre code issu de ce code.

% Pour citer ce code: Cycle de formations sur la gestion dynamique de la biodiversité
% cultivée dans les Maisons des Semences Paysannes en réseau. Code tex de la structure des fiches. Réseau Semences Paysannes.
% Version 1 du 15 janvier 2106. Licence CC BY NC SA.

\documentclass[12pt]{book}

\usepackage[top=2cm, bottom=2cm, left=3cm, right=2cm]{geometry} % gérer les marges

\usepackage{titlesec} % Allows customization of titles
%\usepackage[titletoc]{appendix} % To add Appendix into annex section number i.e. Appendix A

% citations %%%%%%%%%%%%%%%%%%%%%%%%%%%%%%%%%%%%%%%%
\usepackage[a4paper=true, colorlinks=true, linkcolor=black,urlcolor=blue, citecolor=black]{hyperref}
%\usepackage[authoryear,round,colon]{natbib}
\usepackage[sorting=none]{biblatex}
\addbibresource{../../ressources/biblio}



%%%%%%%%%%%%%%%%%%%%%%%%%%%%%%%%%%%%%%%%
%%%%%%%%%%%%%%%%%%%%%%%%%%%%%%%%%%%%%%%%
%%%%%%%%%%%%%%%%%%%%%%%%%%%%%%%%%%%%%%%%
% pour intégrer les instruction pour SHiNeMaS
%%%%%%%%%%%%%%%%%%%%%%%%%%%%%%%%%%%%%%%%
%%%%%%%%%%%%%%%%%%%%%%%%%%%%%%%%%%%%%%%%
%%%%%%%%%%%%%%%%%%%%%%%%%%%%%%%%%%%%%%%%
\newcommand{\BDinput}[1]{
\setlength{\fboxrule}{.2mm}
\noindent
\fcolorbox{black}{black}{\makebox[\textwidth][c]{\textcolor{white}{$-$ Protocole \BD ~$-$}}} \\
\fcolorbox{black}{white}{
\begin{minipage}[]{\textwidth}
#1
\end{minipage}
}}

\newcommand{\Rinput}[1]{
\setlength{\fboxrule}{.2mm}
\noindent
\fcolorbox{black}{black}{\makebox[\textwidth][c]{\textcolor{white}{$-$ Code \R ~$-$}}} \\
\fcolorbox{black}{white}{\makebox[\textwidth][l]{#1}}
}


\newcommand{\Routput}[1]{
\setlength{\fboxrule}{.2mm}
\noindent
\fcolorbox{black}{black}{\makebox[\textwidth][c]{\textcolor{white}{$-$ Sorties ~$-$}}} \\
\fcolorbox{black}{white}{\makebox[\textwidth][l]{#1}}
}

%%%%%%%%%%%%%%%%%%%%%%%%%%%%%%%%%%%%%%%%
%%%%%%%%%%%%%%%%%%%%%%%%%%%%%%%%%%%%%%%%
%%%%%%%%%%%%%%%%%%%%%%%%%%%%%%%%%%%%%%%%
% Boites à messages
%%%%%%%%%%%%%%%%%%%%%%%%%%%%%%%%%%%%%%%%
%%%%%%%%%%%%%%%%%%%%%%%%%%%%%%%%%%%%%%%%
%%%%%%%%%%%%%%%%%%%%%%%%%%%%%%%%%%%%%%%%

% couleurs
\newcommand{\colwarning}{red}
\newcommand{\colloi}{blue}
\newcommand{\colinfosup}{orange}

% les remarques
\setlength{\fboxrule}{1mm}

\newcommand{\RQ}[1]{
~\\
\begin{center}
\fbox{
\begin{minipage}[]{.7\textwidth}
\textcolor{red}{\textbf{{#1}}}
\end{minipage}
}
\end{center}
}

% les warnings
\newcommand{\warning}[1]{
\setlength{\fboxrule}{1.5mm}\noindent
\begin{minipage}[t]{.1\textwidth}
\parbox[c]{\textwidth}{\includegraphics[width=\textwidth]{warning}}
\end{minipage}
\fcolorbox{\colwarning}{white}{
\begin{minipage}[t]{.85\textwidth}
\textcolor{\colwarning}{\textbf{{#1}}}
\end{minipage}
}~\\
}

% les infos sur les textes de loi
\newcommand{\loi}[1]{
\setlength{\fboxrule}{1.5mm}\noindent
\begin{minipage}[t]{.1\textwidth}
\parbox[c]{\textwidth}{\includegraphics[width=\textwidth]{semences_code_barre}}
\end{minipage}
\fcolorbox{\colloi}{white}{
\begin{minipage}[t]{.85\textwidth}
\textcolor{\colloi}{\textbf{{#1}}}
\end{minipage}
}~\\
}

% pour aller plus loin
\newcommand{\infosup}[1]{
\setlength{\fboxrule}{1.5mm}\noindent
\begin{minipage}[t]{.1\textwidth}
\parbox[c]{\textwidth}{\includegraphics[width=\textwidth]{livre}}
\end{minipage}
\fcolorbox{\colinfosup}{white}{
\begin{minipage}[t]{.85\textwidth}
\textcolor{\colinfosup}{\textbf{{#1}}}
\end{minipage}
}~\\
}







% impossible de mettre du verbatim dans une fonction, j'ai pas mal cherché !!!
\newcommand{\TOTO}[3]{
\begin{table}[H]
\begin{tabular}{|p{.5\textwidth} | p{.5\textwidth}|}
\hline
\multicolumn{1}{|c|}{\cellcolor{black} \textcolor{white}{$-$ Interface \BD ~$-$}}
&
\multicolumn{1}{|c|}{\cellcolor{white} \textcolor{black}{$-$ Code \R ~$-$}}
\\
\hline
#1
&
#2
\\
\hline
\multicolumn{2}{|c|}{\cellcolor{gray} \textcolor{white}{$-$ Sorties ~$-$}} \\
\hline
\multicolumn{2}{|l|}{ #3 } \\
\hline
\end{tabular}
\end{table}
}

% custom color of toc


%\titlecontents{section}[1.5cm]
%{\bfseries\color{mln-green}}
%{\contentslabel{1cm}}
%{0cm}
%{}

%
%\titlecontents{subsection}[2.5cm]
%{\color{mln-brown}\bfseries}
%{\color{mln-brown}\contentslabel{1.3cm}}
%{}
%{\color{mln-brown}\titlerule*{}\contentspage}
%
%
%\titlecontents{subsubsection}[3.5cm]
%{\color{gray}\bfseries}
%{\color{gray}\contentslabel{1.7cm}}
%{}
%{\color{gray}\titlerule*{}\contentspage}
%

%\titleformat{\chapter}[hang]{\normalfont\Large\bfseries\color{mln-green}}{\thechapter}{0pt}{~}[~]
%\titleformat{\section}[hang]{\normalfont\Large\bfseries\color{mln-green}}{\thesection.~}{0pt}{}[]
%\titleformat{\subsection}[hang]{\normalfont\large\bfseries\color{mln-brown}}{\hspace{0.8cm}\thesubsection.~}{0pt}{}[]
%\titleformat{\subsubsection}[hang]{\normalfont\normalsize\bfseries\color{gray}}{\hspace{1.9cm}\thesubsubsection.~}{0pt}{}[]
%\titleformat{\paragraph}[hang]{\normalfont\normalsize\bfseries\color{gray}}{}{0pt}{}[]
%\titleformat{\subparagraph}[hang]{\normalfont\normalsize\bfseries\color{gray}}{}{0pt}{}[]

\newcommand{\logoITABMIARSPDEAP}{
\includegraphics[width=0.15\textwidth]{Logo-ITAB}
\hspace{1cm}
\includegraphics[width=0.15\textwidth]{Logo-maiage}
\hspace{1cm}
\includegraphics[height=0.15\textwidth]{Logo-RSP}
\hspace{1cm}
\includegraphics[width=0.15\textwidth]{Logo-UMRGV}
}


\newcommand{\logoITABRSPDEAP}{
\includegraphics[width=0.15\textwidth]{Logo-ITAB}
\hspace{1cm}
\includegraphics[height=0.15\textwidth]{Logo-RSP}
\hspace{1cm}
\includegraphics[width=0.15\textwidth]{Logo-UMRGV}
}

\newcommand{\logoMIARSPDEAP}{
\includegraphics[width=0.15\textwidth]{Logo-maiage}
\hspace{1cm}
\includegraphics[height=0.15\textwidth]{Logo-RSP}
\hspace{1cm}
\includegraphics[width=0.15\textwidth]{Logo-UMRGV}
}

\newcommand{\logoRSPDEAP}{
\includegraphics[height=0.15\textwidth]{Logo-RSP.jpg}
\hspace{1cm}
\includegraphics[width=0.15\textwidth]{Logo-UMRGV.jpg}
}

\newcommand{\logoRSP}{
\includegraphics[height=0.15\textwidth]{Logo-RSP.jpg}
}

\newcommand{\headtitlepagefiche}[2]{

{#1}

\vfill

\colorbox{mln-green}{
\begin{minipage}{\textwidth}
\color{white}
\sffamily\centering\LARGE
\vspace{.4cm}
\titre \\ {#2}
\vspace{.4cm}
\end{minipage}
}

\vfill

}


%----------------------------------------------------------------------------------------
%	VARIOUS REQUIRED PACKAGES
%----------------------------------------------------------------------------------------

\usepackage{tikz} % Required for drawing custom shapes

\usepackage{enumitem} % Customize lists
\setlist{nolistsep} % Reduce spacing between bullet points and numbered lists

\usepackage{booktabs} % Required for nicer horizontal rules in tables

\usepackage{eso-pic} % Required for specifying an image background in the title page


%----------------------------------------------------------------------------------------
%	MAIN TABLE OF CONTENTS
%----------------------------------------------------------------------------------------

\usepackage{titletoc} % Required for manipulating the table of contents

%\contentsmargin{0cm} % Removes the default margin
% Chapter text styling

% ca sert à rien ça en fait !!!
%\titleformat*{\chapter}{\normalfont\huge\bfseries\color{black}}
%\titleformat*{\section}{\normalfont\huge\bfseries\color{mln-green}}
%\titleformat*{\subsection}{\normalfont\huge\bfseries\color{mln-brown}}
%\titleformat*{\subsubsection}{\normalfont\huge\bfseries\color{gray}}



\titlecontents{chapter}
{\bfseries\color{black}}
{\color{black}\contentslabel{0.7cm}}
{}
{\color{black}\titlerule*{}\contentspage}


\titlecontents{section}[1.5cm]
{\bfseries\color{mln-green}}
{\contentslabel{1cm}}
{}
{\color{mln-green}\titlerule*{}\contentspage}


\titlecontents{subsection}[2.5cm]
{\color{mln-brown}\bfseries}
{\color{mln-brown}\contentslabel{1.3cm}}
{}
{\color{mln-brown}\titlerule*{}\contentspage}


\titlecontents{subsubsection}[3.5cm]
{\color{subsubsection.color}\bfseries}
{\color{subsubsection.color}\contentslabel{1.7cm}}
{}
{\color{subsubsection.color}\titlerule*{}\contentspage}


% Section text styling
%\titlecontents{section}[1.25cm] % Indentation
%{\color{red}\addvspace{5pt}\sffamily\bfseries} % Spacing and font options for sections
%{\contentslabel[\thecontentslabel]{1.25cm}} % Section number
%{}
%{\sffamily\hfill\thecontentspage} % Page number
%[]

% Subsection text styling
%\titlecontents{subsection}[1.25cm] % Indentation
%{\color{red}\addvspace{1pt}\sffamily\small} % Spacing and font options for subsections
%{\contentslabel[\thecontentslabel]{1.25cm}} % Subsection number
%{}
%{\sffamily\;\titlerule*[.5pc]{.}\;\thecontentspage} % Page number
%[] 


%----------------------------------------------------------------------------------------
%	MINI TABLE OF CONTENTS IN CHAPTER HEADS
%----------------------------------------------------------------------------------------

% Section text styling
\titlecontents{lsection}[0em] % Indendating
{\footnotesize\sffamily} % Font settings
{}
{}
{}

% Subsection text styling
\titlecontents{lsubsection}[.5em] % Indentation
{\normalfont\footnotesize\sffamily} % Font settings
{}
{}
{}

%----------------------------------------------------------------------------------------
%	PAGE HEADERS
%----------------------------------------------------------------------------------------

\usepackage{fancyhdr} % Required for header and footer configuration

\pagestyle{fancy}
\renewcommand{\chaptermark}[1]{\markboth{\sffamily\normalsize\bfseries\hspace{.7cm}#1}{}} % Chapter text font settings
\renewcommand{\sectionmark}[1]{\markright{\sffamily\normalsize\thesection\hspace{.3cm}#1}{}} % Section text font settings
\fancyhf{} \fancyhead[LE,RO]{\sffamily\normalsize\thepage} % Font setting for the page number in the header
\fancyhead[LO]{\rightmark} % Print the nearest section name on the left side of odd pages
\fancyhead[RE]{\leftmark} % Print the current chapter name on the right side of even pages
\renewcommand{\headrulewidth}{0.5pt} % Width of the rule under the header
\addtolength{\headheight}{2.5pt} % Increase the spacing around the header slightly
\renewcommand{\footrulewidth}{0pt} % Removes the rule in the footer
\fancypagestyle{plain}{\fancyhead{}\renewcommand{\headrulewidth}{0pt}} % Style for when a plain pagestyle is specified

% Removes the header from odd empty pages at the end of chapters
\makeatletter
\renewcommand{\cleardoublepage}{
\clearpage\ifodd\c@page\else
\hbox{}
\vspace*{\fill}
\thispagestyle{empty}
\newpage
\fi}

		  
%----------------------------------------------------------------------------------------
%	SECTION NUMBERING IN THE MARGIN
%----------------------------------------------------------------------------------------

\makeatletter

%\renewcommand\section{%
 %\def\@seccntformat##1{\csname the##1\endcsname\hspace{1em} \color{mln-green}}
 %\@startsection{\textcolor{mln-green}}
%}

%\renewcommand\subsection{%
% \def\@seccntformat##1{\csname the##1\endcsname\hspace{1em} \color{mln-brown}}
% \@startsection{\textcolor{mln-brown}}
%}

                  
\renewcommand{\section}{\@startsection{section}{1}{\z@}
{-4ex \@plus -1ex \@minus -.4ex}
{1ex \@plus.2ex }
{\normalfont\large\sffamily\bfseries\color{mln-green}}}

\renewcommand{\subsection}{\@startsection {subsection}{2}{\z@}
{-3ex \@plus -0.1ex \@minus -.4ex}
{0.5ex \@plus.2ex }
{\normalfont\sffamily\bfseries\color{mln-brown}}}

\renewcommand{\subsubsection}{\@startsection {subsubsection}{3}{\z@}
{-2ex \@plus -0.1ex \@minus -.2ex}
{0.2ex \@plus.2ex }
{\normalfont\small\sffamily\bfseries\color{gray}}}                        

\renewcommand\paragraph{\@startsection{paragraph}{4}{\z@}
{-2ex \@plus-.2ex \@minus .2ex}
{0.1ex}
{\normalfont\small\sffamily\bfseries\color{gray}}}

\renewcommand\subparagraph{\@startsection{subparagraph}{4}{\z@}
{-2ex \@plus-.2ex \@minus .2ex}
{0.1ex}
{\normalfont\small\sffamily\bfseries\color{gray}}}


%----------------------------------------------------------------------------------------
%	CHAPTER HEADINGS
%----------------------------------------------------------------------------------------

\newcommand{\thechapterimage}{}
\newcommand{\chapterimage}[1]{\renewcommand{\thechapterimage}{#1}}
\def\thechapter{\arabic{chapter}}
\def\@makechapterhead#1{
\thispagestyle{empty}
{\centering \normalfont\sffamily
\ifnum \c@secnumdepth >\m@ne
\if@mainmatter
\startcontents
\begin{tikzpicture}[remember picture,overlay]
\node at (current page.north west)
{\begin{tikzpicture}[remember picture,overlay]

\node[anchor=north west,inner sep=0pt] at (0,0) {\includegraphics[width=\paperwidth]{\thechapterimage}};

\draw[anchor=west] (3cm,-3cm) node [rounded corners=25pt,fill=white,fill opacity=.6,text opacity=1,draw=mln-green,draw opacity=1,line width=2pt,inner sep=15pt]{\huge\sffamily\bfseries\textcolor{black}{ 
\begin{minipage}{17cm}
\thechapter\ .\ #1
\end{minipage}
\vphantom{plPQq}\makebox[22cm]{}}};

\draw[anchor=west] (1cm,-8cm) node [rounded corners=0pt,fill=white,fill opacity=.6,text opacity=1,draw=mln-green,draw opacity=1,line width=0pt,inner sep=15pt]{\huge\sffamily\bfseries\textcolor{mln-green}{ 
\begin{minipage}{17cm}
\printcontents{l}{1}{\setcounter{tocdepth}{1}}
\end{minipage}
\vphantom{plPQq}\makebox[22cm]{}}};

%Commenting the 3 lines below removes the small contents box in the chapter heading
%\draw[fill=white,opacity=.6] (1cm,-10cm) rectangle (20cm,-6.5cm);
%\node[anchor=north west] at (1cm,-6cm) {\parbox[t][8cm][t]{18cm}{\color{mln-green}\notmalsize\bfseries\flushleft \printcontents{l}{1}{\setcounter{tocdepth}{1}}}}; % \setcounter{tocdepth}{1}, mettre 2 pour avoir les sous section, etc

%\begin{minipage}
%\parbox[t][8cm][t]{18cm}{\color{mln-green}\notmalsize\bfseries\flushleft %\printcontents{l}{1}{\setcounter{tocdepth}{1}}}
%end{minipage}

\end{tikzpicture}};
\end{tikzpicture}}\par\vspace*{230\p@}
\fi
\fi
}
\def\@makeschapterhead#1{
\thispagestyle{empty}
{\centering \normalfont\sffamily
\ifnum \c@secnumdepth >\m@ne
\if@mainmatter
\startcontents
\begin{tikzpicture}[remember picture,overlay]
\node at (current page.north west)
{\begin{tikzpicture}[remember picture,overlay]
\node[anchor=north west] at (-4pt,4pt) {\includegraphics[width=\paperwidth]{\thechapterimage}};
\draw[anchor=west] (5cm,-9cm) node [rounded corners=25pt,fill=white,opacity=.7,inner sep=15.5pt]{\huge\sffamily\bfseries\textcolor{black}{\vphantom{plPQq}\makebox[22cm]{}}};
\draw[anchor=west] (5cm,-9cm) node [rounded corners=25pt,draw=mln-green,line width=2pt,inner sep=15pt]{\huge\sffamily\bfseries\textcolor{black}{#1\vphantom{plPQq}\makebox[22cm]{}}};
\end{tikzpicture}};
\end{tikzpicture}}\par\vspace*{230\p@}
\fi
\fi
}
\makeatother


