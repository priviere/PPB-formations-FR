\section{Perspectives}
\begin{frame}\small\tableofcontents[currentsection,currentsubsection,subsectionstyle=show/show/hide]\end{frame}



\begin{frame}
\frametitle{Perspectives}
\framesubtitle{Installation de \BDD}

cf diapo Yannick

%- date dispo V1

%- accès pour le groupe blé d'une base hébergée par le moulon dans le cadre des collaboration avec l'INRA du Moulon

%- pour les autres espèces: chaque groupe se débrouille
%Une doc d'installation et le logiciel sont disponibles dans qqls mois
%Pour le maïs, comme une dynamique nationale s'enclenche si le groupe le décide, 
%il pourrait être envisagé :

%	- que le logiciel soit hébergé 

%		- sur un serveur dans une asso et que Yannick s'occupe des mises à jour
%		mais pas de la maintenance tehcnique ni des sauvegardes

%		- au moulon (mais collaboration à mettre en place)


\end{frame}



\begin{frame}
\frametitle{Perspectives}
\framesubtitle{Quelle organisation ?}

\begin{itemize}
\item dans une maison des semences paysannes (MSP) : chaque MSP est indépendante des autres MSP (mode de gouvernance propre aux MSP)
\item dans un réseau de MSP : le mieux car permet de travailler ensemble pour mettre en réseau les données et les résultats (mode de gouvernance propre aux MSP \yo{et} au RSP)
\begin{itemize}
\item une seule base de données nationale pour toutes les MSP?
\item des bases de données locales mais reliées entre elles pour quelques données (par exemple les noms de personnes et de variétés mais pas les données liées aux relations?)
\end{itemize}
\end{itemize}

\end{frame}



\begin{frame}
\frametitle{Accès aux données}
\framesubtitle{Quels liens entre les bases de données?}

\begin{center}
\includegraphics[width=.8\textwidth,page=1]{reseau_BDD}
\end{center}

\end{frame}


\begin{frame}
\frametitle{Accès aux données}
\framesubtitle{Quels liens entre les bases de données?}

\begin{center}
\includegraphics[width=.8\textwidth,page=2]{reseau_BDD}
\end{center}

\end{frame}

\begin{frame}
\frametitle{Accès aux données}
\framesubtitle{Quels liens entre les bases de données?}

\begin{center}
\includegraphics[width=.8\textwidth,page=3]{reseau_BDD}
\end{center}

\end{frame}


\begin{frame}
\frametitle{Perspectives}
\framesubtitle{Service après installation}

\begin{itemize}
\item Procédures pour receuillir

\begin{itemize}
\item vos demandes d'évolutions de l'outils (interface d'insertion, d'admin, de requetes, etc)
\item les bugs
\end{itemize}

\begin{block}{}
=> envoyer un mail à Pierre et Yannick, en attendant une liste INRA.
il faut être le plus exhaustif possible:
\begin{itemize}
\item décrire au mieux ce que vous avez fait et qui a mené au bug
\item joindre un fichier si le bug vient d'un fichier
\item si vous trouvez une page jaune: copier coller le message
\end{itemize}
\end{block}

\item Partage des questions / experiences : créer un forum ?

\item formation tous les ans (techniques et règles d'usage, partage d'expériences)

\end{itemize}

\end{frame}


\begin{frame}
\frametitle{Perspectives}
\framesubtitle{Analyser les données}

Développement de deux packages \R: 
\begin{itemize}
\item Analyse descriptive : \texttt{shinemas2R}
\item Analyses statistiques rencontrées dans les programmes de sélection participative : \texttt{PPBstats}.
Quatre types d'analyses à implémenter:
\begin{itemize}
\item réseau (v1 prévue fin 2019)
\item agronomique et nutritionnelle (v1 prévue fin 2017)
\item organoleptique  (v1 prévue fin 2018)
\item moléculaire  (v1 prévue ???)
\end{itemize}

\end{itemize}

\vfill 

\R~: langage statistique open source. 

\vfill 

Il y aura deux moyens de l'utiliser : 
\begin{itemize}
\item En ligne de commande pour les utilisateurs avancés de \R
\item Par l'interface, ce qui ne nécessite pas de connaître \R
\end{itemize}

\end{frame}



\begin{frame}
\frametitle{Perspectives}
\framesubtitle{Exemple d'analyses de données}
\begin{center}
\includegraphics[width=.5\textwidth]{/home/pierre/Documents/geek/R-stats/Rdev/R_package_shinemas2R/shinemas2R/vignettes/figures/shinemas2R_unnamed-chunk-23-1.pdf}
\hfill
\includegraphics[width=.5\textwidth]{/home/pierre/Documents/geek/R-stats/Rdev/R_package_shinemas2R/shinemas2R/vignettes/figures/shinemas2R_unnamed-chunk-54-1.pdf}
\end{center}
\end{frame}


\begin{frame}
\frametitle{Perspectives}
\framesubtitle{Exemple d'analyses de données}
\begin{center}
\includegraphics[width=.5\textwidth]{/home/pierre/Documents/geek/R-stats/Rdev/R_package_shinemas2R/shinemas2R/vignettes/figures/shinemas2R_unnamed-chunk-86-1.pdf}
\hfill
\includegraphics[width=.5\textwidth]{/home/pierre/Documents/geek/R-stats/Rdev/R_package_shinemas2R/shinemas2R/vignettes/figures/shinemas2R_unnamed-chunk-100-1.pdf}
\end{center}
\end{frame}


\begin{frame}
\frametitle{Perspectives}
\framesubtitle{Exemple de communication des résultats}

\begin{center}
\includegraphics[height=.9\textheight]{dossier_retour}
\end{center}

\end{frame}


\begin{frame}
\frametitle{Perspectives}
\framesubtitle{Captures d'écran de l'interface en cours de développement}

\begin{center}
\includegraphics[width=.5\textwidth,page=1]{/home/pierre/Documents/RSP/formations/formation_Fa/presentation/figures/2_get_ggplot_data_comment.pdf}
\includegraphics[width=.5\textwidth,page=1]{/home/pierre/Documents/RSP/formations/formation_Fa/presentation/figures/screenshots_template.pdf}
\end{center}

\end{frame}


\begin{frame}
\frametitle{Perspectives}
\framesubtitle{Temps de discussion}

\yo{Quelle vie du groupe "outils pour les MSPs" ?}

\vfill

\begin{itemize}
\item Comment continuer à échanger sur SHiNeMaS 
	\begin{itemize}
	\item au niveau techique
	\item sur les règles usages
	\end{itemize}
\item Quelles organisations ?
\end{itemize}

\end{frame}
