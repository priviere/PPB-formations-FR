\section{Introduction} \begin{frame}\small\tableofcontents[currentsection,currentsubsection,subsectionstyle=show/show/hide]\end{frame}

% Copyright Réseau Semences Paysannes.

% Ce code est sous licence creative commons BY-NC-SA. Vous êtes autorisé à partager et à
% adapter son contenu tant que vous citez les auteurs de ce document et indiquez si des changements
% ont eu lieu, que vous ne faites pas un usage commercial de ce code, tout ou partie du matériel
% le composant, que vous partagez dans les mêmes conditions votre code issu de ce code.

% Pour citer ce code: Cycle de formations sur la gestion dynamique de la biodiversité
% cultivée dans les Maisons des Semences Paysannes en réseau. Code tex de la présentation du RSP, commune à toutes les présentations. Réseau Semences Paysannes.
% Version 1 du 15 janvier 2106. Licence CC BY NC SA.

\subsection{Le \RSP}

\begin{frame}
\frametitle{Le \RSP}
Le \RSP~(RSP) est un réseau constitué de plus de 90 organisations, toutes impliquées dans des initiatives de promotion et de défense de la biodiversité cultivée et des savoir-faire associés.

\vfill

Les grandes actions du \RSP:

\begin{enumerate}
\item la coordination et la consolidation des initiatives locales
\item la promotion de modes de gestion collectifs et de protection des semences paysannes
\item la reconnaissance scientifique et juridique des pratiques paysannes de production et d'échange de semences et de plants
\end{enumerate}

\end{frame}


\subsection{Les \MSPs}


\begin{frame}
\frametitle{Les \MSPs}
\framesubtitle{Principales missions}

Les \MSPs~(MSP) sont des lieux, physiques ou non, qui regroupe un collectif qui oeuvre au maintien et au renouvellement de la diversité cultivée.
Ces maisons sont très diversifiées et partagent, pour la plupart, les missions suivantes:

\begin{enumerate}[<+->]
\item la prospection, à la recherche de variétés anciennes ou locales
\item la  gestion dynamique des semences (conservation, sélection, expérimentation, multiplication, échanges de semences, stockage)
\item échanges de savoirs et savoir-faire
\item la valorisation des semences paysannes ou des produits qui en sont issus
\item la communication
\item l'animation du collectif (gestion des moyens humains, matériels et financiers)
\end{enumerate}

\end{frame}



\begin{frame}
\frametitle{Les \MSPs}

\begin{columns}

\begin{column}{.5\textwidth}
Les \MSPs~au sein du \RSP.

\end{column}

\begin{column}{.5\textwidth}
\includegraphics[width=.95\textwidth, page=1]{msp_objectifs_acteurs} \tiny \cite{msp_objectifs_acteurs}
\end{column}

\end{columns}

\end{frame}


\begin{frame}
\frametitle{Les \MSPs}
\framesubtitle{Pour en savoir plus}

\begin{columns}

\begin{column}{.5\textwidth}
\includegraphics[width=.8\textwidth, page=1]{WEB_ok_Les_MSP} \tiny \cite{WEB_ok_Les_MSP}
\end{column}

\begin{column}{.5\textwidth}
\includegraphics[width=.8\textwidth, page=4]{WEB_ok_Les_MSP} \tiny \cite{WEB_ok_Les_MSP}
\end{column}

\end{columns}

\end{frame}


\subsection{Quelle méthodologie de la sélection?}

\begin{frame}
\frametitle{Quelle méthodologie de la sélection?}

Le programme de sélection est \yo{dynamique} et \yo{récurrent}.
Il se fait dans un réseau d'acteurs connectés. 
Il y a deux grands volets dans un programme de sélection:

\begin{enumerate}

\item \yo{des étapes en routine} dans les \MSPs~ afin de sélectionner des variétés-populations issues de semences paysannes adaptées à la diversité des pratiques:

\begin{itemize}
\item Mobilisation et brassage de la diversité
\item Evaluation et sélection agronomique et organoleptique
\item Production
\end{itemize}

Ces étapes sont construites entre les acteurs au cours du programme et sont parties intégrantes de la gestion de la biodiversité cultivée.

\item \textbf{\color{mln-green}des évaluations bilans} afin de répondre à des questions précises pour évaluer ce qui se passe dans les étapes en routines.

\end{enumerate}

\end{frame}

\begin{frame}
\begin{center}
\includegraphics[width=.95\textwidth, page=1]{methodo-globale} \tiny \cite{methodo-globale}
\end{center}
\end{frame}

\begin{frame}

\begin{tabular} {p{.3\textwidth} p{.3\textwidth} p{.3\textwidth}} 
\includegraphics[width=0.3\textwidth]{sp1.jpg} \tiny \cite{sp1} & \includegraphics[width=0.3\textwidth]{sp2.jpg} \tiny \cite{sp2}  & \includegraphics[width=0.3\textwidth]{sp3.jpg} \tiny \cite{sp3} \\
\end{tabular}

	\begin{block}{}
		Ce type de programme est possible s'il est \textbf{co-construit} au sein d'un collectif d'acteurs : paysans, jardiniers, artisans semenciers, animateurs, équipe de recherche (approche multi-disciplinaire: génétique des populations, génétique quantitative, agronomie, statistique, sociologie, bioinformatique).
	\end{block}		

\begin{tabular} {p{.3\textwidth} p{.3\textwidth} p{.3\textwidth}} 
\includegraphics[width=0.3\textwidth]{sp8.jpg} \tiny \cite{sp8} & \includegraphics[width=0.3\textwidth]{sp5.jpg} \tiny \cite{sp5} & \includegraphics[width=0.3\textwidth]{sp6.jpg} \tiny \cite{sp6} \\
\end{tabular}


\end{frame}


\begin{frame}
\frametitle{Quelle méthodologie de la sélection?}
\framesubtitle{Les étapes en routine}

Trois étapes, en routine dans les \MSPs~
\yo{co-construites} entre les différents acteurs du projet \yo{mis en réseau} et 
participant à la gestion de la biodiversité cultivée: \\

\vfill

\begin{tabular}{p{.3\textwidth} p{.35\textwidth} p{.25\textwidth}}
\hline\hline
\yo{Mobilisation et brassage de la diversité} &
\yo{Evaluation et sélection agronomique et organoleptique} & 
\yo{Production} \\
\hline

\begin{itemize}
\item biodiversité existante
\item mélanges
\item croisements
\end{itemize}
&


\begin{itemize}
\item décentralisation
\item variétés-populations
\end{itemize}

&

\begin{itemize}
\item grains
\item semences
\item produits transformés
\end{itemize}

\\

\hline\hline
\end{tabular}

\end{frame}



\begin{frame}
\frametitle{Quelle méthodologie de la sélection?}
\framesubtitle{Les évaluations bilans}

Des \yo{évaluations bilans} sont nécessaires afin d'étudier ce qui se passe dans ce processus en routine : 
\begin{itemize}
\item évolution
\item adaptation
\item comparaison
\item ...
\end{itemize}

Ces évaluations sont plus contraignantes au niveau expérimental et ont vocation a être effectuées une année ou deux.

Elles répondent à une question précise de recherche.

\end{frame}


\begin{frame}

\begin{columns}

\begin{column}{.4\textwidth}
\begin{enumerate}

\item \yo{Les étapes en routine }

\begin{itemize}
\item la gestion, la mobilisation et le brassage de diversité
\item l'évaluation et la sélection agronomique et organoleptique
\item la production
\end{itemize}

\item \yo{les évaluations bilans }
\end{enumerate}

issues d'une co-construction et d'une mise en réseau des acteurs.


\end{column}

\begin{column}{.6\textwidth}
\includegraphics[width=.95\textwidth, page=1]{methodo-globale} \tiny \cite{methodo-globale}
\end{column}

\end{columns}


\end{frame}




\subsection{Le cycle de formations}

\begin{frame}
\frametitle{Le cycle de formations}

\begin{itemize}

\item \formationA
\item \formationB
\item \formationC
\item \formationD
\item \formationE
\item \yo{\formationFa}
\item \formationFb
\item \formationG
\item \formationH

\end{itemize}

\end{frame}



\begin{frame}
\frametitle{Gérer les données pour les valoriser}

Deux aspects : 
\begin{itemize}
\item technique
\item règles d'usages : cf \yo{\formationFb}
\end{itemize}

\end{frame}

\begin{frame}
\frametitle{Gérer les données pour les valoriser}

Pour mener à bien des programmes de sélection et de gestion des semences, il est important de bien \yo{organiser} et \yo{gérer} ses données.
Cela permet de les \yo{valoriser}.

\vfill

Une base de données permet d'\yo{organiser} les données.

Une interface permet de : 
\begin{itemize}
\item \yo{Gérer} les données : rentrer des fichiers-types, administrer
\item \yo{Valoriser} les données :
\begin{itemize}
\item requêtes
\item analyses
\item fiches et dossiers personnalisés
\end{itemize}
\end{itemize}

\end{frame}

