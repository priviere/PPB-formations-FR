% Copyright Réseau Semences Paysannes.

% Ce code est sous licence creative commons BY-NC-SA. Vous êtes autorisé à partager et à
% adapter son contenu tant que vous citez les auteurs de ce document et indiquez si des changements
% ont eu lieu, que vous ne faites pas un usage commercial de ce code, tout ou partie du matériel
% le composant, que vous partagez dans les mêmes conditions votre code issu de ce code.

% Pour citer ce code: Cycle de formations sur la gestion dynamique de la biodiversité
% cultivée dans les Maisons des Semences Paysannes en réseau. Code tex de la structure des fiches. Réseau Semences Paysannes.
% Version 1 du 15 janvier 2106. Licence CC BY NC SA.

\documentclass[12pt]{book}

\usepackage[top=2cm, bottom=2cm, left=3cm, right=2cm]{geometry} % gérer les marges

\usepackage{titlesec} % Allows customization of titles
%\usepackage[titletoc]{appendix} % To add Appendix into annex section number i.e. Appendix A

% citations %%%%%%%%%%%%%%%%%%%%%%%%%%%%%%%%%%%%%%%%
\usepackage[a4paper=true, colorlinks=true, linkcolor=black,urlcolor=blue, citecolor=black]{hyperref}
%\usepackage[authoryear,round,colon]{natbib}
\usepackage[sorting=none]{biblatex}
\addbibresource{../../ressources/biblio}


%%%%%%%%%%%%%%%%%%%%%%%%%%%%%%%%%%%%%%%%
% Boites à messages
%%%%%%%%%%%%%%%%%%%%%%%%%%%%%%%%%%%%%%%%

% couleurs
\newcommand{\colwarning}{red}
\newcommand{\colloi}{blue}
\newcommand{\colinfosup}{orange}

% les remarques
\setlength{\fboxrule}{1mm}

\newcommand{\RQ}[1]{
~\\
\begin{center}
\fbox{
\begin{minipage}[]{.7\textwidth}
\textcolor{red}{\textbf{{#1}}}
\end{minipage}
}
\end{center}
}

% les warnings
\newcommand{\warning}[1]{
\setlength{\fboxrule}{1.5mm}\noindent
\begin{minipage}[t]{.1\textwidth}
\parbox[c]{\textwidth}{\includegraphics[width=\textwidth]{warning}}
\end{minipage}
\fcolorbox{\colwarning}{white}{
\begin{minipage}[t]{.85\textwidth}
\textcolor{\colwarning}{\textbf{{#1}}}
\end{minipage}
}~\\
}

% les infos sur les textes de loi
\newcommand{\loi}[1]{
\setlength{\fboxrule}{1.5mm}\noindent
\begin{minipage}[t]{.1\textwidth}
\parbox[c]{\textwidth}{\includegraphics[width=\textwidth]{semences_code_barre}}
\end{minipage}
\fcolorbox{\colloi}{white}{
\begin{minipage}[t]{.85\textwidth}
\textcolor{\colloi}{\textbf{{#1}}}
\end{minipage}
}~\\
}

% pour aller plus loin
\newcommand{\infosup}[1]{
\setlength{\fboxrule}{1.5mm}\noindent
\begin{minipage}[t]{.1\textwidth}
\parbox[c]{\textwidth}{\includegraphics[width=\textwidth]{livre}}
\end{minipage}
\fcolorbox{\colinfosup}{white}{
\begin{minipage}[t]{.85\textwidth}
\textcolor{\colinfosup}{\textbf{{#1}}}
\end{minipage}
}~\\
}



% impossible de mettre du verbatim dans une fonction, j'ai pas mal cherché !!!
\newcommand{\TOTO}[3]{
\begin{table}[H]
\begin{tabular}{|p{.5\textwidth} | p{.5\textwidth}|}
\hline
\multicolumn{1}{|c|}{\cellcolor{black} \textcolor{white}{$-$ Interface \BD ~$-$}}
&
\multicolumn{1}{|c|}{\cellcolor{white} \textcolor{black}{$-$ Code \R ~$-$}}
\\
\hline
#1
&
#2
\\
\hline
\multicolumn{2}{|c|}{\cellcolor{gray} \textcolor{white}{$-$ Sorties ~$-$}} \\
\hline
\multicolumn{2}{|l|}{ #3 } \\
\hline
\end{tabular}
\end{table}
}

% custom color of toc


%\titlecontents{section}[1.5cm]
%{\bfseries\color{mln-green}}
%{\contentslabel{1cm}}
%{0cm}
%{}

%
%\titlecontents{subsection}[2.5cm]
%{\color{mln-brown}\bfseries}
%{\color{mln-brown}\contentslabel{1.3cm}}
%{}
%{\color{mln-brown}\titlerule*{}\contentspage}
%
%
%\titlecontents{subsubsection}[3.5cm]
%{\color{gray}\bfseries}
%{\color{gray}\contentslabel{1.7cm}}
%{}
%{\color{gray}\titlerule*{}\contentspage}
%

%\titleformat{\chapter}[hang]{\normalfont\Large\bfseries\color{mln-green}}{\thechapter}{0pt}{~}[~]
%\titleformat{\section}[hang]{\normalfont\Large\bfseries\color{mln-green}}{\thesection.~}{0pt}{}[]
%\titleformat{\subsection}[hang]{\normalfont\large\bfseries\color{mln-brown}}{\hspace{0.8cm}\thesubsection.~}{0pt}{}[]
%\titleformat{\subsubsection}[hang]{\normalfont\normalsize\bfseries\color{gray}}{\hspace{1.9cm}\thesubsubsection.~}{0pt}{}[]
%\titleformat{\paragraph}[hang]{\normalfont\normalsize\bfseries\color{gray}}{}{0pt}{}[]
%\titleformat{\subparagraph}[hang]{\normalfont\normalsize\bfseries\color{gray}}{}{0pt}{}[]

\newcommand{\logoITABMIARSPDEAP}{
\includegraphics[width=0.15\textwidth]{Logo-ITAB}
\hspace{1cm}
\includegraphics[width=0.15\textwidth]{Logo-maiage}
\hspace{1cm}
\includegraphics[height=0.15\textwidth]{Logo-RSP}
\hspace{1cm}
\includegraphics[width=0.15\textwidth]{Logo-UMRGV}
}


\newcommand{\logoITABRSPDEAP}{
\includegraphics[width=0.15\textwidth]{Logo-ITAB}
\hspace{1cm}
\includegraphics[height=0.15\textwidth]{Logo-RSP}
\hspace{1cm}
\includegraphics[width=0.15\textwidth]{Logo-UMRGV}
}

\newcommand{\logoMIARSPDEAP}{
\includegraphics[width=0.15\textwidth]{Logo-maiage}
\hspace{1cm}
\includegraphics[height=0.15\textwidth]{Logo-RSP}
\hspace{1cm}
\includegraphics[width=0.15\textwidth]{Logo-UMRGV}
}

\newcommand{\logoRSPDEAP}{
\includegraphics[height=0.15\textwidth]{Logo-RSP.jpg}
\hspace{1cm}
\includegraphics[width=0.15\textwidth]{Logo-UMRGV.jpg}
}

\newcommand{\logoRSP}{
\includegraphics[height=0.15\textwidth]{Logo-RSP.jpg}
}

\newcommand{\headtitlepagefiche}[2]{

{#1}

\vfill

\colorbox{mln-green}{
\begin{minipage}{\textwidth}
\color{white}
\sffamily\centering\LARGE
\vspace{.4cm}
\titre \\ {#2}
\vspace{.4cm}
\end{minipage}
}

\vfill

}

%----------------------------------------------------------------------------------------
%	MAIN TABLE OF CONTENTS
%----------------------------------------------------------------------------------------

%\contentsmargin{0cm} % Removes the default margin
% Chapter text styling

% ca sert à rien ça en fait !!!
%\titleformat*{\chapter}{\normalfont\huge\bfseries\color{black}}
%\titleformat*{\section}{\normalfont\huge\bfseries\color{mln-green}}
%\titleformat*{\subsection}{\normalfont\huge\bfseries\color{mln-brown}}
%\titleformat*{\subsubsection}{\normalfont\huge\bfseries\color{gray}}


\titlecontents{chapter}
{\bfseries\color{black}}
{\color{black}\contentslabel{0.7cm}}
{}
{\color{black}\titlerule*{}\contentspage}


\titlecontents{section}[1.5cm]
{\bfseries\color{mln-green}}
{\contentslabel{1cm}}
{}
{\color{mln-green}\titlerule*{}\contentspage}


\titlecontents{subsection}[2.5cm]
{\color{mln-brown}\bfseries}
{\color{mln-brown}\contentslabel{1.3cm}}
{}
{\color{mln-brown}\titlerule*{}\contentspage}


\titlecontents{subsubsection}[3.5cm]
{\color{subsubsection.color}\bfseries}
{\color{subsubsection.color}\contentslabel{1.7cm}}
{}
{\color{subsubsection.color}\titlerule*{}\contentspage}


% Section text styling
%\titlecontents{section}[1.25cm] % Indentation
%{\color{red}\addvspace{5pt}\sffamily\bfseries} % Spacing and font options for sections
%{\contentslabel[\thecontentslabel]{1.25cm}} % Section number
%{}
%{\sffamily\hfill\thecontentspage} % Page number
%[]

% Subsection text styling
%\titlecontents{subsection}[1.25cm] % Indentation
%{\color{red}\addvspace{1pt}\sffamily\small} % Spacing and font options for subsections
%{\contentslabel[\thecontentslabel]{1.25cm}} % Subsection number
%{}
%{\sffamily\;\titlerule*[.5pc]{.}\;\thecontentspage} % Page number
%[] 


%----------------------------------------------------------------------------------------
%	MINI TABLE OF CONTENTS IN CHAPTER HEADS
%----------------------------------------------------------------------------------------

% Section text styling
\titlecontents{lsection}[0em] % Indendating
{\footnotesize\sffamily} % Font settings
{}
{}
{}

% Subsection text styling
\titlecontents{lsubsection}[.5em] % Indentation
{\normalfont\footnotesize\sffamily} % Font settings
{}
{}
{}

%----------------------------------------------------------------------------------------
%	PAGE HEADERS
%----------------------------------------------------------------------------------------

\pagestyle{fancy}
\renewcommand{\chaptermark}[1]{\markboth{\sffamily\normalsize\bfseries\hspace{.7cm}#1}{}} % Chapter text font settings
\renewcommand{\sectionmark}[1]{\markright{\sffamily\normalsize\thesection\hspace{.3cm}#1}{}} % Section text font settings
\fancyhf{} \fancyhead[LE,RO]{\sffamily\normalsize\thepage} % Font setting for the page number in the header
\fancyhead[LO]{\rightmark} % Print the nearest section name on the left side of odd pages
\fancyhead[RE]{\leftmark} % Print the current chapter name on the right side of even pages
\renewcommand{\headrulewidth}{0.5pt} % Width of the rule under the header
\addtolength{\headheight}{2.5pt} % Increase the spacing around the header slightly
\renewcommand{\footrulewidth}{0pt} % Removes the rule in the footer
\fancypagestyle{plain}{\fancyhead{}\renewcommand{\headrulewidth}{0pt}} % Style for when a plain pagestyle is specified

% Removes the header from odd empty pages at the end of chapters
\makeatletter
\renewcommand{\cleardoublepage}{
\clearpage\ifodd\c@page\else
\hbox{}
\vspace*{\fill}
\thispagestyle{empty}
\newpage
\fi}

		  
%----------------------------------------------------------------------------------------
%	SECTION NUMBERING IN THE MARGIN
%----------------------------------------------------------------------------------------

\makeatletter

%\renewcommand\section{%
 %\def\@seccntformat##1{\csname the##1\endcsname\hspace{1em} \color{mln-green}}
 %\@startsection{\textcolor{mln-green}}
%}

%\renewcommand\subsection{%
% \def\@seccntformat##1{\csname the##1\endcsname\hspace{1em} \color{mln-brown}}
% \@startsection{\textcolor{mln-brown}}
%}

                  
\renewcommand{\section}{\@startsection{section}{1}{\z@}
{-4ex \@plus -1ex \@minus -.4ex}
{1ex \@plus.2ex }
{\normalfont\large\sffamily\bfseries\color{mln-green}}}

\renewcommand{\subsection}{\@startsection {subsection}{2}{\z@}
{-3ex \@plus -0.1ex \@minus -.4ex}
{0.5ex \@plus.2ex }
{\normalfont\sffamily\bfseries\color{mln-brown}}}

\renewcommand{\subsubsection}{\@startsection {subsubsection}{3}{\z@}
{-2ex \@plus -0.1ex \@minus -.2ex}
{0.2ex \@plus.2ex }
{\normalfont\small\sffamily\bfseries\color{gray}}}                        

\renewcommand\paragraph{\@startsection{paragraph}{4}{\z@}
{-2ex \@plus-.2ex \@minus .2ex}
{0.1ex}
{\normalfont\small\sffamily\bfseries\color{gray}}}

\renewcommand\subparagraph{\@startsection{subparagraph}{4}{\z@}
{-2ex \@plus-.2ex \@minus .2ex}
{0.1ex}
{\normalfont\small\sffamily\bfseries\color{gray}}}


%----------------------------------------------------------------------------------------
%	CHAPTER HEADINGS
%----------------------------------------------------------------------------------------

\newcommand{\thechapterimage}{}
\newcommand{\chapterimage}[1]{\renewcommand{\thechapterimage}{#1}}
\def\thechapter{\arabic{chapter}}
\def\@makechapterhead#1{
\thispagestyle{empty}
{\centering \normalfont\sffamily
\ifnum \c@secnumdepth >\m@ne
\if@mainmatter
\startcontents
\begin{tikzpicture}[remember picture,overlay]
\node at (current page.north west)
{\begin{tikzpicture}[remember picture,overlay]

\node[anchor=north west,inner sep=0pt] at (0,0) {\includegraphics[width=\paperwidth]{\thechapterimage}};

\draw[anchor=west] (3cm,-3cm) node [rounded corners=25pt,fill=white,fill opacity=.6,text opacity=1,draw=mln-green,draw opacity=1,line width=2pt,inner sep=15pt]{\huge\sffamily\bfseries\textcolor{black}{ 
\begin{minipage}{17cm}
\thechapter\ .\ #1
\end{minipage}
\vphantom{plPQq}\makebox[22cm]{}}};

\draw[anchor=west] (1cm,-8cm) node [rounded corners=0pt,fill=white,fill opacity=.6,text opacity=1,draw=mln-green,draw opacity=1,line width=0pt,inner sep=15pt]{\huge\sffamily\bfseries\textcolor{mln-green}{ 
\begin{minipage}{17cm}
\printcontents{l}{1}{\setcounter{tocdepth}{1}}
\end{minipage}
\vphantom{plPQq}\makebox[22cm]{}}};

%Commenting the 3 lines below removes the small contents box in the chapter heading
%\draw[fill=white,opacity=.6] (1cm,-10cm) rectangle (20cm,-6.5cm);
%\node[anchor=north west] at (1cm,-6cm) {\parbox[t][8cm][t]{18cm}{\color{mln-green}\notmalsize\bfseries\flushleft \printcontents{l}{1}{\setcounter{tocdepth}{1}}}}; % \setcounter{tocdepth}{1}, mettre 2 pour avoir les sous section, etc

%\begin{minipage}
%\parbox[t][8cm][t]{18cm}{\color{mln-green}\notmalsize\bfseries\flushleft %\printcontents{l}{1}{\setcounter{tocdepth}{1}}}
%end{minipage}

\end{tikzpicture}};
\end{tikzpicture}}\par\vspace*{230\p@}
\fi
\fi
}
\def\@makeschapterhead#1{
\thispagestyle{empty}
{\centering \normalfont\sffamily
\ifnum \c@secnumdepth >\m@ne
\if@mainmatter
\startcontents
\begin{tikzpicture}[remember picture,overlay]
\node at (current page.north west)
{\begin{tikzpicture}[remember picture,overlay]
\node[anchor=north west] at (-4pt,4pt) {\includegraphics[width=\paperwidth]{\thechapterimage}};
\draw[anchor=west] (5cm,-9cm) node [rounded corners=25pt,fill=white,opacity=.7,inner sep=15.5pt]{\huge\sffamily\bfseries\textcolor{black}{\vphantom{plPQq}\makebox[22cm]{}}};
\draw[anchor=west] (5cm,-9cm) node [rounded corners=25pt,draw=mln-green,line width=2pt,inner sep=15pt]{\huge\sffamily\bfseries\textcolor{black}{#1\vphantom{plPQq}\makebox[22cm]{}}};
\end{tikzpicture}};
\end{tikzpicture}}\par\vspace*{230\p@}
\fi
\fi
}
\makeatother



% FR %%%%%%%%%%%%%%%%%%%%%%%%%%%%%%%%%%%%%%%%
\usepackage[utf8]{inputenc}
\usepackage[francais]{babel}
\usepackage[cyr]{aeguill}
\usepackage[T1]{fontenc}
\usepackage{lmodern}
% policepar défaut : Computer Modern sans serif
% http://mcclinews.free.fr/latex/introbeamer/elements_contenu.html
% http://deic.uab.es/~iblanes/beamer_gallery/
% http://mcclinews.free.fr/latex/introbeamer/elements_diaporama.html#toc6
% \Huge (géant)
% \huge (énorme)
% \LARGE (très grand)
% \Large (plus grand)
% \large (grand)
% \normalsize (normal)
% \small (petit)
% \footnotesize (assez petit)
% \scriptsize (très petit)
% \tiny (minuscule)

% Maths %%%%%%%%%%%%%%%%%%%%%%%%%%%%%%%%%%%%%%%%%
\usepackage{amsmath}
\setcounter{MaxMatrixCols}{50}

% graphs %%%%%%%%%%%%%%%%%%%%%%%%%%%%%%%%%%%%%%%%
\usepackage{float} % [H] dans figure et ça bouge pas!
\usepackage{graphicx}
\usepackage{tikz} % pour dessiner sur des figures
\usepackage{pstricks} % pour dessiner sur des figures

\usetikzlibrary{shapes,arrows}

\usepackage{wasysym} % pour faire des smileys
\usepackage{pict2e} % pour faire des dessins
\setlength{\unitlength}{1mm} % pour faire des dessins, valeur de référence : 1mm
\usepackage{pdfpages} % pour rentrer des page.pdf


% tableaux %%%%%%%%%%%%%%%%%%%%%%%%%%%%%%%%%%%%%%%%
\usepackage{lscape}
\usepackage{supertabular}

% couleurs %%%%%%%%%%%%%%%%%%%%%%%%%%%%%%%%%%%%%%%%
\usepackage{colortbl}
\usepackage{xcolor}
\definecolor{mln-green}{RGB}{13,109,35}
\definecolor{mln-brown}{RGB}{141,162,69}
\definecolor{jaune}{RGB}{255,255,0}
\definecolor{rose}{RGB}{255,0,255}
\definecolor{marron}{RGB}{136,127,0}
\definecolor{bleu}{RGB}{0,255,255}
\definecolor{vert}{RGB}{0,255,0}

% autres
\usepackage{multicol}


\newcommand\guill[1]{\og #1 \fg}
\renewcommand{\thefootnote}{\roman{footnote}} % pour ne pas confondre avec les notes de bas de page
\newcommand{\BDD}{\texttt{SHiNeMaS}}
\newcommand{\BDDfull}{\texttt{Seed History and Network Management System}}
\newcommand{\R}{\texttt{R}}
\newcommand{\RG}{ressources génétiques}
\newcommand{\agec}{agroécologie}
\newcommand{\SP}{sélection collaborative}
\newcommand{\FR}{fermes régionales}
\newcommand{\FS}{fermes satellites}
\newcommand{\CRB}{Centre de Ressources Biologiques}
\newcommand{\CRBs}{Centres de Ressources Biologiques}
\newcommand{\MSP}{Maison des Semences Paysannes}
\newcommand{\MSPs}{Maisons des Semences Paysannes}
\newcommand{\RSP}{Réseau Semences Paysannes}
\newcommand{\INRA}{Institut National de la Recherche Agronomique}
\newcommand{\ITAB}{Institut Technique de l'Agriculture Biologique}
\newcommand{\exsitu}{\textit{ex-situ}}
\newcommand{\insitu}{\textit{in-situ}}
\newcommand{\COV}{Certificat d'Obtention Végétal}
\newcommand{\yo}[1]{\textbf{\color{mln-green}#1}}


% Titre commun à toutes les fiches et présentation
\newcommand{\titre}{\textbf{La gestion dynamique de la biodiversité cultivée dans les \MSPs~en réseau}}

% Noms des formations
\newcommand{\formationA}{Formation A. Sélection décentralisée et collaborative en réseau}
\newcommand{\descriptifFA}{
Cette formation présente les grandes lignes de la méthodologie des programmes de sélection dynamique et récurrente dans les \MSPs en réseau. 

Elle vise à donner les bases, sans rentrer dans les détails, de grands axes qui pourront être approfondis lors de formations spécifiques. 
La formation prendra comme exemple un programme sur le blé tendre. %, la tomate, le mais ou le châtaignier. 

Points abordés lors de la journée :
\begin{itemize}
\item Les céréales (généalogie, évolution, contexte historique)
\item Mobilisation et le brassage de la biodiversité cultivée

\item Evaluation et sélection
\begin{itemize}
\item Evaluation et sélection agrononomique
\begin{itemize}
\item principes de génétique quantitative et des populations pour mieux comprendre comment
il est possible de sélectionner au niveau agronomique
\item sélection décentralisée et participative en réseau
\end{itemize}
\item Evaluation et sélection organoleptique et nutritionelle
\end{itemize} 

\item Production

\item Mise en réseau des acteurs

\item Evaluation du programme et les premiers résultats

\item Règles et droits d’usage collectifs
\end{itemize}

Une visite dans les champs ou dans une ferme pourra permettre de faire une petite pause après le
repas.

Un temps de discussion sera prévu à la fin de la journée afin de réfléchir ensemble à la mise en
place d’un programme de sélection collaborative.
}

\newcommand{\formationB}{Formation B. Gestion, mobilisation et brassage de la diversité}
\newcommand{\descriptifFB}{
Cette formation approfondit les méthodes et les techniques pour gérer, mobiliser et brasser la biodiversité cultivée du blé tendre à la ferme.
Une première partie fait l'état des lieux des méthodes et des techniques en lien avec les dernières 
publications sur le sujet.
Une deuxième partie pratique permettra de réaliser ses propres croisements.
}

\newcommand{\formationC}{Formation C. Evaluation et sélection agronomique}
\newcommand{\descriptifFC}{
En cours de rédaction
}

\newcommand{\formationD}{Formation D. Evaluation et sélection organoleptique}
\newcommand{\descriptifFD}{
En cours de rédaction
}

\newcommand{\formationE}{Formation E. Evaluation bilan}
\newcommand{\descriptifFE}{
En cours de rédaction
}

\newcommand{\formationF}{Formation F. Outils de gestion des données}
\newcommand{\descriptifFF}{
En cours de rédaction
%Cette formation permet de maitriser au niveau technique la base de données \BDD~(\BDDfull).
%Des tutoriels sont proposés sur différentes espèces (céréales, maïs, tomates, oignons, fourragères, arbres).
%Un temps de tests sur les ordinateurs important est prévu.
%Les aspects d'accès à l'information et d'organisation d'un réseau de base de données est traité.
%Les notions de règles et de droits d'usage sont abordés en perspectives.
}

\newcommand{\formationG}{Formation G. Outils d'analyse des données}
\newcommand{\descriptifFG}{
En cours de rédaction
}



\newcommand{\formationHa}{Formation Ha. Règles et droits d'usage collectifs}
\newcommand{\descriptifFHa}{
En cours de rédaction
}


\newcommand{\formationHb}{Formation Hb. Règles d'usage relatives aux otils de gestion et d'analyse des données}
\newcommand{\descriptifFHb}{
En cours de rédaction
%Cette formation présente les règles et droits d'usage liés aux données eu égard de l'organisation d'un collectif et des risques de biopiraterie.
%Une première partie présente les méthodes et les outils utilisés dans les programmes de sélection.
%Ensuite, les risques de biopiraterie associés aux données est abordé.
%Enfin des éléments sont données pour se poser des questions au sein d'un collectif eu égard de la gestion des données. 
%Un temps important de travail de groupe est prévu.
}
% Copyright Réseau Semences Paysannes.

% Ce code est sous licence creative commons BY-NC-SA. Vous êtes autorisé à partager et à
% adapter son contenu tant que vous citez les auteurs de ce document et indiquez si des changements
% ont eu lieu, que vous ne faites pas un usage commercial de ce code, tout ou partie du matériel
% le composant, que vous partagez dans les mêmes conditions votre code issu de ce code.

% Pour citer ce code: Cycle de formations sur la gestion dynamique de la biodiversité
% cultivée dans les Maisons des Semences Paysannes en réseau. Code tex de la licence et du copyright. Réseau Semences Paysannes.
% Version 1 du 15 janvier 2106. Licence CC BY NC SA.

\newcommand{\licencefiche}[3]{

\begin{center}
Copyright #2.

~\\

\href{http://creativecommons.org/licenses/by-nc-sa/4.0/deed.fr}{\includegraphics[width=.2\textwidth]{cc-by-nc-sa}}
\end{center}

\small
Ce chapitre est sous licence creative commons BY-NC-SA.
Vous êtes autorisé à partager et à adapter son contenu tant 
que vous citez les auteurs de ce chapitre et indiquez si des changements ont eu lieu, 
que vous ne faites pas un usage commercial de ce chapitre, tout ou partie du matériel le composant,
que vous partagez dans les mêmes conditions votre travail issu de ce chapitre. 
Plus d'informations ici: \url{creativecommons.org/licenses/by-nc-sa/4.0/deed.fr}.

\vfill

\yo{Pour citer ce chapitre}:
\textit{
Cycle de formations sur la gestion dynamique de la biodiversité cultivée dans les
\MSPs~en réseau.
#1}.
#2.
Version #3.
Fiche technique.
Licence CC BY NC SA.
}


\newcommand{\licencepresentation}[3]{
\begin{frame}

\small

Copyright #2.

\vfill

Ce document est sous licence creative commons BY-NC-SA.

\begin{center}
\href{http://creativecommons.org/licenses/by-nc-sa/4.0/deed.fr}{\includegraphics[width=.25\textwidth]{cc-by-nc-sa}}
\end{center}

Vous êtes autorisé à partager et à adapter son contenu tant 
que vous citez les auteurs de ce document et indiquez si des changements ont eu lieu, 
que vous ne faites pas un usage commercial de ce document, tout ou partie du matériel le composant,
que vous partagez dans les mêmes conditions votre travail issu de ce document. 
Plus d'informations \href{http://creativecommons.org/licenses/by-nc-sa/4.0/deed.fr}{ici}: \url{creativecommons.org/licenses/by-nc-sa/4.0/deed.fr}.

\vfill

\yo{Pour citer ce document}:
\textit{
Cycle de formations sur la gestion dynamique de la biodiversité cultivée dans les
\MSPs~en réseau.
#1}.
#2.
Version #3.
Présentation.
Licence CC BY NC SA.

\end{frame}

}


\graphicspath{{../ressources/figures/}}
%\newcommand{\versionCF}{1}
%\newcommand{\dateversionCF}{21 mars 2016}
% premier jet du texte

%\newcommand{\versionCF}{2}
%\newcommand{\dateversionCF}{1 juillet 2016}
% mise à jour des noms de versions et du texte

\newcommand{\versionCF}{3}
\newcommand{\dateversionCF}{A METTRE A JOUR}
% - mise à jour des noms de versions et du texte
% - ajout de l'intro des présentations
% - mise en forme de la biblio





\input{../formation_A/versionFA.tex}

%%%%%%%%%%%%%%%%%%%%%%%%%%%%%%%%%%%%%%%%%%%%%%%%%%%%%%%%%%%%%%%%%%%%%%%%%%%%%%%%%%%%%%%%%%%%%%%%
%\newcommand{\version}{1.0.}
%\newcommand{\dateversion}{26 mai 2015}
% Présentation: Version de départ
% Fiche: Titres des différentes parties
%%%%%%%%%%%%%%%%%%%%%%%%%%%%%%%%%%%%%%%%%%%%%%%%%%%%%%%%%%%%%%%%%%%%%%%%%%%%%%%%%%%%%%%%%%%%%%%%

%%%%%%%%%%%%%%%%%%%%%%%%%%%%%%%%%%%%%%%%%%%%%%%%%%%%%%%%%%%%%%%%%%%%%%%%%%%%%%%%%%%%%%%%%%%%%%%%
%\newcommand{\version}{1.0.1.}
%\newcommand{\dateversion}{10 juin 2015}
% Présentation: 
% Mise à jour de la diapo 94 avec plus d'illusatrations
% Suppression du sous titre diapo 41
%%%%%%%%%%%%%%%%%%%%%%%%%%%%%%%%%%%%%%%%%%%%%%%%%%%%%%%%%%%%%%%%%%%%%%%%%%%%%%%%%%%%%%%%%%%%%%%%

%%%%%%%%%%%%%%%%%%%%%%%%%%%%%%%%%%%%%%%%%%%%%%%%%%%%%%%%%%%%%%%%%%%%%%%%%%%%%%%%%%%%%%%%%%%%%%%%
\newcommand{\versionFB}{2}
\newcommand{\dateversionFB}{17 juin 2015}
% Présentation: 
% - Ajouts des forces évolutives
% Mise à jour schéma méthodo
%%%%%%%%%%%%%%%%%%%%%%%%%%%%%%%%%%%%%%%%%%%%%%%%%%%%%%%%%%%%%%%%%%%%%%%%%%%%%%%%%%%%%%%%%%%%%%%%


%%%%%%%%%%%%%%%%%%%%%%%%%%%%%%%%%%%%%%%%%%%%%%%%%%%%%%%%%%%%%%%%%%%%%%%%%%%%%%%%%%%%%%%%%%%%%%%%
% to do list

% Voir à appeler avec un input dans formation 1?  mais avec qql diapos en moins, plus, qui change: à voir ce qui est faisable pour ne pas s'emmerder à refaire des copier coller / mise à jour

% - pour lamarak: exemple d'une plante malade qui reste et n'est plus malade après (demander à isa si ok: non a priori car sur temps court a voir sur la tomate ?!?)

% pratique exsitu dans les CRB (cf Mathieu)
% résultats julie suite à FSO (ACP et on différentie toujours les pops)
% résultats isa et jerome sur GD en station cf enjalbert 2011

% fonction R pour mixture  CCP PS => faire des simuls (cf thèse gaelle) en lien avec taille des pops (f pour partie genet des pops dans formation sélection)

% parler des autres méthodes comme les OGM et la mutagénèse ?

% Mieux coller le nom des auteurs de photos! Faire une fonction pour ça? Idem pour formation 1, ds structure_common

% mettre les liens des sites de julien de cetab des banques de graines
%Pour info , je vous donne le lien vers une association anglaise qui bosse sur les blés anciens et qui possède un site  qui permet de faire des recherches dans les divers conservatoires mondiaux en les regroupant sur un seul moteur de recherche/

%lien: http://www.brockwell-bake.org.uk/wheat/

%mais aussi une bibliographie sur divers sujets liés au blé (sélection,généalogie, gluten, maladies...)

%lien: http://www.brockwell-bake.org.uk/docs/

%Seul problème tout est quasiment en Anglais donc pas très adapté à la diffusion auprès des paysans surtout qu'il y a pas mal d'études scientifiques dont la lecture est assez ardue...


%%%%%%%%%%%%%%%%%%%%%%%%%%%%%%%%%%%%%%%%%%%%%%%%%%%%%%%%%%%%%%%%%%%%%%%%%%%%%%%%%%%%%%%%%%%%%%%%



%
%%%%%%%%%%%%%%%%%%%%%%%%%%%%%%%%%%%%%%%%%%%%%%%%%%%%%%%%%%%%%%%%%%%%%%%%%%%%%%%%%%%%%%%%%%%%%%%%
\newcommand{\versionFFa}{1}
\newcommand{\dateversionFFa}{6 mars 2017}
% Présentation: Version de départ
%%%%%%%%%%%%%%%%%%%%%%%%%%%%%%%%%%%%%%%%%%%%%%%%%%%%%%%%%%%%%%%%%%%%%%%%%%%%%%%%%%%%%%%%%%%%%%%%

 
%
%%%%%%%%%%%%%%%%%%%%%%%%%%%%%%%%%%%%%%%%%%%%%%%%%%%%%%%%%%%%%%%%%%%%%%%%%%%%%%%%%%%%%%%%%%%%%%%%
\newcommand{\versionFFa}{1}
\newcommand{\dateversionFFa}{6 mars 2017}
% Présentation: Version de départ
%%%%%%%%%%%%%%%%%%%%%%%%%%%%%%%%%%%%%%%%%%%%%%%%%%%%%%%%%%%%%%%%%%%%%%%%%%%%%%%%%%%%%%%%%%%%%%%%

 
%
%%%%%%%%%%%%%%%%%%%%%%%%%%%%%%%%%%%%%%%%%%%%%%%%%%%%%%%%%%%%%%%%%%%%%%%%%%%%%%%%%%%%%%%%%%%%%%%%
\newcommand{\versionFFa}{1}
\newcommand{\dateversionFFa}{6 mars 2017}
% Présentation: Version de départ
%%%%%%%%%%%%%%%%%%%%%%%%%%%%%%%%%%%%%%%%%%%%%%%%%%%%%%%%%%%%%%%%%%%%%%%%%%%%%%%%%%%%%%%%%%%%%%%%

 
%
%%%%%%%%%%%%%%%%%%%%%%%%%%%%%%%%%%%%%%%%%%%%%%%%%%%%%%%%%%%%%%%%%%%%%%%%%%%%%%%%%%%%%%%%%%%%%%%%
\newcommand{\versionFFa}{1}
\newcommand{\dateversionFFa}{6 mars 2017}
% Présentation: Version de départ
%%%%%%%%%%%%%%%%%%%%%%%%%%%%%%%%%%%%%%%%%%%%%%%%%%%%%%%%%%%%%%%%%%%%%%%%%%%%%%%%%%%%%%%%%%%%%%%%

 
%
%%%%%%%%%%%%%%%%%%%%%%%%%%%%%%%%%%%%%%%%%%%%%%%%%%%%%%%%%%%%%%%%%%%%%%%%%%%%%%%%%%%%%%%%%%%%%%%%
\newcommand{\versionFFa}{1}
\newcommand{\dateversionFFa}{6 mars 2017}
% Présentation: Version de départ
%%%%%%%%%%%%%%%%%%%%%%%%%%%%%%%%%%%%%%%%%%%%%%%%%%%%%%%%%%%%%%%%%%%%%%%%%%%%%%%%%%%%%%%%%%%%%%%%

 
%
%%%%%%%%%%%%%%%%%%%%%%%%%%%%%%%%%%%%%%%%%%%%%%%%%%%%%%%%%%%%%%%%%%%%%%%%%%%%%%%%%%%%%%%%%%%%%%%%
\newcommand{\versionFFa}{1}
\newcommand{\dateversionFFa}{6 mars 2017}
% Présentation: Version de départ
%%%%%%%%%%%%%%%%%%%%%%%%%%%%%%%%%%%%%%%%%%%%%%%%%%%%%%%%%%%%%%%%%%%%%%%%%%%%%%%%%%%%%%%%%%%%%%%%

 
%
%%%%%%%%%%%%%%%%%%%%%%%%%%%%%%%%%%%%%%%%%%%%%%%%%%%%%%%%%%%%%%%%%%%%%%%%%%%%%%%%%%%%%%%%%%%%%%%%
\newcommand{\versionFFa}{1}
\newcommand{\dateversionFFa}{6 mars 2017}
% Présentation: Version de départ
%%%%%%%%%%%%%%%%%%%%%%%%%%%%%%%%%%%%%%%%%%%%%%%%%%%%%%%%%%%%%%%%%%%%%%%%%%%%%%%%%%%%%%%%%%%%%%%%

 


\begin{document}
\pagestyle{empty}

\begin{center}

\headtitlepagefiche{\logoITABMIARSPDEAP}{Cycle de formations} 

\vfill

\large
Version \versionCF~du \dateversionCF

\vfill

\normalsize

Pierre~Rivière\up{1*} \hspace{.5cm}
Sophie~Pin\up{2} \hspace{.5cm}
Yannick~de~Oliveira\up{2} \hspace{.5cm}
Patrick~de~Kochko\up{1} \hspace{.5cm}
[... les paysans ...]\up{1} \hspace{.5cm}
[... les animateurs locaux ...]\up{1} \hspace{.5cm}
Olivier~David\up{3} \hspace{.5cm}
Camille~Vindras\up{4} \hspace{.5cm}
Isabelle~Goldringer\up{2} \\
\end{center}

\small
\noindent\up{1}~Réseau Semences Paysannes, 3, avenue de la Gare F-47190 Aiguillon, France 

\noindent\up{2}~INRA, Génétique Quantatitive et Evolution, ferme du Moulon F-91190 Gif sur Yvette, France

\noindent\up{3}~INRA, UR 1404 Unité Mathématiques et Informatique Appliquées du Génome à l'Environnement, F-78352 Jouy-en-Josas, France

\noindent\up{4}~ITAB, Ferme Expérimentale 2485 Route des Pécolets, F-26800 Etoile-sur-Rhône, France

\noindent\up{*} \textbf{Contact}: \href{mailto:pierre@semencespaysannes.org}{\textcolor{mln-green} {pierre@semencespaysannes.org}}

\normalsize

\vfill

\begin{center}
Copyright \RSP, \INRA, \ITAB.

~\\

\href{http://creativecommons.org/licenses/by-nc-sa/4.0/deed.fr}{\includegraphics[width=.2\textwidth]{cc-by-nc-sa}}
\end{center}

\small
Ce document est sous licence creative commons BY-NC-SA.
Vous êtes autorisé à partager et à adapter son contenu tant 
que vous citez les auteurs de ce document et indiquez si des changements ont eu lieu, 
que vous ne faites pas un usage commercial de ce document, tout ou partie du matériel le composant,
que vous partagez dans les mêmes conditions votre travail issu de ce document. 
Plus d'informations ici: \url{creativecommons.org/licenses/by-nc-sa/4.0/deed.fr}.

\vfill

\yo{Pour citer ce document}:
\textit{
Cycle de formations sur la gestion dynamique de la biodiversité cultivée dans les
\MSPs~en réseau}.
\RSP, \INRA, \ITAB.
Version \versionCF~du \dateversionCF.
Fiche technique.
Licence CC BY NC SA.


\newpage ~\\ \newpage \tableofcontents \newpage \pagestyle{plain}

\chapter{Présentation du cycle de formation}
\startcontents[chapters]
\printcontents[chapters]{}{1}{}

\section{Auteurs et objectifs de ce cycle de formation}

\subsection{Les auteurs}

Ce cycle de formation est coordoné par le \RSP, en collaboration avec des équipes de l'\INRA~et de l'\ITAB.

\subsubsection{\RSP}
Le \RSP~(RSP) est un réseau constitué de plus de quatre-vingt organisations, toutes impliquées dans des initiatives de promotion et de défense de la biodiversité cultivée et des savoir-faire associés.
Outre la coordination et la consolidation des initiatives locales, le RSP travaille à la promotion de modes de gestion collectifs et de protection des semences paysannes, ainsi qu'à la reconnaissance scientifique et juridique des pratiques paysannes de production et d'échange de semences et de plants.
Plus d'informations sur le site internet: \url{www.semencespaysannes.org}.

\subsubsection{\INRA}
Les équipes 
Diversité Evolution et Adaptation des Populations (DEAP) de l'INRA de Moulon, 
Dynenvie de l'INRA de Jouy en Josas et 
Biodiversité Cultivée et Recherche Participative (BCRP) de l'INRA de Rennes, 
collaborent dans ce projet.

L'équipe DEAP travaille sur la gestion dynamique de la biodiversité cultivée et sa valorisation à travers des agro-écosystèmes durables innovants. 
Ses recherches s'appuient sur l'expérimentation et la modélisation et son espèce modèle est le blé tendre.
Plus d'informations sur le site internet: \url{moulon.inra.fr/index.php/fr/equipes/deap}.

L'équipe Dynenvie travaille sur la modélisation dynamique et statistique pour les écosystèmes, l'épidémiologie et l'agronomie.
Plus d'informations sur le site internet: \url{maiage.jouy.inra.fr/?q=fr/dynenvie}.

L'équipe BCRP travaille sur !!!!! A COMPLETER !!!!!

\subsubsection{\ITAB}
L'\ITAB~(ITAB) est l'institut technique agricole dédié à la recherche-expérimentation en agriculture biologique.
Plus d'informations sur le site internet: \url{http://www.itab.asso.fr/}.


\subsection{Les objectifs}
Les objectifs de cycle de formation sont

\begin{itemize}
\item d'apporter des éléments aux collectifs, animateurs, paysans, techniciens, chercheurs, souhaitant mettre en place des programmes de sélection participative.
\item d'etre un ensemble de documents ressources pour adapter de nouvelles formations selon les demandes des collectifs locaux. C'est pour cela que tous les documents sont sous liences CC BY NC SA.
\end{itemize}

Ce cycle de formations est composé de neufs formations (présenté dans chaque chapitre) qui reprennent abordent les méthodes, les outils pour mettre en place les méthodes et les règles et droits d’usage.
Chacune des formations est composée d’une présentation et d’une fiche technique qui l’accompagne. 
Des documents en annexes permettent d’aller plus dans le détail mais ne pas les consulter ne
compromet pas la compréhension des thèmes abordés. 
Ces documents sont mis à jours régulièrement à partir des retours du terrain, des dernières résultats et de la bilbiographie. 
Un numéro de version et une date permettent de suivre les mises à jour en début de chaque chapitre.

Ce chapitre d'introduction reprend le contexte dans lequel se place la sélection participative.
Ces éléments se retrouvent dans la partie introductive de toutes les présentations.
Tout d'abord, la question de la souveraineté alimentaire, de l'agroécologie et des semences est abordée.
Ensuite, un rapide historique de la sélection est présenté.
Puis les \MSPs, 
la méthodologie de la gestion dynamique de la biodiversité cultivée et de la sélection paysanne et enfin
un descriptif des formations proposées dans chaque chapitre.

\newpage


\section{Souveraineté alimentaire, \agec~et semences}
\subsection{Du concept de sécurité alimentaire à celui de souveraineté alimentaire}

Lors du sommet mondial de l'alimentation à la FAO en novembre 1996, la Déclaration de Rome pose les bases d'une concertation mondiale sur les problèmes de l'alimentation dans le monde \cite{fao_declaration_1996}.

Les parties contractantes s'engagent à mettre en oeuvre et soutenir le Plan d’Action du Sommet Mondial de l’Alimentation et les délégations s'engagent à éradiquer \guill{\textit{la faim dans tous les pays}}.

La Déclaration du forum des Organisations Non Gouvernementales et paysannes à Rome en 1996 rejoint le Plan d’Action du Sommet Mondial de l’Alimentation et pointe \guill{\textit{certaines causes fondamentales de l'insécurité alimentaire}} : 
\begin{enumerate}
\item La globalisation de l'économie mondiale livrée aux multinationales et au modèle de surconsommation qui amène à la \guill{\textit{destruction des économies rurales et à la faillite de l'agriculture familiale}}.
\item L'agriculture industrialisée qui détruit \guill{\textit{l'exploitation traditionnelle et empoisonne la planète et tous les êtres vivants}}
\item Les exportations subventionnées qui %provoquent la réduction du stock mondial de céréales et 
%accroissent \guill{\textit{l'instabilité du marché au détriment des agriculteurs familiaux}}.
\item Le rôle du FMI et de la Banque Mondiale qui obligent \guill{\textit{les agriculteurs familiaux et les personnes vulnérables}} à payer \guill{\textit{le prix de l'ajustement structurel et du remboursement de la dette}}.
\end{enumerate}

Pour répondre à ces difficultés, ces organisations proposent un modèle alternatif \guill{\textit{basé sur la décentralisation}} qui redistribuerait la richesse et le pouvoir et permettrait de parvenir à la sécurité alimentaire et à la protection des écosystèmes qui rendent la vie possible sur terre.

La Déclaration souligne \guill{\textit{six éléments clés de ce modèle alternatif}} : 
aider la capacité à produire des petits producteurs, 
inverser la concentration des richesses et du pouvoir,
développer l'\agec,
améliorer la capacité des états et des gouvernements à garantir la sécurité alimentaire,
renforcer la participation des organisations populaires et des ONG à tous les niveaux
et faire garantir le droit à l'alimentation par le droit international.

C'est de cette Déclaration que le concept de souveraineté alimentaire émerge.
Les questions liées à la gouvernance et à l'autonomie s'ajoutent aux questions liées à la sécurité alimentaire.
La dimension politique du contrôle des richesses et du pouvoir par les populations est au centre du concept de souveraineté.
En d'autres termes, comme le souligne la coordination européenne Via Campesina\footnote{
La Via Campesina est, d'après leur site internet, \guill{\textit{un mouvement international qui rassemble des millions de paysannes et de paysans, de petits et de moyens producteurs, de sans terre, de femmes et de jeunes du monde rural, d'indigènes, de migrants et de travailleurs agricoles ...  Elle défend l'agriculture durable de petite échelle comme moyen de promouvoir la justice sociale et la dignité.}}.} dans un communiqué de 2009 lors du Sommet Mondial sur la Sécurité Alimentaire à la FAO :
\guill{\textit{La construction de la souveraineté alimentaire repose sur la démocratisation de la prise de décision.}}.


\subsection{Etat des lieux et menaces pour la sécurité alimentaire dans le monde}
Selon la FAO \cite{fao_letat_2012}, en 2012, 870 millions de personnes souffrent de la faim dans le monde dont 850 millions dans les pays en voie de développement.

Dans les pays du nord, le système agricole intensif mis en place n'est pas durable.
L'impact sur la sécurité alimentaire est direct et négatif \cite{fao_biodiversity_2010,mea_ecosystems_2005,pimentel_environmental_2005,fao_statistical_2013,iaastd_agriculture_2008}.
Sont dangereusement réduits, les bienfaits rendus par les agrosystèmes à l'humanité : 
\begin{enumerate}
\item équilibre des cycles des nutriments, de la formation des sols, de la pollinisation, du contrôle biologique,
\item régulation du climat et des maladies,
\item approvisionnement en nourriture, eau, bois,
\item et par là même, accès à la culture, à l'esthétisme, à l'éducation. 
\end{enumerate}
Ces services éco-systémiques sont malmenés et la biodiversité recule \cite{mea_ecosystems_2005}.
La production de nourriture est basée sur une dépendance accrue aux intrants tels que les fertilisants chimiques, les produits phyto-sanitaires (herbicides, pesticides, fongicides), à l'irrigation qui provoque salinisation, érosion et abaissement des nappes phréatiques; et également aux antibiotiques utilisés pour les élevages \cite{fao_biodiversity_2010}.

Par ailleurs, pour la société, ce modèle agricole a un coût élevé pour sauvegarder la santé des hommes, des végétaux, des animaux et plus généralement de tout l'écosystème \cite{pimentel_environmental_2005,bommelear_couts_2011}.\\

A ces problèmes liés à l'agriculture non durable s'ajoute le changement climatique qui accentue l'insécurité alimentaire \cite{smith_agriculture_2007}.
La biodiversité diminue et les systèmes biologiques sont déstabilisés \cite{suarez_les_2002}.
Le changement climatique engendre une irrégularité et généralement une baisse des rendements agricoles \cite{smith_agriculture_2007,lobell_climate_2011}.

\subsection{Les alternatives : agroécologie et participation des acteurs}

Il y a un consensus au niveau des rapports internationaux : 
pour atteindre un système agricole qui permette, à la fois, de produire assez de nourriture saine et nutritive, qui s'adapte au changement climatique, qui respecte l'environnement, qui soit durable et fasse participer les différents acteurs aux processus de décisions : l'agriculture doit changer de paradigme \cite{fao_biodiversity_2010,iaastd_agriculture_2008,mea_ecosystems_2005,fao_declaration_2009,unctad_wake_2013}.

Pour réaliser un tel objectif, l'IAASTD - un consortium de 400 experts d'origines et de disciplines très différentes qui ont travaillé pendant quatre ans - appelle à une réorientation des sciences agronomiques vers une approche plus holistique des systèmes \cite{iaastd_agriculture_2008,even_liaastd_2009}.
Il s'agit de prendre en compte la complexité, la diversité, l'aspect pluri-factoriel et multi-fonctionnel des systèmes agricoles.
Selon le rapport de l'IAASTD, les dimensions sociales, culturelles, économiques et écologiques liées aux services éco-systémiques et aussi la diversité des processus d’innovation technologique doivent associer connaissances et savoirs locaux avec les savoirs scientifiques formels. \cite{iaastd_agriculture_2008,fao_report_2010,mea_ecosystems_2005}.

L'IAASTD souligne l'importance de l'accès aux ressources, notamment l'eau et la terre, du rapprochement entre science et société et souligne la nécessité de faire évoluer les modes de gouvernance, afin de renforcer la participation des différents acteurs du monde agricole aux processus de décision et d’évaluation.

Le rapport promeut le développement de l'\agec~et des recherches interdisciplinaires \cite{even_liaastd_2009}.
Ce point est corroboré par d'autres rapports de la FAO \cite{fao_international_2007,fao_biodiversity_2010}.


\section{L'histoire de la sélection}
Cette section retrace l'histoire de la sélection depuis la naissance de l'agriculture.
Elle se divise en trois étapes:
\begin{itemize}
\item la naissance de l'agriculture, la gestion et la sélection des semences par les paysan(ne)s pendant plus de 10 000 ans.
\item l'industrialisation de l'agriculture et la professionalisation des semences
\item la réapproprition de la gestion et de la sélection par les collectifs de paysans
\end{itemize}

\subsection{La domestication}

\subsubsection{Exemple des céréales}

\subsubsection{Exemple des tomates}

\subsubsection{Exemple du mais}


%\subsubsection{La domestication}
%
%La domestication est la modification des caractéristiques génétiques ou du comportement d’une espèce sauvage pour l’adapter aux besoins de l’homme. \\
%
%Syndrome de domestication des céréales:
%\begin{itemize}
%\item Grain non caduque
%\item Grain non vêtu
%\item Augmentation de la taille des fruits
%\item Augmentation du rapport fruit/végétation
%\item Synchronisation de la fructification
%\end{itemize}
%
%La domestication est maintenant vue comme une série d’épisodes distincts se produisant à différents moments et différents endroits plutôt qu’une unique « révolution »
%cf Brown et al 2010
%
%%http://fr.wikipedia.org/wiki/Taxonomie_du_bl%C3%A9#Domestication
%
%%cf photo téosinte et mais cultivé
%
%\subsubsection{L'exemple des blés: \textit{turgidum monococcum}}
%\includegraphics[width=.8\textwidth,page=6]{genealogie_ble} \cite{genealogie_ble}
%
%\includegraphics[width=.8\textwidth]{diffusion_monococcum} \cite{diffusion_monococcum}
%
%\subsubsection{L'exemple des blés: \textit{turgidum turgidum}}
%
%\includegraphics[width=.8\textwidth]{diffusion_turgidum} \cite{diffusion_turgidum}
%
%\subsubsection{L'exemple des blés: \textit{turgidum aestivum}}
%
%\includegraphics[width=.8\textwidth]{diffusion_aestivum} \cite{diffusion_aestivum}
%
%\subsubsection{-10000 $\rightarrow$ XIX\up{ème}siècle : les paysans sélectionneurs depuis le néolithique.}
%
%\includegraphics[width=\textwidth]{moisson_egypte} \tiny \cite{moisson_egypte} 
%\includegraphics[width=\textwidth]{moisson_moyen_age1} \tiny \cite{moisson_moyen_age1}
%\includegraphics[width=\textwidth]{moisson_moyen_age2} \tiny \cite{moisson_moyen_age2}
%\includegraphics[width=\textwidth]{moisson_1800} \tiny \cite{moisson_1800}
%\includegraphics[width=\textwidth]{moisson_moyen_age3} \tiny \cite{moisson_moyen_age3}
%
%\subsubsection{XIX\up{ème}siècle  $\rightarrow$ Aujourd'hui : professionnalisation de la sélection et institutionnalisation de la gestion des ressources génétiques.}
%
%\begin{itemize}
%\item Sélection pour une agriculture intensive par des sélectionneurs.\\
%\item Les ressources génétiques comme réservoir dans une gestion \textit{ex-situ}.
%\item Mise en place du catalogue avec les critères de Distinction, d'Homogénéité et de Stabilité (DHS).\\
%Mise en place d'un système de propriété intellectuelle sur le vivant.
%\end{itemize}
%
%
%\includegraphics[page=1,width=.65\textwidth]{bormans} \tiny \cite{bormans} \\
%\includegraphics[width=.95\textwidth]{Chambrefroide} \tiny \cite{Chambrefroide} \\ 
%\includegraphics[width=.95\textwidth]{selection-hierarchisee-site-UPOV-acces-le-2012-09-25} \tiny \cite{selection-hierarchisee-site-UPOV-acces-le-2012-09-25}  \\
%
%\subsubsection{Conséquence de cette évolution}
%
%\begin{itemize}
%\item La biodiversité cultivée diminue %\citep{fao_state_1996,goffaux_quels_2011} :
%		\begin{itemize}
%		\item à l'intérieur des variétés,
%		\item entre les variétés,
%		\item dans les paysages.
%		\end{itemize}
%
%	
%	\includegraphics[width=.5\textwidth]{VP.png} \tiny \cite{VP}
%	\includegraphics[width=.5\textwidth]{VM.png} \tiny \cite{VM}  \\
%		populations hétérogènes, adaptées localement, évolutives
%		variétés homogènes : lignées pures, hybrides
%
%
%
%	\item Il y a un manque de variétés pour les systèmes agroécologiques
%	
%	\end{itemize}
%
%	=> Besoin de développer de nouvelles variétés adaptées localement à la diversité des systèmes agroécologiques et qui contribuent à plus de biodiversité.
%
%\subsubsection{2000  $\rightarrow$ Aujourd'hui : la réappropriation de la sélection et de la gestion des ressources génétiques par la société civile.}
%
%Pour répondre à la diminution de la biodiversité cultivée et à l'uniformisation de la sélection variétale, la société civile s'organise.
%
%La société civile : 
%des paysans,
%des jardiniers,
%des consommateurs,
%des artisans semenciers, etc. % d'autres associés autour de la conservation plus ou moins patrimoniale des variétés locales.
%
%\begin{itemize}
%\item Remise en question du système semencier actuel basé sur le productivisme, la centralisation et la propriété intellectuelle sur le vivant %\citep{demeulenaere_cultiver_2010}.
%\item Besoin de nouvelles variétés adaptées à la diversité des contextes agroécologiques et socio-économiques.
%\end{itemize}
%
%
%Ces acteurs de la société civile se regroupent en associations.
%

Pour aller plus loin: Bonneuil et al


\section{\MSPs~en réseau}

\begin{multicols}{2}
Les \MSPs~sont des lieux, physiques ou non, qui regroupent un collectif qui oeuvre au maintien et au renouvellement de la diversité cultivée.
Ces maisons sont très diversifiées et partagent, pour la plupart, les missions suivantes \cite{rsp_msp_2014}:

\begin{enumerate}
\item la prospection, à la recherche de variétés anciennes ou locales
\item la  gestion dynamique des semences (conservation, sélection, expérimentation, multiplication, échanges de semences, stockage)
\item échanges de savoirs et savoir-faire
\item la valorisation des semences paysannes ou des produits qui en sont issus
\item la communication
\item l'animation du collectif (gestion des moyens humains, matériels et financiers)
\end{enumerate}

Ce document traite particulièrement des points 1, 2, 3 et 6.
Il receuille différentes formations mises en place par le \RSP, l'\ITAB~et l'\INRA~pour accompagner les \MSPs~dans leurs programmes de gestion et de sélection sur différentes espèces telles que 
les céréales à paille, 
la tomate, 
le maïs et 
le châtaignier.

\columnbreak

\begin{figure}[H]
\centering\includegraphics[width=.5\textwidth, page=1]{msp_objectifs_acteurs}
\caption{Les \MSPs~en réseau \cite{msp_objectifs_acteurs}.}
\end{figure}

\end{multicols}


\section{Méthodologie de la gestion dynamique de la biodiversité cultivée et de la sélection paysanne}

La gestion dynamique et la sélection paysanne sont des processus dynamiques et récurrents.
Dynamiques car il ne se produisent pas de manière linéaires et sont en perpétuels mouvements.
Récurrents car certaines étapes se répétent au cours du temps.

Cette vision des choses peut être assez troublantes car on ne sait pas quand le processus se fini, ni quand il commence \dots
La diversité est dynamique et évolue, sa gestion et la sélection dans celle-ci font de même.

Au \RSP, cette sélection se fait dans un réseau d’acteurs connectés: praticiens (paysans, jardiniers, artisans semenciers) animateurs et chercheurs.
Ces acteurs collaborent pour mettre au point des méthodes et des outils pour mener la sélection.

La mise en place de cette méthodologie peut se diviser en deux grands volets (Figure \ref{methodo-globale}):

\begin{enumerate}

\item \textbf{\color{mln-green} des étapes en routine} dans les \MSPs~ afin de sélectionner des variétés paysannes adaptées à la diversité des pratiques:

\begin{itemize}
\item Mobilisation et brassage de la diversité (utilistion de la biodiversité existante, brasage par mélanges et croisements, etc.)
\item Evaluation et sélection agronomique et organoleptique (basée sur la décentralisation de la sélection, l'utilisation de variétés-populations)
\item Production de grains, semences, produits transformés
\end{itemize}

En amont de la sélection, il faut mobiliser la diversité existante, voire la brasser avec des croisements par exemple.
Ensuite, cette diversité peut être sélectionnée au niveau agronomique, organoleptique et nutritionnel. 
Plusieurs années de sélection sont nécessaires pour sélectionner une population sur différents critères.
La sélection peut se faire plusieurs années de suite sur les mêmes critères.
Une population ou un mélange de population sélectionné pourra ensuite être mis en production dans les fermes ou dans les jardins.

Ces différentes étapes sont construites entre les acteurs (paysans, jardiniers, artisans semenceirs, animateurs, chercheurs) au cours du programme et sont parties intégrantes de la gestion de la biodiversité cultivée.


\item \textbf{\color{mln-green}des évaluations bilans} afin de répondre à des questions précises pour évaluer ce qui se passe dans les étapes en routines.
Par exemple évaluer l'adaptation des populations à son terroir et à ses pratiques.
Ces évaluations sont plus contraignantes au niveau expérimental et ont vocation a être effectuées une année ou deux.

\end{enumerate}


\begin{figure}[H]
\centering\includegraphics[width=.8\textwidth, page=1]{methodo-globale}
\caption{Programme de sélection dynamique et récurrente. Les flèches représentent les lots de semences récoltés \cite{methodo-globale}.}
\label{methodo-globale}
\end{figure}

Ce type de programme est possible s'il est \yo{co-construit} au sein d'un collectif d'acteurs: paysans, jardiniers, artisans semenciers, animateurs, chercheurs (approche multi-disciplinaire: génétique des populations, génétique quantitative, agronomie, statistique, sociologie, bioinformatique).

La mise en réseau des acteurs et des \MSPs~est primordiale. 
C'est l'organisation sociale qui permet d'échanger semences, savoirs, savoir-faire et résultats.

Bien sûr, cela n'est qu'une vision et n'est pas représentatif de la diversié des approches de sélection.

\section{Descriptif des formations}

Ce cycle de formations est composé de neufs formations qui reprennent ces différentes étapes et abordent les méthodes (formations A, B, C, D, E), les outils pour mettre en place les méthodes (formations Fa. et G) et les règles et droits d'usage (formations Fb. et H).

Chacune des formations est composée d'une présentation et d'une fiche technique qui l'accompagne. 
Des documents en annexes permettent d'aller plus dans le détail mais ne pas les consulter ne compromet pas la compréhension des thèmes abordés.
Ces documents sont mis à jours régulièrement à partir des retours du terrain, des dernières résultats et de la bilbiographie.
Un numéro de version et une date permettent de suivre les mises à jour.

\subsection{\formationA}
\paragraph{Description} \descriptifFA
\paragraph{Version} Version \versionFA~du \dateversionFA

\subsection{\formationB}
\paragraph{Description} \descriptifFB
\paragraph{Version} Version \versionFB~du \dateversionFB


\subsection{\formationC}
\paragraph{Description} \descriptifFC
%\paragraph{Version} Version \versionFC~du \dateversionFC


\subsection{\formationD}
\paragraph{Description} \descriptifFD
%\paragraph{Version} Version \versionFD~du \dateversionFD

\subsection{\formationE}
\paragraph{Description} \descriptifFE
%\paragraph{Version} Version \versionFE~du \dateversionFE

\subsection{\formationFa}
\paragraph{Description} \descriptifFFa
%\paragraph{Version} Version \versionFFa~du \dateversionFFa

\subsection{\formationFb}
\paragraph{Description} \descriptifFFb
%\paragraph{Version} Version \versionFFb~du \dateversionFFb

\subsection{\formationG}
\paragraph{Description} \descriptifFG
%\paragraph{Version} Version \versionFG~du \dateversionFG

\subsection{\formationH}
\paragraph{Description} \descriptifFH
%\paragraph{Version} Version \versionFH~du \dateversionFH


\section{Licence, copyright et sources}

Tous les documents \texttt{pdf} et les codes \texttt{tex} (qui ont permis de générer les \texttt{pdf} avec \LaTeX) sont sous lience creative commons BY-NC-SA. 
Vous êtes autorisé à partager et à adapter leurs contenus tant que vous citez les auteurs des documents ou des codes et indiquez si des changements ont eu lieu, que vous ne faites pas un usage commercial des documents ou des codes, tout ou partie du matériel le composant, que vous partagez dans les mêmes conditions votre travail issu de ces documents ou ces codes. 
Plus d'informations ici : \url{creativecommons.org/licenses/by-nc-sa/4.0/deed.fr}.
Le copyright est indiqué avec la licence.

~\\

Les \texttt{pdf} des fiches techniques et des formations sont disponibles en téléchargement ici: \url{www.semencespaysannes.org}.
La manière de citer les documents est indiquée au début des fiches et à la fin des présentations.

~\\

Les figures utilisées dans les fiches et les présentations sont disponibles en téléchargement au format \texttt{odp}, \texttt{pdf} ou \texttt{png} ici: \url{www.semencespaysannes.org}.
Les conditions d'utilisation (licence et citation) de ces figures sont dans le fichier \texttt{photos\_images\_figures\_fiche.pdf} téléchargeable ici:  \url{www.semencespaysannes.org}.
~\\

Les fichiers \texttt{tex} sont disponibles en téléchargement ici: \url{github/priviere/formations_MSP}.
La manière de citer les codes est indiquée au début des fichiers \texttt{tex}.

\section{Collaborer}
Tout le monde est bienvenu pour collaborer à ce cycle de formations et enrichir son contenu afin d'ajouter: 
des espèces,
des résultats,
des témoignages,
de la bibliographie, ...

L'idéal est d'envoyer un mail à \href{mailto:pierre@semencespaysannes.org}{\textcolor{mln-green} {pierre@semencespaysannes.org}} afin de se coordoner! Merci!

\chapter{\formationA}
\startcontents[chapters]
\input{../formation_A/fiche/formation_A_fiche}
\newpage
\printcontents[chapters]{}{1}{}
\newpage
\section{Introduction} \begin{frame}\small\tableofcontents[currentsection,currentsubsection,subsectionstyle=show/show/hide]\end{frame}

% Copyright Réseau Semences Paysannes.

% Ce code est sous licence creative commons BY-NC-SA. Vous êtes autorisé à partager et à
% adapter son contenu tant que vous citez les auteurs de ce document et indiquez si des changements
% ont eu lieu, que vous ne faites pas un usage commercial de ce code, tout ou partie du matériel
% le composant, que vous partagez dans les mêmes conditions votre code issu de ce code.

% Pour citer ce code: Cycle de formations sur la gestion dynamique de la biodiversité
% cultivée dans les Maisons des Semences Paysannes en réseau. Code tex de la présentation du RSP, commune à toutes les présentations. Réseau Semences Paysannes.
% Version 1 du 15 janvier 2106. Licence CC BY NC SA.

\subsection{Le \RSP}

\begin{frame}
\frametitle{Le \RSP}
Le \RSP~(RSP) est un réseau constitué de plus de 90 organisations, toutes impliquées dans des initiatives de promotion et de défense de la biodiversité cultivée et des savoir-faire associés.

\vfill

Les grandes actions du \RSP:

\begin{enumerate}
\item la coordination et la consolidation des initiatives locales
\item la promotion de modes de gestion collectifs et de protection des semences paysannes
\item la reconnaissance scientifique et juridique des pratiques paysannes de production et d'échange de semences et de plants
\end{enumerate}

\end{frame}


\subsection{Les \MSPs}


\begin{frame}
\frametitle{Les \MSPs}
\framesubtitle{Principales missions}

Les \MSPs~(MSP) sont des lieux, physiques ou non, qui regroupe un collectif qui oeuvre au maintien et au renouvellement de la diversité cultivée.
Ces maisons sont très diversifiées et partagent, pour la plupart, les missions suivantes:

\begin{enumerate}[<+->]
\item la prospection, à la recherche de variétés anciennes ou locales
\item la  gestion dynamique des semences (conservation, sélection, expérimentation, multiplication, échanges de semences, stockage)
\item échanges de savoirs et savoir-faire
\item la valorisation des semences paysannes ou des produits qui en sont issus
\item la communication
\item l'animation du collectif (gestion des moyens humains, matériels et financiers)
\end{enumerate}

\end{frame}



\begin{frame}
\frametitle{Les \MSPs}

\begin{columns}

\begin{column}{.5\textwidth}
Les \MSPs~au sein du \RSP.

\end{column}

\begin{column}{.5\textwidth}
\includegraphics[width=.95\textwidth, page=1]{msp_objectifs_acteurs} \tiny \cite{msp_objectifs_acteurs}
\end{column}

\end{columns}

\end{frame}


\begin{frame}
\frametitle{Les \MSPs}
\framesubtitle{Pour en savoir plus}

\begin{columns}

\begin{column}{.5\textwidth}
\includegraphics[width=.8\textwidth, page=1]{WEB_ok_Les_MSP} \tiny \cite{WEB_ok_Les_MSP}
\end{column}

\begin{column}{.5\textwidth}
\includegraphics[width=.8\textwidth, page=4]{WEB_ok_Les_MSP} \tiny \cite{WEB_ok_Les_MSP}
\end{column}

\end{columns}

\end{frame}


\subsection{Quelle méthodologie de la sélection?}

\begin{frame}
\frametitle{Quelle méthodologie de la sélection?}

Le programme de sélection est \yo{dynamique} et \yo{récurrent}.
Il se fait dans un réseau d'acteurs connectés. 
Il y a deux grands volets dans un programme de sélection:

\begin{enumerate}

\item \yo{des étapes en routine} dans les \MSPs~ afin de sélectionner des variétés-populations issues de semences paysannes adaptées à la diversité des pratiques:

\begin{itemize}
\item Mobilisation et brassage de la diversité
\item Evaluation et sélection agronomique et organoleptique
\item Production
\end{itemize}

Ces étapes sont construites entre les acteurs au cours du programme et sont parties intégrantes de la gestion de la biodiversité cultivée.

\item \textbf{\color{mln-green}des évaluations bilans} afin de répondre à des questions précises pour évaluer ce qui se passe dans les étapes en routines.

\end{enumerate}

\end{frame}

\begin{frame}
\begin{center}
\includegraphics[width=.95\textwidth, page=1]{methodo-globale} \tiny \cite{methodo-globale}
\end{center}
\end{frame}

\begin{frame}

\begin{tabular} {p{.3\textwidth} p{.3\textwidth} p{.3\textwidth}} 
\includegraphics[width=0.3\textwidth]{sp1.jpg} \tiny \cite{sp1} & \includegraphics[width=0.3\textwidth]{sp2.jpg} \tiny \cite{sp2}  & \includegraphics[width=0.3\textwidth]{sp3.jpg} \tiny \cite{sp3} \\
\end{tabular}

	\begin{block}{}
		Ce type de programme est possible s'il est \textbf{co-construit} au sein d'un collectif d'acteurs : paysans, jardiniers, artisans semenciers, animateurs, équipe de recherche (approche multi-disciplinaire: génétique des populations, génétique quantitative, agronomie, statistique, sociologie, bioinformatique).
	\end{block}		

\begin{tabular} {p{.3\textwidth} p{.3\textwidth} p{.3\textwidth}} 
\includegraphics[width=0.3\textwidth]{sp8.jpg} \tiny \cite{sp8} & \includegraphics[width=0.3\textwidth]{sp5.jpg} \tiny \cite{sp5} & \includegraphics[width=0.3\textwidth]{sp6.jpg} \tiny \cite{sp6} \\
\end{tabular}


\end{frame}


\begin{frame}
\frametitle{Quelle méthodologie de la sélection?}
\framesubtitle{Les étapes en routine}

Trois étapes, en routine dans les \MSPs~
\yo{co-construites} entre les différents acteurs du projet \yo{mis en réseau} et 
participant à la gestion de la biodiversité cultivée: \\

\vfill

\begin{tabular}{p{.3\textwidth} p{.35\textwidth} p{.25\textwidth}}
\hline\hline
\yo{Mobilisation et brassage de la diversité} &
\yo{Evaluation et sélection agronomique et organoleptique} & 
\yo{Production} \\
\hline

\begin{itemize}
\item biodiversité existante
\item mélanges
\item croisements
\end{itemize}
&


\begin{itemize}
\item décentralisation
\item variétés-populations
\end{itemize}

&

\begin{itemize}
\item grains
\item semences
\item produits transformés
\end{itemize}

\\

\hline\hline
\end{tabular}

\end{frame}



\begin{frame}
\frametitle{Quelle méthodologie de la sélection?}
\framesubtitle{Les évaluations bilans}

Des \yo{évaluations bilans} sont nécessaires afin d'étudier ce qui se passe dans ce processus en routine : 
\begin{itemize}
\item évolution
\item adaptation
\item comparaison
\item ...
\end{itemize}

Ces évaluations sont plus contraignantes au niveau expérimental et ont vocation a être effectuées une année ou deux.

Elles répondent à une question précise de recherche.

\end{frame}


\begin{frame}

\begin{columns}

\begin{column}{.4\textwidth}
\begin{enumerate}

\item \yo{Les étapes en routine }

\begin{itemize}
\item la gestion, la mobilisation et le brassage de diversité
\item l'évaluation et la sélection agronomique et organoleptique
\item la production
\end{itemize}

\item \yo{les évaluations bilans }
\end{enumerate}

issues d'une co-construction et d'une mise en réseau des acteurs.


\end{column}

\begin{column}{.6\textwidth}
\includegraphics[width=.95\textwidth, page=1]{methodo-globale} \tiny \cite{methodo-globale}
\end{column}

\end{columns}


\end{frame}




\subsection{Le cycle de formations}

\begin{frame}
\frametitle{Le cycle de formations}

\begin{itemize}

\item \formationA
\item \formationB
\item \formationC
\item \formationD
\item \formationE
\item \yo{\formationFa}
\item \formationFb
\item \formationG
\item \formationH

\end{itemize}

\end{frame}



\begin{frame}
\frametitle{Gérer les données pour les valoriser}

Deux aspects : 
\begin{itemize}
\item technique
\item règles d'usages : cf \yo{\formationFb}
\end{itemize}

\end{frame}

\begin{frame}
\frametitle{Gérer les données pour les valoriser}

Pour mener à bien des programmes de sélection et de gestion des semences, il est important de bien \yo{organiser} et \yo{gérer} ses données.
Cela permet de les \yo{valoriser}.

\vfill

Une base de données permet d'\yo{organiser} les données.

Une interface permet de : 
\begin{itemize}
\item \yo{Gérer} les données : rentrer des fichiers-types, administrer
\item \yo{Valoriser} les données :
\begin{itemize}
\item requêtes
\item analyses
\item fiches et dossiers personnalisés
\end{itemize}
\end{itemize}

\end{frame}

 
%\input{2.diversite} 
%\input{3.1.agro}
%\input{3.2.agro_decentralisation}
%\input{3.3.agro_varietes}
%\input{3.4.agro_ferme_reseau}
%\input{3.5.organo}
%\input{4.production} 
%\input{5.reseau}
%\input{6.evaluation} 
%\input{7.regles_droits_usage} 
%\input{8.conlusion-perspectives} 

\chapter*{Remerciements}\addcontentsline{toc}{chapter}{Remerciements}
% Copyright Réseau Semences Paysannes.

% Ce code est sous licence creative commons BY-NC-SA. Vous êtes autorisé à partager et à
% adapter son contenu tant que vous citez les auteurs de ce document et indiquez si des changements
% ont eu lieu, que vous ne faites pas un usage commercial de ce code, tout ou partie du matériel
% le composant, que vous partagez dans les mêmes conditions votre code issu de ce code.

% Pour citer ce code: Cycle de formations sur la gestion dynamique de la biodiversité
% cultivée dans les Maisons des Semences Paysannes en réseau. Code tex des remerciements. Réseau Semences Paysannes.
% Version 1 du 15 janvier 2106. Licence CC BY NC SA.

Ce projet est soutenu par la Fondation de France et par le programme du fond européen Horizon 2020  pour la recherche et l'innovation (No 633571, projet DIVERSIFOOD: \url{www.diversifood.eu}).

\begin{center}
\includegraphics[width=.28\textwidth]{Logo-Diversifood} \hspace{.5cm}
\includegraphics[width=.2\textwidth]{Logo-EU} \hspace{.5cm}
\includegraphics[width=.18\textwidth]{Logo-FdF}
\end{center}










\chapter*{Bibliographie}\addcontentsline{toc}{chapter}{Bibliographie}
\printbibliography[type=book,title={\large Livres}]
\printbibliography[type=misc,title={\large Photos, images et figures}]
\printbibliography[type=article,title={\large Articles}]

\end{document}


