\section{Méthodologie de la gestion dynamique de la biodiversité cultivée et de la sélection paysanne}

La gestion dynamique et la sélection paysanne sont des processus dynamiques et récurrents.
Dynamiques car il ne se produisent pas de manière linéaires et sont en perpétuels mouvements.
Récurrents car certaines étapes se répétent au cours du temps.

Cette vision des choses peut être assez troublantes car on ne sait pas quand le processus se fini, ni quand il commence \dots
La diversité est dynamique et évolue, sa gestion et la sélection dans celle-ci font de même.

Au \RSP, cette sélection se fait dans un réseau d’acteurs connectés: praticiens (paysans, jardiniers, artisans semenciers) animateurs et chercheurs.
Ces acteurs collaborent pour mettre au point des méthodes et des outils pour mener la sélection.

La mise en place de cette méthodologie peut se diviser en deux grands volets (Figure \ref{methodo-globale}):

\begin{enumerate}

\item \textbf{\color{mln-green} des étapes en routine} dans les \MSPs~ afin de sélectionner des variétés paysannes adaptées à la diversité des pratiques:

\begin{itemize}
\item Mobilisation et brassage de la diversité (utilistion de la biodiversité existante, brasage par mélanges et croisements, etc.)
\item Evaluation et sélection agronomique et organoleptique (basée sur la décentralisation de la sélection, l'utilisation de variétés-populations)
\item Production de grains, semences, produits transformés
\end{itemize}

En amont de la sélection, il faut mobiliser la diversité existante, voire la brasser avec des croisements par exemple.
Ensuite, cette diversité peut être sélectionnée au niveau agronomique, organoleptique et nutritionnel. 
Plusieurs années de sélection sont nécessaires pour sélectionner une population sur différents critères.
La sélection peut se faire plusieurs années de suite sur les mêmes critères.
Une population ou un mélange de population sélectionné pourra ensuite être mis en production dans les fermes ou dans les jardins.

Ces différentes étapes sont construites entre les acteurs (paysans, jardiniers, artisans semenceirs, animateurs, chercheurs) au cours du programme et sont parties intégrantes de la gestion de la biodiversité cultivée.


\item \textbf{\color{mln-green}des évaluations bilans} afin de répondre à des questions précises pour évaluer ce qui se passe dans les étapes en routines.
Par exemple évaluer l'adaptation des populations à son terroir et à ses pratiques.
Ces évaluations sont plus contraignantes au niveau expérimental et ont vocation a être effectuées une année ou deux.

\end{enumerate}


\begin{figure}[H]
\centering\includegraphics[width=.8\textwidth, page=1]{methodo-globale}
\caption{Programme de sélection dynamique et récurrente. Les flèches représentent les lots de semences récoltés \cite{methodo-globale}.}
\label{methodo-globale}
\end{figure}

Ce type de programme est possible s'il est \yo{co-construit} au sein d'un collectif d'acteurs: paysans, jardiniers, artisans semenciers, animateurs, chercheurs (approche multi-disciplinaire: génétique des populations, génétique quantitative, agronomie, statistique, sociologie, bioinformatique).

La mise en réseau des acteurs et des \MSPs~est primordiale. 
C'est l'organisation sociale qui permet d'échanger semences, savoirs, savoir-faire et résultats.

Bien sûr, cela n'est qu'une vision et n'est pas représentatif de la diversié des approches de sélection.
