\section{Histoire de la sélection}
Cette section retrace l'histoire de la sélection depuis la naissance de l'agriculture.
Elle se divise en trois étapes:
\begin{itemize}
\item la naissance de l'agriculture, la gestion et la sélection des semences par les paysan(ne)s pendant plus de 10 000 ans.
\item l'industrialisation de l'agriculture et la professionalisation des semences
\item la réapproprition de la gestion et de la sélection par les collectifs de paysans
\end{itemize}

\subsection{La domestication}

\subsubsection{Exemple des céréales}

\subsubsection{Exemple des tomates}

\subsubsection{Exemple du mais}


%\subsubsection{La domestication}
%
%La domestication est la modification des caractéristiques génétiques ou du comportement d’une espèce sauvage pour l’adapter aux besoins de l’homme. \\
%
%Syndrome de domestication des céréales:
%\begin{itemize}
%\item Grain non caduque
%\item Grain non vêtu
%\item Augmentation de la taille des fruits
%\item Augmentation du rapport fruit/végétation
%\item Synchronisation de la fructification
%\end{itemize}
%
%La domestication est maintenant vue comme une série d’épisodes distincts se produisant à différents moments et différents endroits plutôt qu’une unique « révolution »
%cf Brown et al 2010
%
%%http://fr.wikipedia.org/wiki/Taxonomie_du_bl%C3%A9#Domestication
%
%%cf photo téosinte et mais cultivé
%
%\subsubsection{L'exemple des blés: \textit{turgidum monococcum}}
%\includegraphics[width=.8\textwidth,page=6]{genealogie_ble} \cite{genealogie_ble}
%
%\includegraphics[width=.8\textwidth]{diffusion_monococcum} \cite{diffusion_monococcum}
%
%\subsubsection{L'exemple des blés: \textit{turgidum turgidum}}
%
%\includegraphics[width=.8\textwidth]{diffusion_turgidum} \cite{diffusion_turgidum}
%
%\subsubsection{L'exemple des blés: \textit{turgidum aestivum}}
%
%\includegraphics[width=.8\textwidth]{diffusion_aestivum} \cite{diffusion_aestivum}
%
%\subsubsection{-10000 $\rightarrow$ XIX\up{ème}siècle : les paysans sélectionneurs depuis le néolithique.}
%
%\includegraphics[width=\textwidth]{moisson_egypte} \tiny \cite{moisson_egypte} 
%\includegraphics[width=\textwidth]{moisson_moyen_age1} \tiny \cite{moisson_moyen_age1}
%\includegraphics[width=\textwidth]{moisson_moyen_age2} \tiny \cite{moisson_moyen_age2}
%\includegraphics[width=\textwidth]{moisson_1800} \tiny \cite{moisson_1800}
%\includegraphics[width=\textwidth]{moisson_moyen_age3} \tiny \cite{moisson_moyen_age3}
%
%\subsubsection{XIX\up{ème}siècle  $\rightarrow$ Aujourd'hui : professionnalisation de la sélection et institutionnalisation de la gestion des ressources génétiques.}
%
%\begin{itemize}
%\item Sélection pour une agriculture intensive par des sélectionneurs.\\
%\item Les ressources génétiques comme réservoir dans une gestion \textit{ex-situ}.
%\item Mise en place du catalogue avec les critères de Distinction, d'Homogénéité et de Stabilité (DHS).\\
%Mise en place d'un système de propriété intellectuelle sur le vivant.
%\end{itemize}
%
%
%\includegraphics[page=1,width=.65\textwidth]{bormans} \tiny \cite{bormans} \\
%\includegraphics[width=.95\textwidth]{Chambrefroide} \tiny \cite{Chambrefroide} \\ 
%\includegraphics[width=.95\textwidth]{selection-hierarchisee-site-UPOV-acces-le-2012-09-25} \tiny \cite{selection-hierarchisee-site-UPOV-acces-le-2012-09-25}  \\
%
%\subsubsection{Conséquence de cette évolution}
%
%\begin{itemize}
%\item La biodiversité cultivée diminue %\citep{fao_state_1996,goffaux_quels_2011} :
%		\begin{itemize}
%		\item à l'intérieur des variétés,
%		\item entre les variétés,
%		\item dans les paysages.
%		\end{itemize}
%
%	
%	\includegraphics[width=.5\textwidth]{VP.png} \tiny \cite{VP}
%	\includegraphics[width=.5\textwidth]{VM.png} \tiny \cite{VM}  \\
%		populations hétérogènes, adaptées localement, évolutives
%		variétés homogènes : lignées pures, hybrides
%
%
%
%	\item Il y a un manque de variétés pour les systèmes agroécologiques
%	
%	\end{itemize}
%
%	=> Besoin de développer de nouvelles variétés adaptées localement à la diversité des systèmes agroécologiques et qui contribuent à plus de biodiversité.
%
%\subsubsection{2000  $\rightarrow$ Aujourd'hui : la réappropriation de la sélection et de la gestion des ressources génétiques par la société civile.}
%
%Pour répondre à la diminution de la biodiversité cultivée et à l'uniformisation de la sélection variétale, la société civile s'organise.
%
%La société civile : 
%des paysans,
%des jardiniers,
%des consommateurs,
%des artisans semenciers, etc. % d'autres associés autour de la conservation plus ou moins patrimoniale des variétés locales.
%
%\begin{itemize}
%\item Remise en question du système semencier actuel basé sur le productivisme, la centralisation et la propriété intellectuelle sur le vivant %\citep{demeulenaere_cultiver_2010}.
%\item Besoin de nouvelles variétés adaptées à la diversité des contextes agroécologiques et socio-économiques.
%\end{itemize}
%
%
%Ces acteurs de la société civile se regroupent en associations.
%

Pour aller plus loin: Bonneuil et al


