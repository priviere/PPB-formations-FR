\section{\MSPs~en réseau}

\begin{multicols}{2}
Les \MSPs~sont des lieux, physiques ou non, qui regroupent un collectif qui oeuvre au maintien et au renouvellement de la diversité cultivée.
Ces maisons sont très diversifiées et partagent, pour la plupart, les missions suivantes \cite{rsp_msp_2014}:

\begin{enumerate}
\item la prospection, à la recherche de variétés anciennes ou locales
\item la  gestion dynamique des semences (conservation, sélection, expérimentation, multiplication, échanges de semences, stockage)
\item échanges de savoirs et savoir-faire
\item la valorisation des semences paysannes ou des produits qui en sont issus
\item la communication
\item l'animation du collectif (gestion des moyens humains, matériels et financiers)
\end{enumerate}

Ce document traite particulièrement des points 1, 2, 3 et 6.
Il receuille différentes formations mises en place par le \RSP, l'\ITAB~et l'\INRA~pour accompagner les \MSPs~dans leurs programmes de gestion et de sélection sur différentes espèces telles que 
les céréales à paille, 
la tomate, 
le maïs et 
le châtaignier.

\columnbreak

\begin{figure}[H]
\centering\includegraphics[width=.5\textwidth, page=1]{msp_objectifs_acteurs}
\caption{Les \MSPs~en réseau \cite{msp_objectifs_acteurs}.}
\end{figure}

\end{multicols}

