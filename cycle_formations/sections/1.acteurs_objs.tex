\section{Auteurs et objectifs de ce cycle de formation}

\subsection{Les auteurs}

Ce cycle de formation est coordoné par le \RSP, en collaboration avec des équipes de l'\INRA~et de l'\ITAB.

\subsubsection{\RSP}
Le \RSP~(RSP) est un réseau constitué de plus de quatre-vingt organisations, toutes impliquées dans des initiatives de promotion et de défense de la biodiversité cultivée et des savoir-faire associés.
Outre la coordination et la consolidation des initiatives locales, le RSP travaille à la promotion de modes de gestion collectifs et de protection des semences paysannes, ainsi qu'à la reconnaissance scientifique et juridique des pratiques paysannes de production et d'échange de semences et de plants.
Plus d'informations sur le site internet: \url{www.semencespaysannes.org}.

\subsubsection{\INRA}
Les équipes 
Diversité Evolution et Adaptation des Populations (DEAP) de l'INRA de Moulon, 
Dynenvie de l'INRA de Jouy en Josas et 
Biodiversité Cultivée et Recherche Participative (BCRP) de l'INRA de Rennes, 
collaborent dans ce projet.

L'équipe DEAP travaille sur la gestion dynamique de la biodiversité cultivée et sa valorisation à travers des agro-écosystèmes durables innovants. 
Ses recherches s'appuient sur l'expérimentation et la modélisation et son espèce modèle est le blé tendre.
Plus d'informations sur le site internet: \url{moulon.inra.fr/index.php/fr/equipes/deap}.

L'équipe Dynenvie travaille sur la modélisation dynamique et statistique pour les écosystèmes, l'épidémiologie et l'agronomie.
Plus d'informations sur le site internet: \url{maiage.jouy.inra.fr/?q=fr/dynenvie}.

L'équipe BCRP travaille sur !!!!! A COMPLETER !!!!!

\subsubsection{\ITAB}
L'\ITAB~(ITAB) est l'institut technique agricole dédié à la recherche-expérimentation en agriculture biologique.
Plus d'informations sur le site internet: \url{http://www.itab.asso.fr/}.


\subsection{Les objectifs}
Les objectifs de cycle de formation sont

\begin{itemize}
\item d'apporter des éléments aux collectifs, animateurs, paysans, techniciens, chercheurs, souhaitant mettre en place des programmes de sélection participative.
\item d'etre un ensemble de documents ressources pour adapter de nouvelles formations selon les demandes des collectifs locaux. C'est pour cela que tous les documents sont sous liences CC BY NC SA.
\end{itemize}

Ce cycle de formations est composé de neufs formations (présenté dans chaque chapitre) qui reprennent abordent les méthodes, les outils pour mettre en place les méthodes et les règles et droits d’usage.
Chacune des formations est composée d’une présentation et d’une fiche technique qui l’accompagne. 
Des documents en annexes permettent d’aller plus dans le détail mais ne pas les consulter ne
compromet pas la compréhension des thèmes abordés. 
Ces documents sont mis à jours régulièrement à partir des retours du terrain, des dernières résultats et de la bilbiographie. 
Un numéro de version et une date permettent de suivre les mises à jour en début de chaque chapitre.

Ce chapitre d'introduction reprend le contexte dans lequel se place la sélection participative.
Ces éléments se retrouvent dans la partie introductive de toutes les présentations.
Tout d'abord, la question de la souveraineté alimentaire, de l'agroécologie et des semences est abordée.
Ensuite, un rapide historique de la sélection est présenté.
Puis les \MSPs, 
la méthodologie de la gestion dynamique de la biodiversité cultivée et de la sélection paysanne et enfin
un descriptif des formations proposées dans chaque chapitre.

