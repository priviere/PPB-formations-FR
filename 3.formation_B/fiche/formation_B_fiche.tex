% Copyright Réseau Semences Paysannes.

% Ce code est sous licence creative commons BY-NC-SA. Vous êtes autorisé à partager et à
% adapter son contenu tant que vous citez les auteurs de ce document et indiquez si des changements
% ont eu lieu, que vous ne faites pas un usage commercial de ce code, tout ou partie du matériel
% le composant, que vous partagez dans les mêmes conditions votre code issu de ce code.

% Pour citer ce code: Cycle de formations sur la gestion dynamique de la biodiversité
% cultivée dans les Maisons des Semences Paysannes en réseau. Code tex de la structure des fiches. Réseau Semences Paysannes.
% Version 1 du 15 janvier 2106. Licence CC BY NC SA.

\documentclass[12pt]{book}

\usepackage[top=2cm, bottom=2cm, left=3cm, right=2cm]{geometry} % gérer les marges

\usepackage{titlesec} % Allows customization of titles
%\usepackage[titletoc]{appendix} % To add Appendix into annex section number i.e. Appendix A

% citations %%%%%%%%%%%%%%%%%%%%%%%%%%%%%%%%%%%%%%%%
\usepackage[a4paper=true, colorlinks=true, linkcolor=black,urlcolor=blue, citecolor=black]{hyperref}
%\usepackage[authoryear,round,colon]{natbib}
\usepackage[sorting=none]{biblatex}
\addbibresource{../../ressources/biblio}


%%%%%%%%%%%%%%%%%%%%%%%%%%%%%%%%%%%%%%%%
% Boites à messages
%%%%%%%%%%%%%%%%%%%%%%%%%%%%%%%%%%%%%%%%

% couleurs
\newcommand{\colwarning}{red}
\newcommand{\colloi}{blue}
\newcommand{\colinfosup}{orange}

% les remarques
\setlength{\fboxrule}{1mm}

\newcommand{\RQ}[1]{
~\\
\begin{center}
\fbox{
\begin{minipage}[]{.7\textwidth}
\textcolor{red}{\textbf{{#1}}}
\end{minipage}
}
\end{center}
}

% les warnings
\newcommand{\warning}[1]{
\setlength{\fboxrule}{1.5mm}\noindent
\begin{minipage}[t]{.1\textwidth}
\parbox[c]{\textwidth}{\includegraphics[width=\textwidth]{warning}}
\end{minipage}
\fcolorbox{\colwarning}{white}{
\begin{minipage}[t]{.85\textwidth}
\textcolor{\colwarning}{\textbf{{#1}}}
\end{minipage}
}~\\
}

% les infos sur les textes de loi
\newcommand{\loi}[1]{
\setlength{\fboxrule}{1.5mm}\noindent
\begin{minipage}[t]{.1\textwidth}
\parbox[c]{\textwidth}{\includegraphics[width=\textwidth]{semences_code_barre}}
\end{minipage}
\fcolorbox{\colloi}{white}{
\begin{minipage}[t]{.85\textwidth}
\textcolor{\colloi}{\textbf{{#1}}}
\end{minipage}
}~\\
}

% pour aller plus loin
\newcommand{\infosup}[1]{
\setlength{\fboxrule}{1.5mm}\noindent
\begin{minipage}[t]{.1\textwidth}
\parbox[c]{\textwidth}{\includegraphics[width=\textwidth]{livre}}
\end{minipage}
\fcolorbox{\colinfosup}{white}{
\begin{minipage}[t]{.85\textwidth}
\textcolor{\colinfosup}{\textbf{{#1}}}
\end{minipage}
}~\\
}



% impossible de mettre du verbatim dans une fonction, j'ai pas mal cherché !!!
\newcommand{\TOTO}[3]{
\begin{table}[H]
\begin{tabular}{|p{.5\textwidth} | p{.5\textwidth}|}
\hline
\multicolumn{1}{|c|}{\cellcolor{black} \textcolor{white}{$-$ Interface \BD ~$-$}}
&
\multicolumn{1}{|c|}{\cellcolor{white} \textcolor{black}{$-$ Code \R ~$-$}}
\\
\hline
#1
&
#2
\\
\hline
\multicolumn{2}{|c|}{\cellcolor{gray} \textcolor{white}{$-$ Sorties ~$-$}} \\
\hline
\multicolumn{2}{|l|}{ #3 } \\
\hline
\end{tabular}
\end{table}
}

% custom color of toc


%\titlecontents{section}[1.5cm]
%{\bfseries\color{mln-green}}
%{\contentslabel{1cm}}
%{0cm}
%{}

%
%\titlecontents{subsection}[2.5cm]
%{\color{mln-brown}\bfseries}
%{\color{mln-brown}\contentslabel{1.3cm}}
%{}
%{\color{mln-brown}\titlerule*{}\contentspage}
%
%
%\titlecontents{subsubsection}[3.5cm]
%{\color{gray}\bfseries}
%{\color{gray}\contentslabel{1.7cm}}
%{}
%{\color{gray}\titlerule*{}\contentspage}
%

%\titleformat{\chapter}[hang]{\normalfont\Large\bfseries\color{mln-green}}{\thechapter}{0pt}{~}[~]
%\titleformat{\section}[hang]{\normalfont\Large\bfseries\color{mln-green}}{\thesection.~}{0pt}{}[]
%\titleformat{\subsection}[hang]{\normalfont\large\bfseries\color{mln-brown}}{\hspace{0.8cm}\thesubsection.~}{0pt}{}[]
%\titleformat{\subsubsection}[hang]{\normalfont\normalsize\bfseries\color{gray}}{\hspace{1.9cm}\thesubsubsection.~}{0pt}{}[]
%\titleformat{\paragraph}[hang]{\normalfont\normalsize\bfseries\color{gray}}{}{0pt}{}[]
%\titleformat{\subparagraph}[hang]{\normalfont\normalsize\bfseries\color{gray}}{}{0pt}{}[]

\newcommand{\logoITABMIARSPDEAP}{
\includegraphics[width=0.15\textwidth]{Logo-ITAB}
\hspace{1cm}
\includegraphics[width=0.15\textwidth]{Logo-maiage}
\hspace{1cm}
\includegraphics[height=0.15\textwidth]{Logo-RSP}
\hspace{1cm}
\includegraphics[width=0.15\textwidth]{Logo-UMRGV}
}


\newcommand{\logoITABRSPDEAP}{
\includegraphics[width=0.15\textwidth]{Logo-ITAB}
\hspace{1cm}
\includegraphics[height=0.15\textwidth]{Logo-RSP}
\hspace{1cm}
\includegraphics[width=0.15\textwidth]{Logo-UMRGV}
}

\newcommand{\logoMIARSPDEAP}{
\includegraphics[width=0.15\textwidth]{Logo-maiage}
\hspace{1cm}
\includegraphics[height=0.15\textwidth]{Logo-RSP}
\hspace{1cm}
\includegraphics[width=0.15\textwidth]{Logo-UMRGV}
}

\newcommand{\logoRSPDEAP}{
\includegraphics[height=0.15\textwidth]{Logo-RSP.jpg}
\hspace{1cm}
\includegraphics[width=0.15\textwidth]{Logo-UMRGV.jpg}
}

\newcommand{\logoRSP}{
\includegraphics[height=0.15\textwidth]{Logo-RSP.jpg}
}

\newcommand{\headtitlepagefiche}[2]{

{#1}

\vfill

\colorbox{mln-green}{
\begin{minipage}{\textwidth}
\color{white}
\sffamily\centering\LARGE
\vspace{.4cm}
\titre \\ {#2}
\vspace{.4cm}
\end{minipage}
}

\vfill

}

%----------------------------------------------------------------------------------------
%	MAIN TABLE OF CONTENTS
%----------------------------------------------------------------------------------------

%\contentsmargin{0cm} % Removes the default margin
% Chapter text styling

% ca sert à rien ça en fait !!!
%\titleformat*{\chapter}{\normalfont\huge\bfseries\color{black}}
%\titleformat*{\section}{\normalfont\huge\bfseries\color{mln-green}}
%\titleformat*{\subsection}{\normalfont\huge\bfseries\color{mln-brown}}
%\titleformat*{\subsubsection}{\normalfont\huge\bfseries\color{gray}}


\titlecontents{chapter}
{\bfseries\color{black}}
{\color{black}\contentslabel{0.7cm}}
{}
{\color{black}\titlerule*{}\contentspage}


\titlecontents{section}[1.5cm]
{\bfseries\color{mln-green}}
{\contentslabel{1cm}}
{}
{\color{mln-green}\titlerule*{}\contentspage}


\titlecontents{subsection}[2.5cm]
{\color{mln-brown}\bfseries}
{\color{mln-brown}\contentslabel{1.3cm}}
{}
{\color{mln-brown}\titlerule*{}\contentspage}


\titlecontents{subsubsection}[3.5cm]
{\color{subsubsection.color}\bfseries}
{\color{subsubsection.color}\contentslabel{1.7cm}}
{}
{\color{subsubsection.color}\titlerule*{}\contentspage}


% Section text styling
%\titlecontents{section}[1.25cm] % Indentation
%{\color{red}\addvspace{5pt}\sffamily\bfseries} % Spacing and font options for sections
%{\contentslabel[\thecontentslabel]{1.25cm}} % Section number
%{}
%{\sffamily\hfill\thecontentspage} % Page number
%[]

% Subsection text styling
%\titlecontents{subsection}[1.25cm] % Indentation
%{\color{red}\addvspace{1pt}\sffamily\small} % Spacing and font options for subsections
%{\contentslabel[\thecontentslabel]{1.25cm}} % Subsection number
%{}
%{\sffamily\;\titlerule*[.5pc]{.}\;\thecontentspage} % Page number
%[] 


%----------------------------------------------------------------------------------------
%	MINI TABLE OF CONTENTS IN CHAPTER HEADS
%----------------------------------------------------------------------------------------

% Section text styling
\titlecontents{lsection}[0em] % Indendating
{\footnotesize\sffamily} % Font settings
{}
{}
{}

% Subsection text styling
\titlecontents{lsubsection}[.5em] % Indentation
{\normalfont\footnotesize\sffamily} % Font settings
{}
{}
{}

%----------------------------------------------------------------------------------------
%	PAGE HEADERS
%----------------------------------------------------------------------------------------

\pagestyle{fancy}
\renewcommand{\chaptermark}[1]{\markboth{\sffamily\normalsize\bfseries\hspace{.7cm}#1}{}} % Chapter text font settings
\renewcommand{\sectionmark}[1]{\markright{\sffamily\normalsize\thesection\hspace{.3cm}#1}{}} % Section text font settings
\fancyhf{} \fancyhead[LE,RO]{\sffamily\normalsize\thepage} % Font setting for the page number in the header
\fancyhead[LO]{\rightmark} % Print the nearest section name on the left side of odd pages
\fancyhead[RE]{\leftmark} % Print the current chapter name on the right side of even pages
\renewcommand{\headrulewidth}{0.5pt} % Width of the rule under the header
\addtolength{\headheight}{2.5pt} % Increase the spacing around the header slightly
\renewcommand{\footrulewidth}{0pt} % Removes the rule in the footer
\fancypagestyle{plain}{\fancyhead{}\renewcommand{\headrulewidth}{0pt}} % Style for when a plain pagestyle is specified

% Removes the header from odd empty pages at the end of chapters
\makeatletter
\renewcommand{\cleardoublepage}{
\clearpage\ifodd\c@page\else
\hbox{}
\vspace*{\fill}
\thispagestyle{empty}
\newpage
\fi}

		  
%----------------------------------------------------------------------------------------
%	SECTION NUMBERING IN THE MARGIN
%----------------------------------------------------------------------------------------

\makeatletter

%\renewcommand\section{%
 %\def\@seccntformat##1{\csname the##1\endcsname\hspace{1em} \color{mln-green}}
 %\@startsection{\textcolor{mln-green}}
%}

%\renewcommand\subsection{%
% \def\@seccntformat##1{\csname the##1\endcsname\hspace{1em} \color{mln-brown}}
% \@startsection{\textcolor{mln-brown}}
%}

                  
\renewcommand{\section}{\@startsection{section}{1}{\z@}
{-4ex \@plus -1ex \@minus -.4ex}
{1ex \@plus.2ex }
{\normalfont\large\sffamily\bfseries\color{mln-green}}}

\renewcommand{\subsection}{\@startsection {subsection}{2}{\z@}
{-3ex \@plus -0.1ex \@minus -.4ex}
{0.5ex \@plus.2ex }
{\normalfont\sffamily\bfseries\color{mln-brown}}}

\renewcommand{\subsubsection}{\@startsection {subsubsection}{3}{\z@}
{-2ex \@plus -0.1ex \@minus -.2ex}
{0.2ex \@plus.2ex }
{\normalfont\small\sffamily\bfseries\color{gray}}}                        

\renewcommand\paragraph{\@startsection{paragraph}{4}{\z@}
{-2ex \@plus-.2ex \@minus .2ex}
{0.1ex}
{\normalfont\small\sffamily\bfseries\color{gray}}}

\renewcommand\subparagraph{\@startsection{subparagraph}{4}{\z@}
{-2ex \@plus-.2ex \@minus .2ex}
{0.1ex}
{\normalfont\small\sffamily\bfseries\color{gray}}}


%----------------------------------------------------------------------------------------
%	CHAPTER HEADINGS
%----------------------------------------------------------------------------------------

\newcommand{\thechapterimage}{}
\newcommand{\chapterimage}[1]{\renewcommand{\thechapterimage}{#1}}
\def\thechapter{\arabic{chapter}}
\def\@makechapterhead#1{
\thispagestyle{empty}
{\centering \normalfont\sffamily
\ifnum \c@secnumdepth >\m@ne
\if@mainmatter
\startcontents
\begin{tikzpicture}[remember picture,overlay]
\node at (current page.north west)
{\begin{tikzpicture}[remember picture,overlay]

\node[anchor=north west,inner sep=0pt] at (0,0) {\includegraphics[width=\paperwidth]{\thechapterimage}};

\draw[anchor=west] (3cm,-3cm) node [rounded corners=25pt,fill=white,fill opacity=.6,text opacity=1,draw=mln-green,draw opacity=1,line width=2pt,inner sep=15pt]{\huge\sffamily\bfseries\textcolor{black}{ 
\begin{minipage}{17cm}
\thechapter\ .\ #1
\end{minipage}
\vphantom{plPQq}\makebox[22cm]{}}};

\draw[anchor=west] (1cm,-8cm) node [rounded corners=0pt,fill=white,fill opacity=.6,text opacity=1,draw=mln-green,draw opacity=1,line width=0pt,inner sep=15pt]{\huge\sffamily\bfseries\textcolor{mln-green}{ 
\begin{minipage}{17cm}
\printcontents{l}{1}{\setcounter{tocdepth}{1}}
\end{minipage}
\vphantom{plPQq}\makebox[22cm]{}}};

%Commenting the 3 lines below removes the small contents box in the chapter heading
%\draw[fill=white,opacity=.6] (1cm,-10cm) rectangle (20cm,-6.5cm);
%\node[anchor=north west] at (1cm,-6cm) {\parbox[t][8cm][t]{18cm}{\color{mln-green}\notmalsize\bfseries\flushleft \printcontents{l}{1}{\setcounter{tocdepth}{1}}}}; % \setcounter{tocdepth}{1}, mettre 2 pour avoir les sous section, etc

%\begin{minipage}
%\parbox[t][8cm][t]{18cm}{\color{mln-green}\notmalsize\bfseries\flushleft %\printcontents{l}{1}{\setcounter{tocdepth}{1}}}
%end{minipage}

\end{tikzpicture}};
\end{tikzpicture}}\par\vspace*{230\p@}
\fi
\fi
}
\def\@makeschapterhead#1{
\thispagestyle{empty}
{\centering \normalfont\sffamily
\ifnum \c@secnumdepth >\m@ne
\if@mainmatter
\startcontents
\begin{tikzpicture}[remember picture,overlay]
\node at (current page.north west)
{\begin{tikzpicture}[remember picture,overlay]
\node[anchor=north west] at (-4pt,4pt) {\includegraphics[width=\paperwidth]{\thechapterimage}};
\draw[anchor=west] (5cm,-9cm) node [rounded corners=25pt,fill=white,opacity=.7,inner sep=15.5pt]{\huge\sffamily\bfseries\textcolor{black}{\vphantom{plPQq}\makebox[22cm]{}}};
\draw[anchor=west] (5cm,-9cm) node [rounded corners=25pt,draw=mln-green,line width=2pt,inner sep=15pt]{\huge\sffamily\bfseries\textcolor{black}{#1\vphantom{plPQq}\makebox[22cm]{}}};
\end{tikzpicture}};
\end{tikzpicture}}\par\vspace*{230\p@}
\fi
\fi
}
\makeatother



% FR %%%%%%%%%%%%%%%%%%%%%%%%%%%%%%%%%%%%%%%%
\usepackage[utf8]{inputenc}
\usepackage[francais]{babel}
\usepackage[cyr]{aeguill}
\usepackage[T1]{fontenc}
\usepackage{lmodern}
% policepar défaut : Computer Modern sans serif
% http://mcclinews.free.fr/latex/introbeamer/elements_contenu.html
% http://deic.uab.es/~iblanes/beamer_gallery/
% http://mcclinews.free.fr/latex/introbeamer/elements_diaporama.html#toc6
% \Huge (géant)
% \huge (énorme)
% \LARGE (très grand)
% \Large (plus grand)
% \large (grand)
% \normalsize (normal)
% \small (petit)
% \footnotesize (assez petit)
% \scriptsize (très petit)
% \tiny (minuscule)

% Maths %%%%%%%%%%%%%%%%%%%%%%%%%%%%%%%%%%%%%%%%%
\usepackage{amsmath}
\setcounter{MaxMatrixCols}{50}

% graphs %%%%%%%%%%%%%%%%%%%%%%%%%%%%%%%%%%%%%%%%
\usepackage{float} % [H] dans figure et ça bouge pas!
\usepackage{graphicx}
\usepackage{tikz} % pour dessiner sur des figures
\usepackage{pstricks} % pour dessiner sur des figures

\usetikzlibrary{shapes,arrows}

\usepackage{wasysym} % pour faire des smileys
\usepackage{pict2e} % pour faire des dessins
\setlength{\unitlength}{1mm} % pour faire des dessins, valeur de référence : 1mm
\usepackage{pdfpages} % pour rentrer des page.pdf


% tableaux %%%%%%%%%%%%%%%%%%%%%%%%%%%%%%%%%%%%%%%%
\usepackage{lscape}
\usepackage{supertabular}

% couleurs %%%%%%%%%%%%%%%%%%%%%%%%%%%%%%%%%%%%%%%%
\usepackage{colortbl}
\usepackage{xcolor}
\definecolor{mln-green}{RGB}{13,109,35}
\definecolor{mln-brown}{RGB}{141,162,69}
\definecolor{jaune}{RGB}{255,255,0}
\definecolor{rose}{RGB}{255,0,255}
\definecolor{marron}{RGB}{136,127,0}
\definecolor{bleu}{RGB}{0,255,255}
\definecolor{vert}{RGB}{0,255,0}

% autres
\usepackage{multicol}


\newcommand\guill[1]{\og #1 \fg}
\renewcommand{\thefootnote}{\roman{footnote}} % pour ne pas confondre avec les notes de bas de page
\newcommand{\BDD}{\texttt{SHiNeMaS}}
\newcommand{\BDDfull}{\texttt{Seed History and Network Management System}}
\newcommand{\R}{\texttt{R}}
\newcommand{\RG}{ressources génétiques}
\newcommand{\agec}{agroécologie}
\newcommand{\SP}{sélection collaborative}
\newcommand{\FR}{fermes régionales}
\newcommand{\FS}{fermes satellites}
\newcommand{\CRB}{Centre de Ressources Biologiques}
\newcommand{\CRBs}{Centres de Ressources Biologiques}
\newcommand{\MSP}{Maison des Semences Paysannes}
\newcommand{\MSPs}{Maisons des Semences Paysannes}
\newcommand{\RSP}{Réseau Semences Paysannes}
\newcommand{\INRA}{Institut National de la Recherche Agronomique}
\newcommand{\ITAB}{Institut Technique de l'Agriculture Biologique}
\newcommand{\exsitu}{\textit{ex-situ}}
\newcommand{\insitu}{\textit{in-situ}}
\newcommand{\COV}{Certificat d'Obtention Végétal}
\newcommand{\yo}[1]{\textbf{\color{mln-green}#1}}


% Titre commun à toutes les fiches et présentation
\newcommand{\titre}{\textbf{La gestion dynamique de la biodiversité cultivée dans les \MSPs~en réseau}}

% Noms des formations
\newcommand{\formationA}{Formation A. Sélection décentralisée et collaborative en réseau}
\newcommand{\descriptifFA}{
Cette formation présente les grandes lignes de la méthodologie des programmes de sélection dynamique et récurrente dans les \MSPs en réseau. 

Elle vise à donner les bases, sans rentrer dans les détails, de grands axes qui pourront être approfondis lors de formations spécifiques. 
La formation prendra comme exemple un programme sur le blé tendre. %, la tomate, le mais ou le châtaignier. 

Points abordés lors de la journée :
\begin{itemize}
\item Les céréales (généalogie, évolution, contexte historique)
\item Mobilisation et le brassage de la biodiversité cultivée

\item Evaluation et sélection
\begin{itemize}
\item Evaluation et sélection agrononomique
\begin{itemize}
\item principes de génétique quantitative et des populations pour mieux comprendre comment
il est possible de sélectionner au niveau agronomique
\item sélection décentralisée et participative en réseau
\end{itemize}
\item Evaluation et sélection organoleptique et nutritionelle
\end{itemize} 

\item Production

\item Mise en réseau des acteurs

\item Evaluation du programme et les premiers résultats

\item Règles et droits d’usage collectifs
\end{itemize}

Une visite dans les champs ou dans une ferme pourra permettre de faire une petite pause après le
repas.

Un temps de discussion sera prévu à la fin de la journée afin de réfléchir ensemble à la mise en
place d’un programme de sélection collaborative.
}

\newcommand{\formationB}{Formation B. Gestion, mobilisation et brassage de la diversité}
\newcommand{\descriptifFB}{
Cette formation approfondit les méthodes et les techniques pour gérer, mobiliser et brasser la biodiversité cultivée du blé tendre à la ferme.
Une première partie fait l'état des lieux des méthodes et des techniques en lien avec les dernières 
publications sur le sujet.
Une deuxième partie pratique permettra de réaliser ses propres croisements.
}

\newcommand{\formationC}{Formation C. Evaluation et sélection agronomique}
\newcommand{\descriptifFC}{
En cours de rédaction
}

\newcommand{\formationD}{Formation D. Evaluation et sélection organoleptique}
\newcommand{\descriptifFD}{
En cours de rédaction
}

\newcommand{\formationE}{Formation E. Evaluation bilan}
\newcommand{\descriptifFE}{
En cours de rédaction
}

\newcommand{\formationF}{Formation F. Outils de gestion des données}
\newcommand{\descriptifFF}{
En cours de rédaction
%Cette formation permet de maitriser au niveau technique la base de données \BDD~(\BDDfull).
%Des tutoriels sont proposés sur différentes espèces (céréales, maïs, tomates, oignons, fourragères, arbres).
%Un temps de tests sur les ordinateurs important est prévu.
%Les aspects d'accès à l'information et d'organisation d'un réseau de base de données est traité.
%Les notions de règles et de droits d'usage sont abordés en perspectives.
}

\newcommand{\formationG}{Formation G. Outils d'analyse des données}
\newcommand{\descriptifFG}{
En cours de rédaction
}



\newcommand{\formationHa}{Formation Ha. Règles et droits d'usage collectifs}
\newcommand{\descriptifFHa}{
En cours de rédaction
}


\newcommand{\formationHb}{Formation Hb. Règles d'usage relatives aux otils de gestion et d'analyse des données}
\newcommand{\descriptifFHb}{
En cours de rédaction
%Cette formation présente les règles et droits d'usage liés aux données eu égard de l'organisation d'un collectif et des risques de biopiraterie.
%Une première partie présente les méthodes et les outils utilisés dans les programmes de sélection.
%Ensuite, les risques de biopiraterie associés aux données est abordé.
%Enfin des éléments sont données pour se poser des questions au sein d'un collectif eu égard de la gestion des données. 
%Un temps important de travail de groupe est prévu.
}
% Copyright Réseau Semences Paysannes.

% Ce code est sous licence creative commons BY-NC-SA. Vous êtes autorisé à partager et à
% adapter son contenu tant que vous citez les auteurs de ce document et indiquez si des changements
% ont eu lieu, que vous ne faites pas un usage commercial de ce code, tout ou partie du matériel
% le composant, que vous partagez dans les mêmes conditions votre code issu de ce code.

% Pour citer ce code: Cycle de formations sur la gestion dynamique de la biodiversité
% cultivée dans les Maisons des Semences Paysannes en réseau. Code tex de la licence et du copyright. Réseau Semences Paysannes.
% Version 1 du 15 janvier 2106. Licence CC BY NC SA.

\newcommand{\licencefiche}[3]{

\begin{center}
Copyright #2.

~\\

\href{http://creativecommons.org/licenses/by-nc-sa/4.0/deed.fr}{\includegraphics[width=.2\textwidth]{cc-by-nc-sa}}
\end{center}

\small
Ce chapitre est sous licence creative commons BY-NC-SA.
Vous êtes autorisé à partager et à adapter son contenu tant 
que vous citez les auteurs de ce chapitre et indiquez si des changements ont eu lieu, 
que vous ne faites pas un usage commercial de ce chapitre, tout ou partie du matériel le composant,
que vous partagez dans les mêmes conditions votre travail issu de ce chapitre. 
Plus d'informations ici: \url{creativecommons.org/licenses/by-nc-sa/4.0/deed.fr}.

\vfill

\yo{Pour citer ce chapitre}:
\textit{
Cycle de formations sur la gestion dynamique de la biodiversité cultivée dans les
\MSPs~en réseau.
#1}.
#2.
Version #3.
Fiche technique.
Licence CC BY NC SA.
}


\newcommand{\licencepresentation}[3]{
\begin{frame}

\small

Copyright #2.

\vfill

Ce document est sous licence creative commons BY-NC-SA.

\begin{center}
\href{http://creativecommons.org/licenses/by-nc-sa/4.0/deed.fr}{\includegraphics[width=.25\textwidth]{cc-by-nc-sa}}
\end{center}

Vous êtes autorisé à partager et à adapter son contenu tant 
que vous citez les auteurs de ce document et indiquez si des changements ont eu lieu, 
que vous ne faites pas un usage commercial de ce document, tout ou partie du matériel le composant,
que vous partagez dans les mêmes conditions votre travail issu de ce document. 
Plus d'informations \href{http://creativecommons.org/licenses/by-nc-sa/4.0/deed.fr}{ici}: \url{creativecommons.org/licenses/by-nc-sa/4.0/deed.fr}.

\vfill

\yo{Pour citer ce document}:
\textit{
Cycle de formations sur la gestion dynamique de la biodiversité cultivée dans les
\MSPs~en réseau.
#1}.
#2.
Version #3.
Présentation.
Licence CC BY NC SA.

\end{frame}

}


\graphicspath{{../../ressources/figures/}}

%%%%%%%%%%%%%%%%%%%%%%%%%%%%%%%%%%%%%%%%%%%%%%%%%%%%%%%%%%%%%%%%%%%%%%%%%%%%%%%%%%%%%%%%%%%%%%%%
\newcommand{\versionFFa}{1}
\newcommand{\dateversionFFa}{6 mars 2017}
% Présentation: Version de départ
%%%%%%%%%%%%%%%%%%%%%%%%%%%%%%%%%%%%%%%%%%%%%%%%%%%%%%%%%%%%%%%%%%%%%%%%%%%%%%%%%%%%%%%%%%%%%%%%

 


\begin{document}

%%%%%%%%%%%%%%%%%%%%%%%%%%%%%%%%%%%%%%%%%%%%%%%%%%%%%%%%%%%%%%%%%%%%%%%%%%%%%%%%%%%%%%%%
% Define block styles for the figure
\tikzstyle{block} = [rectangle, draw, fill=mln-green, text width=1cm, text centered, rounded corners, minimum height=1cm]
\tikzstyle{line} = [draw, -latex']
%%%%%%%%%%%%%%%%%%%%%%%%%%%%%%%%%%%%%%%%%%%%%%%%%%%%%%%%%%%%%%%%%%%%%%%%%%%%%%%%%%%%%%%%


\pagestyle{empty}

\begin{center}

\headtitlepagefiche{\logoRSPDEAP}{\formationB}

\vfill

\large
Version \versionFB~du \dateversionFB

\vfill

\normalsize

Pierre~Rivière\up{1*} \hspace{.5cm}
Sophie~Pin\up{2} \hspace{.5cm}
[... les paysans ...]\up{1} \hspace{.5cm}
[... les animateurs locaux ...]\up{1} \hspace{.5cm}
Mathieu~Thomas\up{2} \hspace{.5cm}
Isabelle~Goldringer\up{2} \\

\end{center}

\small
\noindent\up{1}~Réseau Semences Paysannes, 3, avenue de la Gare F-47190 Aiguillon, France 

\noindent\up{2}~INRA, Génétique Quantatitive et Evolution, ferme du Moulon F-91190 Gif sur Yvette, France

\noindent\up{*} \textbf{Contact}: \href{mailto:pierre@semencespaysannes.org}{\textcolor{mln-green} {pierre@semencespaysannes.org}}

\noindent\textbf{Contributions}: PR a rédigé la fiche tehnique et la présentation à partir des travaux de PR, SP, MT et IG. IG a relu et augmenté les documents.

\normalsize

\vfill

\licencefiche{\formationB}{\RSP, \INRA}{\versionFB~du \dateversionFB}

\newpage ~\\ \newpage \tableofcontents \newpage \pagestyle{plain}


\section*{Descriptif de la formation}
\paragraph{Intervenant}
Pierre Rivière (RSP)
Sophie Pin (INRA le Moulon)

\paragraph{Durée}
Une journée.

\paragraph{Public concerné}
Groupe de paysans, jardiniers, artisans semenciers, animateurs qui souhaitent appronfondir les aspects théorique et pratique dans la gestion, la mobilisation et le brassage de la biodiversité cultivée.

\paragraph{Contenu}

% + cf /home/pierre/Documents/RSP/presentations_posters_resume_(pierre)/2013-01-21 evolutionnary breeding UK

La diversité est nécessaire pour alimenter les programmes de \SP. 
C'est à partir d'elle que sont sélectionnées les nouvelles populations adaptées à la diversité des pratiques agro-écologiques.

\descriptifFB


\begin{itemize}
\item \textbf{9h00-12h00 en salle.} Présentation sur
	\begin{itemize}
	\item la place de la diversité dans les programmes de \SP
	\item la gestion de la biodiversité
		\begin{itemize}
			\item \exsitu
			\item \insitu~(génétique des populations et concept de métapopulation)
		\end{itemize}

	\item la mobilisation de la diversité existante
	\item le brassage de la diversité : théorie et valorisation
		\begin{itemize}
		\item les croisements (loi de Mendel)
			\begin{itemize}
			\item croisements bi-parentaux
			\item croisements multi-parentaux (CCP)			
			\end{itemize}
		
		\item les mélanges
		\item les populations mâles-stériles 
		\item autres méthodes non applicables (et peu souhaitables?) à la ferme 
			\begin{itemize}
			\item blé synthétique
			\item mutagénèse
			\end{itemize}
		\end{itemize}
	\end{itemize}

\item \textbf{12h00-13h30 repas.}

\item \textbf{13h30-17h dans les champs.}
	\begin{itemize}
	\item présentation des fleurs du blés
	\item travaux pratiques sur les croisements directement sur les plantes dans les champs ou sur des plantes en pot	
	\end{itemize}

\end{itemize}

\newpage



\section{Méthodologie de la sélection}

\begin{figure}[H]
\includegraphics[width=\textwidth, page=1]{methodo-globale}
\caption{Programme de sélection dynamique et récurrente}
\end{figure}


\section{Principes de biologie moléculaire}

note sur les marqueurs, comment ça marche

Quels types d'analyse possible : diversité, association, QTL, etc

Il peut y avoir autour de 5 cross-over par chromosome ?!? Combien de cross over par chromosomes par espcèes

simuler le nombre de combi possible

\section{Gestion de la biodiversité cultivée}

\subsection{Gestion \exsitu}

\subsubsection{Dans les \CRBs}

\subsubsection{Dans les \MSPs}


\subsection{Gestion \insitu}

\subsubsection{Dans des réseaux de \CRBs}

cf résultats équipe isa, cf papier jérome 2011

\subsubsection{Dans des réseaux de \MSPs}

résultats abdul et mathieu cf wp2 solibam

résultats de ma thèse


\section{Mobilisation de la biodiversité cultivée}

centre RG tomate
%http://w3.avignon.inra.fr/rg_tomate/presentation_reseau.html


dans \CRBs

dans \MSPs

cf BBD indicateur d'Isabelle

\section{Brassage de la biodiversité cultivée}

\subsection{Les croisements}

\subsubsection{Lois de Mendel}

Mendel est un moine ??? qui a travaillé sur les poids et à découvert les lois de l'hérédité.
C'est à dire les lois qui permettent de prédire la ségrégation des caractères sachant deux parents.
C'est à dire la proportion de chaque parent que l'on retourve dans les générations suivantes.


Ajouter en annexe les fiches que l'on a faite quand on envoi les graines du Moulon chez les paysans


\begin{figure}[H]
\begin{center}
\begin{tikzpicture}[node distance = 2.5cm, auto]
    % F0
    \node [block] (P1) {AA};
    \node [block, right of = P1] (P2) {BB};
	
	% F1
    \node [block, below of = P1, right of = P1] (F1) {AB};
    \path [line] (P1) -- (F1);
    \path [line] (P2) -- (F1);

	% F2
    \node [block, below of = F1, right of = F1] (F2a) {AB};
    \node [block, right of = F2a] (F2b) {AA};
    \node [block, left of = F2a] (F2c) {BA};
    \node [block, left of = F2c] (F2d) {BB};

    \path [line] (F1) -- (F2a);
    \path [line] (F1) -- (F2b);
    \path [line] (F1) -- (F2c);
    \path [line] (F1) -- (F2d);

	% F3
    \node [block, below of = F2b] (F3a) {AA};
    \node [block, below of = F2d] (F3b) {BB};
    
    \node [block, below of = F2a, right of = F2a] (F3c) {AB};
    \node [block, right of = F3c] (F3d) {AA};
    \node [block, left of = F3c] (F3e) {BA};
    \node [block, left of = F3e] (F3f) {BB};

    \path [line] (F2b) -- (F3a);
    \path [line] (F2d) -- (F3b);
    
    \path [line] (F2a) -- (F3c);
    \path [line] (F2a) -- (F3d);
    \path [line] (F2a) -- (F3e);
    \path [line] (F2a) -- (F3f);

\end{tikzpicture}
\end{center}
\caption{Ségregation des allèles suite à un croisement bi-parental}
\label{function_relations}
\end{figure}


graph évolution nombre hétéro avec CCP.R


\subsubsection{Croisements bi-parentaux}

\subsubsection{Croisements multi-parentaux}

Les croisements bi-parentaux sont le fait de croiser plusiers parents deux à deux et de mélanger les graines issues de ces croisements.
Un plan de croisement est présenté dans le tableau \ref{mutli-cross}.
Ce travail est très lourd est peut être réalisé par des équipes de recherche avec qui les collectifs paysans collaborent.

\begin{table}[H]
\begin{tabular}{c|ccccccc}
\hline
& Parent 1 & Parent 2 & Parent 3 & Parent 4 & $\ldots$ & Parent $n-1$ & Parent $n$ \\
\hline
Parent 1 &  & C-$[1]$-$[2]$ & C-$[1]$-$[3]$ & C-$[1]$-$[4]$ & $\ldots$ & C-$[1]$-$[n-1]$ & C-$[1]$-$[n]$\\
Parent 2 &  &  & C-$[2]$-$[3]$ & C-$[2]$-$[4]$ & $\ldots$ & C-$[2]$-$[n-1]$ & C-$[2]$-$[n]$\\
Parent 3 &  &  &  & C-$[3]$-$[4]$ & $\ldots$ & C-$[3]$-$[n-1]$ & C-$[3]$-$[n]$\\
Parent 4 &  &  &  & & $\ldots$ & C-$[4]$-$[n-1]$ & C-$[4]$-$[n]$\\
$\vdots$ & & & & & $\vdots$ & $\vdots$ & $\vdots$ \\
Parent $n-1$ &  &  &  &  &  &  & C-$[n-1]$-$[n]$\\
\hline
\end{tabular}
\caption{Exemple d'un plan de croisements avec $n$ parents. 
C-[1]-[4]: croisement issu du parent 1 et du parent 4.
Il y a en tout $(\frac{n(n-1)}{2})$ croisements réalisés.}
\label{mutli-cross}
\end{table}


\begin{figure}
mettre graph issu de CCP.R
le faire vec knitr et trouver un moyen de les mettre en annexes directement!
\caption{Evolution du nombre de croisements en fonction du nombre de parents utilisés.}
\end{figure}

\subsection{Les mélanges}

cf fonction mixture.R


\subsection{Les populations males-stériles}

cf fonction PS.R 'expliquer le nom historique!)

\subsection{Les autres méthodes qui existent}

Faire des CCP et des croisements localement? Un programme par région?
prendre des parents "locaux" + un peu d'exotique?

cf BBD indicateur d'Isabelle pour choisir les parents. Les multiplier avant de les croiser? Si oui (comme JFB), combien de temps? Croiser directement? (voir ce qu'en pense Isabelle)

Demander à Fred (ITAB) s'il a des réfs là dessus

Refaire ça tout les ??? ans? Car on peut rebrasser des poops issues du programme

En lien avec la recherche pour faire ces croisements
F1 dans la station? A voir avec les collectifs locaux

CCP dans une pluus grande parcelle, cf 10 000 ôur Ne (quel Ne pour les autres espèces? Tomate (demander à Fred, gauthier?)?) à 300 grains au m2 => environ 35 m2


blé résistant à la carie (cf travail hongrie de solibam)

Quid de la sélection clonale pour les arbres? A creuser !

> Salut Isa,
> j'ai qql questions pour préparer la formation
>
> - Combien d'allèles par gène en moyenne chez le blé? Ou un ordre de grandeur?
Aucune idée et je ne pense pas qu'il y ait tellement d'info la-dessus, ca dépend de ce que
tu considère comme un gène ... et comme un allèle (séquence différente ou bien exprimant
une fonction différente).
> - Chez le blé: combien de gènes ? Ou un ordre de grandeur!
Je ne sais pas non plus, désolée, il faut chercher sur internet ...
Tu t'en sert pour quoi faire ?


% Copyright Réseau Semences Paysannes.

% Ce code est sous licence creative commons BY-NC-SA. Vous êtes autorisé à partager et à
% adapter son contenu tant que vous citez les auteurs de ce document et indiquez si des changements
% ont eu lieu, que vous ne faites pas un usage commercial de ce code, tout ou partie du matériel
% le composant, que vous partagez dans les mêmes conditions votre code issu de ce code.

% Pour citer ce code: Cycle de formations sur la gestion dynamique de la biodiversité
% cultivée dans les Maisons des Semences Paysannes en réseau. Code tex des remerciements. Réseau Semences Paysannes.
% Version 1 du 15 janvier 2106. Licence CC BY NC SA.

Ce projet est soutenu par la Fondation de France et par le programme du fond européen Horizon 2020  pour la recherche et l'innovation (No 633571, projet DIVERSIFOOD: \url{www.diversifood.eu}).

\begin{center}
\includegraphics[width=.28\textwidth]{Logo-Diversifood} \hspace{.5cm}
\includegraphics[width=.2\textwidth]{Logo-EU} \hspace{.5cm}
\includegraphics[width=.18\textwidth]{Logo-FdF}
\end{center}









\end{document}

