\paragraph{Intervenant}
Pierre Rivière (RSP)
Sophie Pin (INRA le Moulon)

\paragraph{Durée}
Une journée.

\paragraph{Public concerné}
Groupe de paysans, jardiniers, artisans semenciers, animateurs qui souhaitent appronfondir les aspects théorique et pratique dans la gestion, la mobilisation et le brassage de la biodiversité cultivée.

\paragraph{Contenu}

% + cf /home/pierre/Documents/RSP/presentations_posters_resume_(pierre)/2013-01-21 evolutionnary breeding UK

La diversité est nécessaire pour alimenter les programmes de \SP. 
C'est à partir d'elle que sont sélectionnées les nouvelles populations adaptées à la diversité des pratiques agro-écologiques.

\descriptifFB


\begin{itemize}
\item \textbf{9h00-12h00 en salle.} Présentation sur
	\begin{itemize}
	\item la place de la diversité dans les programmes de \SP
	\item la gestion de la biodiversité
		\begin{itemize}
			\item \exsitu
			\item \insitu~(génétique des populations et concept de métapopulation)
		\end{itemize}

	\item la mobilisation de la diversité existante
	\item le brassage de la diversité : théorie et valorisation
		\begin{itemize}
		\item les croisements (loi de Mendel)
			\begin{itemize}
			\item croisements bi-parentaux
			\item croisements multi-parentaux (CCP)			
			\end{itemize}
		
		\item les mélanges
		\item les populations mâles-stériles 
		\item autres méthodes non applicables (et peu souhaitables?) à la ferme 
			\begin{itemize}
			\item blé synthétique
			\item mutagénèse
			\end{itemize}
		\end{itemize}
	\end{itemize}

\item \textbf{12h00-13h30 repas.}

\item \textbf{13h30-17h dans les champs.}
	\begin{itemize}
	\item présentation des fleurs du blés
	\item travaux pratiques sur les croisements directement sur les plantes dans les champs ou sur des plantes en pot	
	\end{itemize}

\end{itemize}

\newpage
