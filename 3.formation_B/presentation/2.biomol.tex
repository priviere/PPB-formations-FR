\section{Biologie moléculaire}
\begin{frame}\tableofcontents[currentsection,currentsubsection,subsectionstyle=show/show/hide]\end{frame}

\subsection{Principes de biologie moléculaire}
\begin{frame}
\frametitle{Principes de biologie moléculaire}
\framesubtitle{La cellule}

\begin{columns}

\begin{column}{.4\textwidth}

\begin{itemize}
\item La cellule est l'unité fonctionelle des êtres vivants.
\item Elle effectue diverses missions à partir de l'information génétique (ADN). Celles-ci sont modulées par les conditions environmentales par l'intermédiaire des hormones.
\end{itemize}


\end{column}

\begin{column}{.6\textwidth}

\begin{center}
\begin{tabular}{cc}
\includegraphics[width=.9\textwidth]{cellule_vegetal} &  \rotatebox{90}{
\tiny l'Eprouvette © UNIL} \\
\small Structure simplifiée d'une cellule végétale & \\
\end{tabular}
\end{center}

\end{column}

\end{columns}

\end{frame}


\begin{frame}
\frametitle{Principes de biologie moléculaire}
\framesubtitle{L'ADN: sa localisation dans la cellule}

L'information génétique des cellules est inscrite dans l'ADN: l'Acide désoxyribonucléique.

Toutes les cellules d'un même organisme ont le même ADN. L'ADN se situe dans 
le  noyau (sous forme de chromosomes),
les mitochondries et
les chloroplastes.

\begin{center}
\begin{tabular}{cc}
\includegraphics[width=.8\textwidth]{cellule_chromosome_adn} &  \rotatebox{90}{
\href{http://www.bbc.co.uk/schools/gcsebitesize/science/add_aqa_pre_2011/celldivision/celldivision1.shtml}{\tiny BBC}
} \\
\small L'ADN au sein des chromosomes dans le noyau de la cellule & \\
\end{tabular}
\end{center}


\end{frame}


\begin{frame}
\frametitle{Principes de biologie moléculaire}
\framesubtitle{L'ADN: sa structure}

L'ADN est composé de 4 nucléotides: 
\textbf{\Huge A}, 
\textbf{\Huge C}, 
\textbf{\Huge G} et 
\textbf{\Huge T}.

\begin{columns}

\begin{column}{.6\textwidth}
\tiny
ACAAGATGCCATTGTCCCCCGGCCTCCTGCTGCTGCTGCTCTCCGGGG\\
CCACGGCCACCGCTGCCCTGCCCCTGGAGGGTGGCCCCACCGGCCGAG\\
ACAGCGAGCATATGCAGGAAGCGGCAGGAATAAGGAAAAGCAGCCTCC\\
TGACTTTCCTCGCTTGGTGGTTTGAGTGGACCTCCCAGGCCAGTGCCG\\
GGCCCCTCATAGGAGAGGAAGCTCGGGAGGTGGCCAGGCGGCAGGAAG\\
GCGCACCCCCCCAGCAATCCGCGCGCCGGGACAGAATGCCCTGCAGGA\\
ACTTCTTCTGGAAGACCTTCTCCTCCTGCAAATAAAACCTCACCCATG\\
AATGCTCACGCAAGTTTAATTACAGACCTGAAACAAGATGCCATTGTC\\
CCCCGGCCTCCTGCTGCTGCTGCTCTCCGGGGCCACGGCCACCGCTGC\\
CCTGGAGGGTGGCCCCACCGGCCGAGACAGCGAGCATATGCAGGAAGC\\
GGCAGGAATAAGGAAAAGCAGCCTCCTGACTTTCCTCGCTTGGTGGTT\\
TGAGTGGACCTCCCAGGCCAGTGCCGGGCCCCTCATAGGAGAGGAAGC\\
TCGGGAGGTGGCCAGGCGGCAGGAAGGCGCACCCCCCCAGCAATCCGC\\
CTGCAGGAACTTCTTCTGGAAGACCTTCTCCTCCTGCAAATAAAACCT\\
CACCCATGAATGCTCACGCAAGTTTAATTACAGACCTGAAACAAGATG\\
GGCAGGAATAAGGAAAAGCAGCCTCCTGACTTTCCTCGCTTGGTGGTT\\
TGAGTGGACCTCCCAGGCCAGTGCCGGGCCCCTCATAGGAGAGGAAGC\\
TCGGGAGGTGGCCAGGCGGCAGGAAGGCGCACCCCCCCAGCAATCCGC\\

\end{column}

\begin{column}{.4\textwidth}
\begin{center}
\begin{tabular}{cc}
\includegraphics[width=.8\textwidth]{611px-DNA_structure_and_bases_FR} &  \rotatebox{90}{
\href{
http://commons.wikimedia.org/wiki/File:DNA_structure_and_bases_FR.svg?uselang=fr}{\tiny Dosto (d): MesserWoland}
} \\
\small La structure de l'ADN & \\
\end{tabular}
\end{center}
\end{column}

\end{columns}

\end{frame}

\begin{frame}
\frametitle{Principes de biologie moléculaire}
\framesubtitle{L'ADN : sa structure}

\begin{itemize}
\item Un \yo{gène} est une séquence d'ADN constituant une unité d'information (héréditaire) qui est exprimée (ou non) dans la cellule. 

Un gène peut etre lié 
\begin{itemize}
\item à 100\% à un caractère. Ex: la texture des graines de pois.
\item partiellement à un caractère. Plusieurs gènes sont responsables du caractère. Ex: le rendement (QTL: Quantitative Trait Loci).
\end{itemize}


\item Un \yo{allèle} est une des séquences alternatives d'un gène. Un allèle est une version d’un gène.

\item Un \yo{locus} est un point précis, un site, du génome qui peut être occupé par un gène donné.

\end{itemize}

L'ADN code pour l'ARN qui va ensuite coder pour des protéines

\begin{center}
ADN => ARN => Protéines
\end{center}

Il existe de nombreux mécanismes de régulation qui complètent ce schéma ...
% ARNi, méthylation

\end{frame}

\begin{frame}
\frametitle{Principes de biologie moléculaire}
\framesubtitle{L'ADN : son organisation selon les espèces}

Il y a plusieurs chromosomes selon les espèces. \\

On note $n$ le nombre d'exemplaire de chacun des chromosomes.
2$n$ signifie qu'il y a deux exemplaires pour chacun des chromosomes. 
En tout, il y a $K=2n$ chromosomes.\\

\vspace{.5cm}
%Il y a une version d'un gène sur chaque exemplaire de chromosome (i.e. plusieurs allèles).

\begin{tabular}{c p{.14\textwidth} p{.18\textwidth} p{.40\textwidth}}
\hline
Dénomination & Notation &  & Exemple d'espèce \\
\hline
%haploide & 1$n$ & A & champignon ($n$=?) \\
diploide & 2$n$ & AA & petit épeautre (\textit{Triticum monococcum}) ($n=7$, $K=2n=14$) \\
tétraploides & 4$n$ & AA BB & blé dur (\textit{Triticum durum}) ($n=7$, $K=4n=28$)\\
hexaploides  & 6$n$ & AA BB DD & blé tendre (\textit{Triticum æstivum}) ($n=7$, $K=6n=42$)\\
... & ... & ... & ... \\
\hline
\end{tabular}

\end{frame}


\begin{frame}
\frametitle{Principes de biologie moléculaire}
\framesubtitle{L'ADN : son organisation selon les espèces}

Chez une espèce diploide (2$n$):

\begin{itemize}
\item 
\yo{homozygote} : cellule présentant deux allèles identiques à un ou plusieurs loci.
\begin{center}
\yo{\Huge  AA} ou \yo{\Huge  aa}
\end{center}

\item 
\yo{hétérozygote} : cellule  possédant des formes allèliques différentes pour un ou plusieurs loci.
\begin{center}
\yo{\Huge Aa}
\end{center}

\end{itemize}

\yo{Dominant / reccessif}: Si l'un des deux allèles s'exprime et l'autre reste \guill{muet}, on dit que le premier est dominant et l'autre récessif.

\yo{\Large Aa} $\rightarrow$ \yo{\Large A} s'exprime et pas \yo{\Large a}. \yo{\Large A} est dominant, \yo{\Large a} est récessif. \\
\yo{\Large AA} $\rightarrow$ \yo{\Large A} s'exprime.\\
\yo{\Large aa} $\rightarrow$ \yo{\Large a} s'exprime.\\

Il existe des états intermédiaires ...

\end{frame}

\begin{frame}
\frametitle{Principes de biologie moléculaire}
\framesubtitle{La méiose : le brassage de l'information génétique}

La méiose est un mode de division qui fait passer une cellule diploïde (2n) à deux cellules haploïdes (n) : les gamètes. 

\begin{block}{}
\begin{center}
\Large 2$n$ $\rightarrow$ méiose $\rightarrow$ $n$ (pollen ou ovaire)
\end{center}
\end{block}

\only<1>{
\begin{center}
\begin{tabular}{ccc}
\hline
\Large 2$n$ & \Large $\rightarrow$ & \Large  $n$ \\
\hline
\yo{\Large Aa} (hétérozygote) &  & \yo{\Large A} ou \yo{\Large a} \\
\yo{\Large AA} (homozygote) &  & \yo{\Large A} ou \yo{\Large A} \\
\yo{\Large aa} (hétérozygote) &  & \yo{\Large a} ou \yo{\Large a} \\
\hline
\end{tabular}
\end{center}
}

\only<2>{
Pour deux locus \yo{A} et \yo{B}
\begin{center}
\begin{tabular}{ccc}
\hline
\Large 2$n$ & \Large  $\rightarrow$ & \Large  $n$ \\
\hline
\yo{\Large [Aa --- Bb]} (hétérozygote) &  & \yo{\Large [A --- B]} ou \yo{\Large [A --- b]} ou \\
& & \yo{\Large [a --- B]} ou \yo{\Large [a --- b]} \\
& & \\
\yo{\Large [AA --- BB]} (homozygote) &  & \yo{\Large [A --- B]} ou \yo{\Large [B --- A]} \\
\hline
\end{tabular}
\end{center}
}


\end{frame}

\begin{frame}
\frametitle{Principes de biologie moléculaire}
\framesubtitle{La méiose : le brassage de l'information génétique}

\begin{columns}

\begin{column}{.4\textwidth}
\begin{itemize}
\item A cette étape il y a \yo{recombinaison} entre les informations génétiques qui viennent de chaque paire de chromosome par crossing-over.

\item Cela génère un brassage important et de nouvelles combinaisons d'allèles.

\item Plus le taux de recombinaison est faible, plus les gamètes sont proches des parents.

\end{itemize}


\end{column}

\begin{column}{.6\textwidth}
\begin{center}
\begin{overprint}
\onslide<1>\centering\includegraphics[width=\textwidth,page=1]{recombinaison}\\
\onslide<2>\centering\includegraphics[width=\textwidth,page=2]{recombinaison}\\
\onslide<3>\centering\includegraphics[width=\textwidth,page=3]{recombinaison}\\
\onslide<4>\centering\includegraphics[width=\textwidth,page=4]{recombinaison}\\
\onslide<5>\centering\includegraphics[width=\textwidth,page=5]{recombinaison}\\
\onslide<6>\centering\includegraphics[width=\textwidth,page=6]{recombinaison}\\
\onslide<7>\centering\includegraphics[width=\textwidth,page=7]{recombinaison}\\
\end{overprint}
\small La recombinaison lors de la méiose \\
\end{center}
\end{column}

\end{columns}

\end{frame}

\begin{frame}
\frametitle{Principes de biologie moléculaire}
\framesubtitle{La méiose : le brassage de l'information génétique}

\begin{overprint}
\onslide<1>\centering\includegraphics[width=.7\textwidth,page=8]{recombinaison}\\
\onslide<2>\centering\includegraphics[width=.7\textwidth,page=9]{recombinaison}\\
\onslide<3>\centering\includegraphics[width=.7\textwidth,page=10]{recombinaison}\\
\onslide<4>\centering\includegraphics[width=.7\textwidth,page=11]{recombinaison}\\
\onslide<5>\centering\includegraphics[width=.7\textwidth,page=12]{recombinaison}\\
\onslide<6-7>\centering\includegraphics[width=.7\textwidth,page=13]{recombinaison}\\
\end{overprint}

\onslide<7>
\begin{block}{}
\centering La méiose génère une infinité de gamète grâce aux crossing-over.
\end{block}

\end{frame}


\begin{frame}
\frametitle{Principes de biologie moléculaire}
\framesubtitle{La méiose : le brassage de l'information génétique}

\begin{center}
\includegraphics[width=.8\textwidth,page=14]{recombinaison}
\end{center}

\end{frame}

\begin{frame}
\frametitle{Principes de biologie moléculaire}
\framesubtitle{La méiose : le brassage de l'information génétique}

Il peut y avoir de un à deux crossing-over par chromosome.

Avec $K$ le nombre totaux de chromosomes, il y a 2\up{$K$} combinaisons de gamètes possibles chez une espèce diploide sans prendre en compte les crossing-over qui augmente la diversité des gamètes (=> valeur sous-estimée).

\vspace{.5cm}

\begin{tabular}{ c c p{.5\textwidth} }
\hline
Espèce & $K$ & Nombre de gamètes possibles (2\up{$K$}) \\
\hline
petit épeautre & 7 & $2\up{7} = 128$ \\
(\textit{Triticum monococcum}) & & \\
blé dur & 14 & $2\up{14} = 16~384$ \\
(\textit{Triticum durum}) & & \\
blé tendre & 21 & $2\up{21} = 2~097~152$ \\
(\textit{Triticum æstivum})  & & \\
\hline
\end{tabular}

\end{frame}




\begin{frame}
\frametitle{Principes de biologie moléculaire}
\framesubtitle{La fécondation chez les plantes}

La fécondation est la fusion de deux gamètes ($n$) pour donner une cellule oeuf (2$n$) qui donnera la graine.

\begin{block}{}
\centering \Large $n$ (pollen) + $n$ (ovaire) =  2$n$ (graine)
\end{block}

Nombre de possibilité lors de la fécondation $\sim$ (nombre de gamète)\up{2}, sans prendre en compte les crossing-over.
\begin{center}

\begin{tabular}{ c c }
\hline
Espèce & Nombre de possibilités \\
\hline
petit épeautre & $128^2 = 16~384$ \\
(\textit{Triticum monococcum}) & \\
blé dur & $16~384^2 = 268~435~456$ \\
(\textit{Triticum durum}) & \\
blé tendre & $2~097~152^2 = 4~398~047~000~000$ \\
(\textit{Triticum æstivum}) & \\
\hline
\end{tabular}

\end{center}


\end{frame}

\begin{frame}
\frametitle{Principes de biologie moléculaire}
\framesubtitle{La fécondation chez les plantes}

\begin{columns}

\begin{column}{.5\textwidth}
\begin{center}
\begin{tabular}{cc}
\includegraphics[width=.7\textwidth]{fecondation_vegetaux} &  \rotatebox{90}{
\tiny Nature et jardin, 13, avril 1996} \\
\small La fécondation chez les végétaux & \\

%\fcolorbox{mln-green}{white}{\href{https://www.youtube.com/watch?v=MJSgn-lONqM}{\small Lien vers une vidéo}} % Compliqué car double fécondation ...

\end{tabular}
\end{center}
\end{column}

\begin{column}{.5\textwidth}

\begin{itemize}
\item L'ovule transmet son demi noyau ($n$) + l'ADN chloroplastique + l'ADN mitochondrial
\item Le pollen transmet son demi noyau ($n$).
\end{itemize}

\end{column}

\end{columns}

\end{frame}


\begin{frame}
\frametitle{Principes de biologie moléculaire}
\framesubtitle{Ce qu'il faut retenir}

\begin{itemize}[<+->]
\item L'ADN est le support de l'information génétique, il se situe dans les chromosomes qui se situent dans le noyau des cellules.

\item Le nombre d'exemplaires de chromosomes dépend des espèces.

\item Les allèles d'un gène peuvent être dominants ou récessifs. Il existe des états intermédiaires. %(gradiant de dominance).

\item La méiose permet de passer du stade $2n$ à $n$ et de donner les gamètes.

\item Lors de la méiose, si l'individu est hétérozygote, la recombinaison (i.e. les crossing over) permet de brasser la diversité au sein d'un individu. Cela génère une infinité de gamète.

\item La fécondation permet de passer du stade $n$ + $n$ à $2n$ et ainsi de mettre en place de nouvelles combinaisons d'allèles. L'ovule apporte l'ADN chloroplastique et mitochondrial.

\end{itemize}


\end{frame}

\subsection{Techniques de biologie moléculaire}

\begin{frame}
\frametitle{Techniques de biologie moléculaire}
\framesubtitle{Les marqueurs moléculaires}

\begin{columns}
\begin{column}{.5\textwidth}
\yo{Marqueurs de l’ADN} : sites dans le génome qui présentent une variation de séquence d’un individu à l’autre.  \\

~\\\vfill

Par exemple: changement d’une base ou insertion ou délétion de quelques bases. \\

~\\\vfill

Il existe plusieurs types de marqueurs par exemple les SNP (Single Nucleotide Polymorphism)
\end{column}

\begin{column}{.5\textwidth}

\small

\begin{tabular}{ccccc}
\hline
Ind 1 & Ind 2 & Ind 3 & Ind 4 & Ind 5 \\
\hline
A & A & A & A & A \\
C & C & C & C & C \\
C & C & C & C & C \\
G & G & G & G & G \\ 
T &T & T & T & T \\
G & G & G & G & G \\
A & A & A & A & A \\
\cellcolor{mln-green} G & \cellcolor{mln-green} A & \cellcolor{mln-green} T & \cellcolor{mln-green} A & \cellcolor{mln-green} G \\ 
T &T & T & T & T \\
C & C & C & C & C \\
T &T & T & T & T \\
A & A & A & A & A \\
\hline
\end{tabular}
\centering\small Marqueurs SNP
\end{column}

\end{columns}

\end{frame}


\begin{frame}
\frametitle{Techniques de biologie moléculaire}
\framesubtitle{Les marqueurs moléculaires}

Trois familles de marqueurs:

\begin{itemize}[<+->]
\item \yo{Marqueurs \guill{neutres}}: localisés dans  une zone non codante (qui ne sera pas  transcrite et traduite en protéine).  Cette zone peut cependant influencer la  régulation de l’expression d’autres gènes.
Des marqueurs neutres répartis sur l’ensemble des chromosomes renseignent sur la diversité et le comportement global du génome et donc sur l’histoire des populations étudiées. 

\item \yo{Marqueurs localisés dans des gènes}: associés la variation de caractères.
Par exemple la précocité de floraison.

\item (\yo{Marqueurs épigénétiques}: informations non transmises par l'ADN mais par des méthylations sur l'ADN. Celles-ci sont issues d'une réponse à un environnement donné.)

\end{itemize} 


\end{frame}


\begin{frame}
\frametitle{Techniques de biologie moléculaire}
\framesubtitle{Quelques analyses avec des marqueurs moléculaires}


Il est possible d'utiliser les marqueurs pour:

\begin{itemize}
\item comprendre les déterminismes génétiques
\item étudier la diversité et l'évolution des populations
\end{itemize}

\end{frame}


\begin{frame}
\frametitle{Techniques de biologie moléculaire}
\framesubtitle{Comprendre les déterminismes génétiques}

\yo{Objectif}: associer un marqueur à un caractère phénotypique. 
On parle de génétique d'associations, de détéction de QTL (Quantitative Trait Loci). 

Pour cela, il faut:
\begin{itemize}
\item des données phénotypiques (\guill{phénomique})
\item des données moléculaires (\guill{génomique}: \guill{puces} de milliers de marqueurs SNPs (420 000 chez le blé tendre))
\end{itemize}

\vspace{.5cm}

\begin{overprint}
\onslide<2>
\begin{center}
\begin{tabular}{ccccccc}
\hline
Ind & pmg & m1 & m2 & m3 & m4 & m5\\
\hline
Ind 1 & 55 & \cellcolor{mln-green} A & C & \cellcolor{mln-green} G & G & T \\
Ind 2 & 50 & T & C & C & G & A \\
Ind 3 & 45 & C & G & A & A & T \\
Ind 4 & 40 & T & T & A & A & C \\
Ind 5 & 35 & G & A & C & G & C \\
\hline
\end{tabular}
\end{center}

\onslide<3>
Ces associations statistiques sont dépendantes de la structure des populations, du nombre d'individus, du nombre de marqueurs ...
\end{overprint}



\end{frame}


\begin{frame}
\frametitle{Techniques de biologie moléculaire}
\framesubtitle{Etudier la diversité et l'évolution de populations}

\yo{Objectif}: Mieux comprendre:

\begin{itemize}
\item quelle est la structure de la diversité génétique au sein des variétés locales ou paysannes,
\item comment la diversité est maintenue au cours des processus de sélection,
\item comment la diversité est mobilisée pour permettre aux populations de répondre à la sélection.
\end{itemize}

Pour cela, il faut
\begin{itemize}
\item des données phénotypiques
\item des données moléculaires: 90 marqueurs SNP suffisent: par exemple 45 neutres et 45 dans des gènes de précocité de floraison.
\end{itemize}



\end{frame}



