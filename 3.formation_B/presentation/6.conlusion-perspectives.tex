\section{Conclusion et perspectives}
\begin{frame}\tableofcontents[currentsection,currentsubsection,subsectionstyle=show/show/hide]\end{frame}


\begin{frame}
\frametitle{Conclusion et perspectives}

\begin{block}{}
\begin{center}
\Huge
Valoriser cette diversité nouvellement brassée pour sélectionner ... c'est à dire choisir
\end{center}
\end{block}

\end{frame}


\begin{frame}
\frametitle{Conclusion et perspectives}

\begin{center}
\includegraphics[width=.8\textwidth, page=1]{methodo-globale}
\end{center}

\end{frame}



\begin{frame}
\frametitle{Conclusion et perspectives}


\begin{itemize}
\item Gestion de la biodiversité cultivée \exsitu~et \insitu.
\begin{itemize}
\item Cultiver, produire et sélectionner des semences paysannes \yo{dans sa ferme}, et \yo{en réseau}, contribue au maintien et au renouvellement de la biodiversité cultivée.
\item Complémentarité entre gestion \exsitu~et \insitu.
\end{itemize}
\end{itemize}

\begin{columns}

\begin{column}{.45\textwidth}
\begin{itemize}
\item Mobilisation de la biodiversité cultivée
\begin{itemize}
\item Dans les \CRBs
\item Dans les \MSPs
\end{itemize}

\item Brassage de la biodiversité cultivée
\end{itemize}

\end{column}

\begin{column}{.55\textwidth}
\begin{center}
\includegraphics[width=\textwidth]{gradiant_brassage}
\end{center}
\end{column}

\end{columns}



\end{frame}



\begin{frame}
\frametitle{Conclusion et perspectives}

Ce qui est issu de ce brassage est \yo{\Huge nouveau} et \yo{\Huge unique}.

\begin{block}{}
\Large Le brassage de la biodiversité cultivée par les paysans, en collaboration avec les animateurs et les chercheurs est une étape vers la sélection de nouvelles semences paysannes adaptées à la diversité des pratiques agroécologique.
\end{block}

\begin{block}{}
\Large Ces nouvelles populations paysannes en développement sont soumises à des règles d'usage propre à chaque collectif de travail.
\end{block}

\end{frame}
