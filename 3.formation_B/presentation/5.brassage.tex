\section{Brassage de la biodiversité cultivée}
\begin{frame}\tableofcontents[currentsection,currentsubsection,subsectionstyle=show/show/hide]\end{frame}

\begin{frame}
\frametitle{Brassage de la biodiversité cultivée}

\yo{Brasser} la diversité permet d'\yo{augmenter} la diversité et ainsi d'avoir plus de possibilité pour sélectionner.

\begin{columns}

\begin{column}{.4\textwidth}

Plusieurs stratégies sont possibles:

\begin{itemize}
\item lignées pures
\item mélange de lignées pures
\item populations
\item mélange de populations
\item croisements  bi-parentaux 
\item croisement multi-parentaux (CCP)
\item population mâles-stériles
\end{itemize}

\end{column}

\begin{column}{.6\textwidth}
\begin{center}
\includegraphics[width=\textwidth]{gradiant_brassage} \\
\end{center}
\end{column}

\end{columns}

\end{frame}


\begin{frame}
\frametitle{Brassage de la biodiversité cultivée}
\framesubtitle{Un levier pour favoriser la sélection}

\begin{itemize}
\item \yo{Sélection naturelle}: les individus les plus adaptés aux pratiques, au sol, à l'environnement, se reproduisent plus : évolution de la fréquence des individus au sein de la population

\item \yo{Sélection intra}: Sélection à l'intérieur d'un groupe de plantes

\item \yo{Sélection inter}: Sélection entre groupe de plantes

\end{itemize}

\begin{center}
\begin{tabular}{cccc}
\hline
Type de diversité | sélection & naturelle & intra & inter \\
\hline
Lignées pures & $--$ & $--$ & $+++$ \\
\hline
Mélanges de lignées pures & $+$ & $+$ & $+++$ \\
\hline
Populations & $++$ & $++$ & $+++$ \\
\hline
Mélanges de populations & $+++$ & $+++$ & $++$ \\
\hline
Croisements bi-parentaux & $+++$ & $+++$ & $++$ \\
\hline
Croisements multi-parentaux & $++++$ & $++++$ & $++$ \\
\hline
Populations mâles-stériles & $+++++$ & $+++++$ & $++$ \\
\hline
\end{tabular}
\end{center}


\end{frame}


\begin{frame}
\frametitle{Brassage de la biodiversité cultivée}
\framesubtitle{Le brassage est contraint par l'espèce}

\begin{center}
\includegraphics[width=.8\textwidth,page=6]{genealogie_ble}
\end{center}

% On peut tenter interspécifique!

\end{frame}

\subsection{Lignées pures}

\begin{frame}
\frametitle{Brassage de la biodiversité cultivée}
\framesubtitle{Les lignées pures}


\begin{columns}
\begin{column}{.5\textwidth}

\begin{itemize}
\item Au sein d'une lignée pure, toutes les plantes sont identiques.
\item Si elles se croisent entre elles, les plantes issues du croisement sont identiques aux plantes parents.
\end{itemize}

\begin{center}
\begin{tabular}{ccc}
\hline
\multicolumn{3}{c}{sélection} \\
naturelle & intra & inter \\
\hline
$--$ & $--$ & $+++$ \\
\hline
\end{tabular}
\end{center}

\end{column}

\begin{column}{.5\textwidth}
\includegraphics[width=\textwidth,page=1]{brassage_biodiv}
\end{column}
\end{columns}


\end{frame}


\subsection{Mélanges de lignées pures}

\begin{frame}
\frametitle{Brassage de la biodiversité cultivée}
\framesubtitle{Les mélanges de lignées pures}

\begin{itemize}
\item Au sein d'un mélange de lignées pures, il y a différentes plantes.
\item Les plantes différents peuvent se croiser et donner de nouvelles plantes (i.e. de nouvelles combinaisons d'allèles).
\end{itemize}


\begin{columns}

\begin{column}{.5\textwidth}

\begin{center}
\begin{tabular}{ccc}
\hline
\multicolumn{3}{c}{sélection} \\
naturelle & intra & inter \\
\hline
$+$ & $+$ & $+++$ \\
\hline
\end{tabular}
\end{center}

\end{column}

\begin{column}{.5\textwidth}
\includegraphics[width=\textwidth,page=2]{brassage_biodiv}
\end{column}
\end{columns}

% cf fonction mixture.R

\end{frame}





\begin{frame}
\frametitle{Brassage de la biodiversité cultivée}
\framesubtitle{Les mélanges de lignées pures}

\begin{columns}
\begin{column}{.3\textwidth}

\begin{itemize}
\item Le taux de croisement dépend des espèces.
\end{itemize}

\begin{center}
\begin{tabular}{ p{.5\textwidth} p{.5\textwidth} }
\hline
Espèces & Taux \\
\hline
Blé tendre & 5\% \\
\hline
\end{tabular}	
\end{center}

\end{column}

\begin{column}{.7\textwidth}
\includegraphics[width=\textwidth,page=3]{brassage_biodiv}
\end{column}
\end{columns}

\end{frame}




\subsection{Populations}

\begin{frame}
\frametitle{Brassage de la biodiversité cultivée}
\framesubtitle{Les populations}

\begin{columns}
\begin{column}{.5\textwidth}

\begin{itemize}
\item Au sein d'une populations, il y a de nombreuses plantes différentes.
\item Les plantes différentes peuvent se croiser et donner de nouvelles plantes (i.e. de nouvelles combinaisons d'allèles).
Le taux de croisement dépend des espèces.
\end{itemize}


\begin{center}
\begin{tabular}{ccc}
\hline
\multicolumn{3}{c}{sélection} \\
naturelle & intra & inter \\
\hline
$++$ & $++$ & $+++$ \\
\hline
\end{tabular}
\end{center}

\end{column}

\begin{column}{.5\textwidth}
\includegraphics[width=\textwidth,page=4]{brassage_biodiv}
\end{column}
\end{columns}

\end{frame}


\subsection{Mélanges de populations}

\begin{frame}
\frametitle{Brassage de la biodiversité cultivée}
\framesubtitle{Les mélanges de populations}


\begin{columns}
\begin{column}{.5\textwidth}

\begin{itemize}
\item Au sein d'un mélange de populations, il y a de nombreuses plantes différentes.
\item Les plantes différentes peuvent se croiser et donner de nouvelles plantes (i.e. de nouvelles combinaisons d'allèles).
Le taux de croisement dépend des espèces.
\end{itemize}

\begin{center}
\begin{tabular}{ccc}
\hline
\multicolumn{3}{c}{sélection} \\
naturelle & intra & inter \\
\hline
$+++$ & $+++$ & $++$ \\
\hline
\end{tabular}
\end{center}

\end{column}

\begin{column}{.5\textwidth}
\includegraphics[width=\textwidth,page=5]{brassage_biodiv}
\end{column}
\end{columns}

\end{frame}


\begin{frame}
\frametitle{Brassage de la biodiversité cultivée}
\framesubtitle{Les croisements}

Il existe des méthodes interventionnistes pour brasser la diversité comme:

\begin{itemize}
\item les croisements bi-parentaux
\item les croisements multi-parentaux
\item les populations mâles-stériles
\end{itemize}

Ces méthodes permettent 

\begin{itemize}
\item de controler et de diriger les croisements
\item de gagner du temps
\end{itemize}

\end{frame}



\subsection{Croisements bi-parentaux}

\begin{frame}
\frametitle{Brassage de la biodiversité cultivée}
\framesubtitle{Croisements bi-parentaux: loi de Mendel}

\begin{columns}
\begin{column}{.5\textwidth}
Mendel (1822 - 1884) est un moine tchèque qui a travaillé sur les petits poids et a découvert les lois de l'hérédité.
Ces lois permettent de prédire la ségrégation des caractères sachant deux parents.
C'est à dire la proportion de chaque parent que l'on retrouve dans les générations suivantes.
\end{column}

\begin{column}{.5\textwidth}

\begin{center}
\begin{tabular}{cc}
\includegraphics[width=.6\textwidth]{Gregor_Mendel_oval} &  \rotatebox{90}{\tiny CC BY 4.0. Itlis, Hugo} \\
Gregor Mendel & \\
\end{tabular}
\end{center}


\end{column}

\end{columns}

\end{frame}

\begin{frame}
\frametitle{Brassage de la biodiversité cultivée}
\framesubtitle{Croisements bi-parentaux: 1 locus, 2 allèles (loi de Mendel)}


\begin{center}

Deux parents diploides \fcolorbox{black}{mln-green}{AA} et \fcolorbox{black}{mln-green}{aa} que l'on croise vont donner:

\begin{tabular}{|c|c|}
\hline
\textbf{F1} & A  \\
\hline
a & Aa (ou aA)\\
\hline
\end{tabular}

\vspace{.5cm}

100\% de Aa (ou aA) qui s'autoféconde : \\

\begin{tabular}{|c|c|c|}
\hline
\textbf{F2}& A & a \\
\hline
A & \cellcolor{mln-green} AA & Aa \\
\hline
a & aA & \cellcolor{mln-green} aa\\
\hline
\end{tabular}
\end{center}

\begin{columns}

\begin{column}{.3\textwidth}
\begin{center}
25\% de AA qui s'autoféconde : \\
\begin{tabular}{|c|c|}
\hline
\textbf{F3}& A  \\
\hline
A & \cellcolor{mln-green} AA \\
\hline
\end{tabular}
\end{center}
\end{column}

\begin{column}{.3\textwidth}
\begin{center}
50\% de  Aa (ou aA) qui s'autoféconde : \\
\begin{tabular}{|c|c|c|}
\hline
\textbf{F3}& A & a \\
\hline
A & \cellcolor{mln-green} AA & Aa \\
\hline
a & aA & \cellcolor{mln-green} aa\\
\hline
\end{tabular}
\end{center}
\end{column}

\begin{column}{.3\textwidth}
\begin{center}
25\% de aa qui s'autoféconde : \\
\begin{tabular}{|c|c|}
\hline
\textbf{F3}& a  \\
\hline
a & \cellcolor{mln-green} aa \\
\hline
\end{tabular}
\end{center}
\end{column}

\end{columns}

\end{frame}


\begin{frame}
\frametitle{Brassage de la biodiversité cultivée}
\framesubtitle{Croisements bi-parentaux: 1 locus, 2 allèles (loi de Mendel)}

\begin{center}
\includegraphics[width=.8\textwidth]{segregation_f2}
\end{center}

\end{frame}


\begin{frame}
\frametitle{Brassage de la biodiversité cultivée}
\framesubtitle{Croisements bi-parentaux: 2 locus, 4 allèles (loi de Mendel)}

\begin{center}

Deux parents diploides \fcolorbox{black}{mln-green}{AA BB} et \fcolorbox{black}{mln-green}{aa bb} va donner:

\begin{tabular}{|c|c|}
\hline
\textbf{F1} & AA BB \\
\hline
aa bb & Aa Bb \\
\hline
\end{tabular}

\vspace{.5cm}

100\% de Aa Bb qui s'autoféconde : \\

\begin{tabular}{|c|c|c|c|c|}
\hline
\textbf{F2}& AB & Ab & aB & ab \\
\hline
AB & \cellcolor{mln-green} AA BB & AA bb & Aa BB & Aa Bb \\
\hline
Ab & AA bB & AA bb & Aa bB & Aa bb \\
\hline
aB & aA BB & aA Bb & aa BB & aa Bb \\
\hline
ab & aA bB & aA bb & aa bB & \cellcolor{mln-green} aa bb \\
\hline
\end{tabular}

\vspace{.5cm}

\textbf{F3} ... \\

\end{center}

\end{frame}


\begin{frame}
\frametitle{Brassage de la biodiversité cultivée}
\framesubtitle{Croisements bi-parentaux: 3 locus, 6 allèles (loi de Mendel)}

\begin{center}

Deux parents diploides \fcolorbox{black}{mln-green}{AA BB CC} et \fcolorbox{black}{mln-green}{aa bb cc} va donner:

\begin{tabular}{|c|c|}
\hline
\textbf{F1} & AA BB CC \\
\hline
aa bb cc & Aa Bb Cc \\
\hline
\end{tabular}

\vspace{.5cm}

100\% de Aa Bb Cc qui s'autoféconde : \\

\tiny
\begin{tabular}{|
c|
p{.083\textwidth}|
p{.083\textwidth}|
p{.081\textwidth}|
p{.082\textwidth}|
p{.081\textwidth}|
p{.083\textwidth}|
p{.081\textwidth}|
p{.082\textwidth}|}
\hline
\textbf{\normalsize F2}& ABC & AbC & ABc & Abc & aBC & abC & aBc & abc \\
\hline
ABC & \cellcolor{mln-green} AA BB CC & AA Bb CC & AA BB Cc & AA Bb Cc & Aa BB CC & Aa Bb CC & Aa BB Cc & Aa Bb Cc \\
AbC & AA bB CC & AA bb CC & AA bB Cc & AA bb Cc & Aa bB CC & Aa bb CC & Aa bB Cc & Aa bb Cc \\
ABc & AA BB cC & AA Bb cC & AA BB cc & AA Bb cc & Aa BB cC & Aa Bb cC & Aa BB cc & Aa Bb cc \\
Abc & AA bB cC & AA bb cC & AA bB cc & AA bb cc & Aa bB cC & Aa bb cC & Aa bB cc & Aa bb cc \\
aBC & aA BB CC & aA Bb CC & aA BB Cc & aA Bb Cc & aa BB CC & aa Bb CC & aa BB Cc & aa Bb Cc \\
abC & aA bB CC & aA bb CC & aA bB Cc & aA bb Cc & aa bB CC & aa bb CC & aa bB Cc & aa bb Cc \\
aBc & aA BB cC & aA Bb cC & aA BB cc & aA Bb cc & aa BB cC & aa Bb cC & aa BB cc & aa Bb cc \\
abc & aA bB cC & aA bb cC & aA bB cc & aA bb cc & aa bB cC & aa bb cC & aa bB cc & \cellcolor{mln-green} aa bb cc \\
\hline
\end{tabular}

\vspace{.5cm}

\normalsize\textbf{F3} ...


\end{center}

\end{frame}

\begin{frame}
\frametitle{Brasser la diversité}
\framesubtitle{Croisements bi-parentaux : loi de Mendel}

\begin{columns}

\begin{column}{.5\textwidth}

\includegraphics[width=.9\textwidth, page=1]{croisement+generations}

\end{column}

\begin{column}{.5\textwidth}
Chez le blé autogamme\\

~\\

\small

\begin{tabular}{cl}
\hline
Génération & taux d'hétérozygote \\
\hline
1 & 100\% \\
2 & 50\% \\
3 & 25\% \\
4 & 12,5\% \\
5 & 6,25\% \\
6 & 3,125\% \\
7 & 1,5625\% \\
$\vdots$ & $\vdots$ \\
\hline
\end{tabular}

\normalsize

~\\

Plus le taux d'hétérozygotie est faible, plus les plantes parents et les plantes enfants sont semblables.

\end{column}

\end{columns}


\end{frame}


\begin{frame}
\frametitle{Brassage de la biodiversité cultivée}
\framesubtitle{Croisements bi-parentaux: loi de Mendel}


Chez une espèce diploide, en faisant l'hypothèse que toutes les combinaisons d'allèles sont possibles (=> valeur sur-estimée car certains locus sont liés).

\begin{center}
\begin{tabular}{cc}
\hline
Nombre de locus & Nombre de combinaison en F2 \\
\hline
1 & 4 \\
2 & 16 \\
3 & 64 \\
$x$ & $4^x$ \\
10 & 1048576 \\
100 & 1.606938e+60 \\
\hline
\end{tabular}
\end{center}

\begin{block}{}
\centering\Large
Le nombre de combinaisons est infini !!!
\end{block}

%Courbe avec CCP.R (cas particulier de deux parents)

%Dans une pop : plus que deux allèles car pas meme parent tout le temps ...
\end{frame}



\subsection{Croisements multi-parentaux}
\begin{frame}
\frametitle{Brassage de la biodiversité cultivée}
\framesubtitle{Croisements multi-parentaux (CCP)}

Les croisements bi-parentaux (CCP = Composite Cross Population) sont le fait de croiser plusieurs parents deux à deux et de mélanger les graines issues de ces croisements.
Ce travail est très lourd est peut être réalisé par des équipes de recherche avec qui les collectifs paysans collaborent.

\scriptsize
\begin{center}
\begin{tabular}{c|ccccccc}
\hline
Parent & 1 & 2 & 3 & 4 & $\ldots$ & $n-1$ & $n$ \\
\hline
1 &  & C-$[1]$-$[2]$ & C-$[1]$-$[3]$ & C-$[1]$-$[4]$ & $\ldots$ & C-$[1]$-$[n-1]$ & C-$[1]$-$[n]$\\
2 &  &  & C-$[2]$-$[3]$ & C-$[2]$-$[4]$ & $\ldots$ & C-$[2]$-$[n-1]$ & C-$[2]$-$[n]$\\
3 &  &  &  & C-$[3]$-$[4]$ & $\ldots$ & C-$[3]$-$[n-1]$ & C-$[3]$-$[n]$\\
4 &  &  &  & & $\ldots$ & C-$[4]$-$[n-1]$ & C-$[4]$-$[n]$\\
$\vdots$ & & & & & $\vdots$ & $\vdots$ & $\vdots$ \\
Parent $n-1$ &  &  &  &  &  &  & C-$[n-1]$-$[n]$\\
\hline
\end{tabular}
\end{center}


\normalsize

Toutes les combinaisons sont envisageables: hybride deux voies, trois voies, etc

\end{frame}


\subsection{Populations mâles-stériles}

\begin{frame}
\frametitle{Brassage de la biodiversité cultivée}
\framesubtitle{Populations mâles-stériles}

C'est une population hautement recombinante (dite MAGIC !) qui a de nombreuses combinaisons d'allèles grâce au fort taux d'allogamie.

60 parents ont été croisés deux à deux au départ puis croisé avec Probus, qui détient un gène de stérilité mâle.

\vspace{.5cm}

La moitié des plantes dans la descendance est mâle-stérile => croisement obligatoire par une plante voisine mâle-fertile.

\begin{itemize}
\item si on ne resème que les plantes mâle-stérile, alors la population est brassée en continue
\item si on ne sélectionne pas, la stérilité mâle va disparaitre au bout de quelques générations
\end{itemize}

Cette population est compliquée à gérer mais peut être un moyen de rebrasser un ensemble de variétés locales. % sélection ensuite indispensable !

\end{frame}


\begin{frame}
\frametitle{Brassage de la biodiversité cultivée}
\framesubtitle{Populations mâles-stériles}
\begin{center}
\begin{tabular}{cccc}
\includegraphics[width=.33\textwidth]{epifert-19Mai} & \rotatebox{90}{\tiny Goldringer, Isabelle} & 
\includegraphics[width=.33\textwidth]{epiMS-19Mai} & \rotatebox{90}{\tiny Goldringer, Isabelle} & 
\\
\small Epi fertile & & Epi stérile & \\
\end{tabular}
\end{center}
\end{frame}


\begin{frame}
\frametitle{Brassage de la biodiversité cultivée}
\framesubtitle{Populations mâles-stériles}
\begin{center}
\begin{tabular}{cccc}
\includegraphics[width=.4\textwidth]{msms5} & \rotatebox{90}{\tiny Enjalbert, Jérôme} & 
\includegraphics[width=.4\textwidth]{msms4} & \rotatebox{90}{\tiny Enjalbert, Jérôme} & 
\\
\end{tabular}
\end{center}
\end{frame}


\begin{frame}
\frametitle{Brassage de la biodiversité cultivée}
\framesubtitle{Populations mâles-stériles}

\begin{center}
\begin{tabular}{cc}
\includegraphics[width=.85\textwidth]{msms2} &  \rotatebox{90}{\tiny Lecarpentier, Christophe} \\
\end{tabular}
\end{center}

\end{frame}




\subsection{Mise en place d'une stratégie}

\begin{frame}
\frametitle{Brassage de la biodiversité cultivée}
\framesubtitle{Mise en place d'une stratégie: les questions à se poser. Ces étapes font parties de la co-construction du projet.}


\begin{itemize}
\item Est ce que je souhaite contrôler le croisement?

\item Quels moyens? croisement bi-parentaux / CCP

\begin{itemize}
\item Faire les croisements est long
\item Plus les populations issues du brassage sont diversifiées, plus elles doivent être cultivées sur de grandes surface.
10 000 plantes permettent de ne pas perdre trop de diversité.
Avec 300 grains au m2 $\rightarrow$ environ 35 m2.
\end{itemize}

\item Croisement en station de recherche ou à la ferme?

\item A quelle échelle? Ferme? Département? Région? 

\item Quelle fréquence pour rebrasser? Tous les 5 ans, 10 ans, 25 ans?

\end{itemize}

\end{frame}


\begin{frame}
\frametitle{Brassage de la biodiversité cultivée}
\framesubtitle{Mise en place d'une stratégie: les questions à se poser. Ces étapes font parties de la co-construction du projet.}


\begin{itemize}

\item Quels parents? Quels critères je souhaite associer?

\begin{itemize}
\item Parents historiques de la région, du pays ou exotique?
\item Combien de génération avant de croiser? (Cf adaptation au terroir)
\end{itemize}

Evaluation du programme de \SP~blé:

\begin{itemize}
\item Population avec une variétés moderne dans les parents: plus de potentiel pour le rendement
\item Populations issues de croisements variétés anciennes $\times$ variétés de pays: très diversifiées
\end{itemize}

\end{itemize}



\end{frame}



\begin{frame}
\frametitle{Brassage de la biodiversité cultivée}
\framesubtitle{Mise en place d'une stratégie}

\begin{block}{}
\centering\Large
De la diversité ... pour sélectionner !!!
\end{block}

\begin{center}
\begin{tabular}{cccc}
\hline
Type de diversité | sélection & naturelle & intra & inter \\
\hline
Lignées pures & $--$ & $--$ & $+++$ \\
\hline
Mélanges de lignées pures & $+$ & $+$ & $+++$ \\
\hline
Populations & $++$ & $++$ & $+++$ \\
\hline
Mélanges de populations & $+++$ & $+++$ & $++$ \\
\hline
Croisements bi-parentaux & $+++$ & $+++$ & $++$ \\
\hline
Croisements multi-parentaux & $++++$ & $++++$ & $++$ \\
\hline
Populations mâles-stériles & $+++++$ & $+++++$ & $++$ \\
\hline
\end{tabular}
\end{center}


\end{frame}




\subsection{Exemple d'un projet sur le blé tendre}

\begin{frame}
\frametitle{Brassage de la biodiversité cultivée}
\framesubtitle{Dans notre projet sur le blé tendre}

\begin{columns}

	\begin{column}{.6\textwidth}
	\begin{itemize}
	\item<1-> 2005-2006 : 90 croisements à la ferme par JFB et l'équipe de recherche. JFB a choisi les parents. Culture sur 5m\up{2}.
	\item<2-> 2008 : sélection dans les F2. F3 distribuées à des fermes du RSP.
	\item<3-> 2008 => aujourd'hui : populations cultivées chaque année dans une ou plusieurs fermes.
Nouvelles populations développées par les paysans à partir de croisements : croisement fait à la station de recherche ou à la ferme, choix des parents par les paysans \dots
	\end{itemize}
	\end{column}

	\begin{column}{.4\textwidth}
	\begin{center}
	\begin{overprint}
    \onslide<1> \includegraphics[width=\textwidth,page=1]{croisement}
    \onslide<2> \includegraphics[width=\textwidth,page=1]{carte_diffusion_2008.png}
	\onslide<3> \includegraphics[width=\textwidth]{carte_2011-2012.png}
	\onslide<4> \includegraphics[width=\textwidth,page=2]{OUT_graph_4-c-2-evolution-etape-ppb__dynamique-relation-reseau_without_MLN}
	\end{overprint}
	\end{center}

	\end{column}
\end{columns}

\end{frame}