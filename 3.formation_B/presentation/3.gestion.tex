\section{Gestion de la biodiversité cultivée}
\begin{frame}\tableofcontents[currentsection,currentsubsection,subsectionstyle=show/show/hide]\end{frame}


\begin{frame}
\includegraphics[width=\textwidth, page=1]{methodo-globale}
\end{frame}

\begin{frame}

Pour développer de nouvelles variétés il faut de la 

\begin{tabular} {p{.3\textwidth} p{.3\textwidth} p{.3\textwidth}} 
\includegraphics[width=0.3\textwidth]{VP.png} & \includegraphics[width=0.3\textwidth]{P1080569} & \includegraphics[width=0.3\textwidth]{P1090306} \\
\end{tabular}

\begin{block}{}
\Huge\centering DIVERSITE !!!
\end{block}

\begin{tabular} {p{.3\textwidth} p{.3\textwidth} p{.3\textwidth}} 
\includegraphics[width=0.3\textwidth]{P1090190} & \includegraphics[width=0.3\textwidth]{P1090191} & \includegraphics[width=0.3\textwidth]{IMG_0350} \\
\end{tabular}

Plus il y a de diversité, plus les chances de trouver ce qui convient augmentent ... et il convient de bien la gérer!

\end{frame}


\subsection{Deux méthodes de gestion de la diversité}

\begin{frame}
\frametitle{Deux méthodes de gestion de la diversité}

\begin{columns}

\begin{column}{.4\textwidth}
Dans les banques de graines. C'est la gestion \exsitu. 
Gestion \yo{statique} par des centres de ressources génétiques.
Conservation d'accessions à l'identique.
\begin{center}
\includegraphics[width=.4\textwidth]{gnis_exsitu} \\
\tiny crédit : GNIS \\
\end{center}
\end{column}

\begin{column}{.6\textwidth}
\begin{center}
\includegraphics[width=.6\textwidth]{VP}
\end{center}
Dans les Maisons des Semences Paysannes et dans les champs (\textit{on-farm}) et dans les jardins. C'est une gestion \yo{dynamique} \insitu.
Conservation de la diversité génétique et des processus évolutifs qui en sont à l'origine.
\end{column}

\end{columns}

\end{frame}


\subsection{Gestion ex-situ}

\begin{frame}
\frametitle{La gestion \exsitu}

La gestion \exsitu~se fait:

\begin{itemize}
\item Dans les \CRBs
\item Dans les \MSPs~en alternance avec la gestion \insitu
\end{itemize}

% Pratique CRB: cf thèse mathieu?

\end{frame}



\subsection{Gestion in-situ}

\begin{frame}
\frametitle{La gestion \insitu}


La diversité génétique évolue grâce aux forces évolutives qui en sont à l'origine:

\begin{itemize}
\item la sélection naturelle
\item la dérive génétique
\item la migration
\item la mutation
\end{itemize}

\end{frame}


\begin{frame}
\frametitle{La gestion \insitu}
\framesubtitle{Forces évolutives: la sélection naturelle}

\begin{columns}

\begin{column}{.5\textwidth}
\centering
\includegraphics[width=.8\textwidth, page=1]{Darwin}
\end{column}


\begin{column}{.5\textwidth}

Charles Darwin (1809 - 1882)

\begin{itemize}
\item \textbf{Reproduction} 
\item \textbf{Variation} dans la descendance
\item \textbf{Tri} des individus les plus adaptés qui se reproduisent plus : évolution de la fréquence des individus au sein de la population
\end{itemize}

\begin{block}{}
\centering
Evolution des populations
\end{block}

\end{column}

\end{columns}

\end{frame}


\begin{frame}
\frametitle{La gestion \insitu}
\framesubtitle{Forces évolutives: la sélection naturelle}

\begin{columns}

\begin{column}{.4\textwidth}
Placer des populations hétérogènes génétiquement dans des environnements contrastés 

$\Rightarrow$ permet à la diversité génétique soumise aux pressions évolutives d'évoluer en s’adaptant aux conditions du milieu
\end{column}

\begin{column}{.6\textwidth}
\begin{overprint}
\onslide<1>\centering\includegraphics[width=\textwidth,page=1]{evolution_pop}
\onslide<2>\centering\includegraphics[width=\textwidth,page=2]{evolution_pop}
\onslide<3>\centering\includegraphics[width=\textwidth,page=3]{evolution_pop}
\onslide<4>\centering\includegraphics[width=\textwidth,page=4]{evolution_pop}
\onslide<5>\centering\includegraphics[width=\textwidth,page=5]{evolution_pop}
\onslide<6>\centering\includegraphics[width=\textwidth,page=6]{evolution_pop}
\onslide<7>\centering\includegraphics[width=\textwidth,page=7]{evolution_pop}
\onslide<8>\centering\includegraphics[width=\textwidth,page=8]{evolution_pop}
\onslide<9>\centering\includegraphics[width=\textwidth,page=9]{evolution_pop}
\onslide<10>\centering\includegraphics[width=\textwidth,page=10]{evolution_pop}
\end{overprint}
\end{column}

\end{columns}


\end{frame}


\begin{frame}
\frametitle{La gestion \insitu}
\framesubtitle{Forces évolutives: la dérive génétique}

Echantillonage au hasard des individus

\begin{overprint}
\onslide<1>\centering\includegraphics[page=23,width=.7\textwidth]{evolution_pop}
\onslide<2>\centering\includegraphics[page=24,width=.7\textwidth]{evolution_pop}
\onslide<3>\centering\includegraphics[page=25,width=.7\textwidth]{evolution_pop}
\onslide<4>\centering\includegraphics[page=26,width=.7\textwidth]{evolution_pop}
\onslide<5>\centering\includegraphics[page=27,width=.7\textwidth]{evolution_pop}
\end{overprint}

\end{frame}


\begin{frame}
\frametitle{La gestion \insitu}
\framesubtitle{Forces évolutives: la migration}

Diversifier une population qui a tendance à perdre en diversité par mélange ou pollinisation croisée.

\begin{overprint}
\onslide<1>\centering\includegraphics[page=28,width=.7\textwidth]{evolution_pop}
\onslide<2>\centering\includegraphics[page=29,width=.7\textwidth]{evolution_pop}
\onslide<3>\centering\includegraphics[page=30,width=.7\textwidth]{evolution_pop}
\end{overprint}

\end{frame}


\begin{frame}
\frametitle{La gestion \insitu}
\framesubtitle{Forces évolutives: la mutation}

Apparition au hasard de nouveaux allèles (i.e. nouvelles versions de gènes)

\begin{overprint}
\onslide<1>\centering\includegraphics[page=31,width=.7\textwidth]{evolution_pop}
\onslide<2>\centering\includegraphics[page=32,width=.7\textwidth]{evolution_pop}
\end{overprint}

\end{frame}


\begin{frame}
\frametitle{La gestion \insitu}
\framesubtitle{L'hérédité des caractères acquis}

\begin{columns}

\begin{column}{.5\textwidth}
\centering
\includegraphics[width=.8\textwidth, page=1]{Lamark}
\end{column}


\begin{column}{.5\textwidth}

Jean-Baptiste Lamarck (1744 - 1829)

\begin{itemize}
\item Les individus transmettent à leur descendance ce qu'ils ont acqui dans leur environnement
\item L'information ne passe pas par l'ADN: caractère épi-génétique
\end{itemize}

\begin{block}{}
\centering
Evolution des individus
\end{block}

\end{column}

\end{columns}

\begin{overprint}

\onslide<2>
\begin{block}{}
\centering
\Large
Il y a un équilibre entre forces évolutives et transmission des caratères acquis
\end{block}

\end{overprint}

%http://fr.wikipedia.org/wiki/Transmission_des_caract%C3%A8res_acquis

%http://onlinelibrary.wiley.com/doi/10.15252/embj.201488883/abstract

\end{frame}


\begin{frame}
\frametitle{La gestion \insitu}

La gestion \insitu~peut se faire de différentes manières:

\begin{itemize}
\item dans les réseaux de station de recherche,
\item dans les réseaux de \MSPs~qui cultivent et produisent,
\item dans les réseaux de \MSPs~qui sélectionnent
\end{itemize}


\end{frame}

%
%\begin{frame}
%\frametitle{La gestion \insitu}
%\framesubtitle{Dans les réseaux de station de recherche}
%
%cf résultats équipe isa, cf papier jérome 2011
%
%\end{frame}
%


\begin{frame}
\frametitle{La gestion \insitu}
\framesubtitle{Dans les réseaux de \MSPs~qui cultivent et produisent}

%Programme de recherche FP6 FarmSeedOpportunities (2007-2010)

\begin{tabular}{ p{.3\textwidth} p{.2\textwidth} p{.4\textwidth} }
\hline
Type de variétés & Variété & Commentaires \\
\hline
Variété moderne & Renan & \\

\hline
Variétés de pays conservées \exsitu 	& Haute et Loire Piave		& En banque de graine depuis moins de 5 ans \\

\hline
Variétés de pays / variétés historiques conservées à la ferme 	& Rouge de Bordeaux et Solina d'Abruzzo & 2 variétés historiques (1880) cultivées continuellement à la ferme \\
															
\hline
Mélanges paysans cultivés à la ferme 	& Zonne Hoeve  	& mélange de deux variétés récentes (1990) cultivé 5 à 10 ans à la ferme \\
									& Redon 		& variétés mélangées issues de banque de graines \\
									& Touzelles 	& mélange cultivé entre 5 et 10 ans à la ferme \\
\hline
\end{tabular}


\end{frame}


\begin{frame}
\frametitle{La gestion \insitu}
\framesubtitle{Dans les réseaux de \MSPs~qui cultivent et produisent}

\begin{columns}

\begin{column}{.5\textwidth}
\begin{itemize}
\item Echantillons initiaux G0 des 8 variétés distribués et cultivés sur 7 fermes de 2006 2007 à 2008	2009. 
\item En 2011: 56 populations $\times$ environ 30 individus ont été génotypés avec 45 marqueurs neutres \& 48 marqueurs situés dans des gènes associés à la précocité et la hauteur. 
\end{itemize}
\end{column}


\begin{column}{.5\textwidth}
\begin{center}
\includegraphics[width=\textwidth]{carte_fso}
\end{center}
\end{column}

\end{columns}

\end{frame}


\begin{frame}
\frametitle{La gestion \insitu}
\framesubtitle{Dans les réseaux de \MSPs~qui cultivent et produisent}

\begin{columns}

\begin{column}{.4\textwidth}
Composition génétique des 8 variétés

Les couleurs correspondent aux groupes détectés:
22 groupes spécifiques d’une variété, 2 groupes partagés.
\end{column}

\begin{column}{.6\textwidth}
\begin{center}
\includegraphics[width=\textwidth]{abdul_pie_groups}
\end{center}
\end{column}

\end{columns}

\end{frame}




\begin{frame}
\frametitle{La gestion \insitu}
\framesubtitle{Dans les réseaux de \MSPs~qui cultivent et produisent}

\begin{columns}

\begin{column}{.75\textwidth}
\begin{center}
\includegraphics[width=\textwidth]{abdul_haplo_network}
\end{center}
\end{column}

\begin{column}{.25\textwidth}
Présentation de tous les génotypes observés plus d’une fois, ils sont connectés s’ils diffèrent au plus à 15 marqueurs. 
\begin{center}
\includegraphics[width=\textwidth]{abdul_haplo_network_legend}
\end{center}
\end{column}

\end{columns}

\end{frame}



\begin{frame}%<1-2>
\frametitle{La gestion \insitu}
\framesubtitle{Dans les réseaux de \MSPs~qui cultivent et produisent}

Différences phénotypiques après 3 générations de culture
\vspace{.5cm}

\tiny
%\alt<1>{\newcolumntype{g}{>{p{.03\textwidth}}}}{}
%\alt<2>{\newcolumntype{g}{>{\columncolor{mln-green}} p{.03\textwidth}}}{}
\newcolumntype{a}{>{\columncolor{mln-green}} p{.03\textwidth}}
\newcolumntype{b}{>{\columncolor{mln-green}} p{.02\textwidth}}

\begin{tabular}{
p{.025\textwidth}
a
b
a
b
p{.03\textwidth}
p{.02\textwidth}
p{.03\textwidth}
p{.02\textwidth}
a
b
p{.03\textwidth}
p{.02\textwidth}
p{.03\textwidth}
p{.02\textwidth}
p{.03\textwidth}
p{.02\textwidth}
}
  \hline
\multicolumn{17}{l}{Date de floraison} \\ 
  \hline
    & HL &  & PI &  & RB &  & RD &  & RN &  & SO &  & TO &  & ZH &  \\ 
\rowcolor{mln-brown} INI & 1250 & b & 666.7 & d & 1055 & c & 1353 & a & 1194 & a & 1229 & abc & 1242 & ab & 1284 & a \\ 
  FFM & 1236 & b & 804.9 & abc & 1100 & abc & 1355 & a & 1187 & a & 1227 & bc & 1252 & ab & 1296 & a \\ 
  GCX & 1261 & ab & 765.2 & bcd & 1201 & a & 1302 & ab & 1184 & ab & 1215 & bc & 1216 & ab & 1206 & b \\ 
  HHF & 1254 & b & 718 & cd & 1173 & ab & 1255 & b & 1179 & ab & 1249 & ab & 1173 & b & 1268 & a \\ 
  JFB &  &   & 754.1 & bcd & 1164 & ab &  &   &  &   & 1203 & c &  &   &  &   \\ 
  PVI & 1242 & b & 901.1 & ab & 1120 & abc & 1319 & a & 1139 & b & 1208 & bc & 1280 & ab & 1301 & a \\ 
  PVZ & 1285 & a &  &   &  &   & 1323 & a &  &   & 1268 & a &  &   &  &   \\ 
  VVC & 1266 & ab & 918.2 & a & 1100 & bc & 1329 & a & 1166 & ab & 1217 & bc & 1296 & a & 1294 & a \\ 
  \hline
\multicolumn{17}{l}{Hauteur} \\ 
  \hline
    & HL &  & PI &  & RB &  & RD &  & RN &  & SO &  & TO &  & ZH &  \\ 
\rowcolor{mln-brown} INI & 119.8 & b & 70.34 & c & 120 & ab & 193.1 & a & 69.97 & b & 121.2 & b & 137.7 & ab & 97.73 & a \\ 
  FFM & 116.2 & b & 75.18 & c & 120.5 & ab & 129.2 & a & 71.45 & ab & 119.1 & b & 126.4 & b & 91.32 & b \\ 
  GCX & 130.7 & a & 77.46 & bc & 124.7 & ab & 126.6 & a & 69.96 & b & 123.1 & ab & 134.5 & ab & 94.21 & ab \\ 
  HHF & 128.2 & a & 75.17 & c & 130.4 & a & 124.5 & a & 75.7 & a & 130.8 & a & 128.2 & b & 97.48 & ab \\ 
  JFB &  &   & 75.43 & bc & 124.5 & ab &  &   &  &   & 123.1 & ab &  &   &  &   \\ 
  PVI & 123.4 & ab & 91.94 & a & 121.8 & ab & 136.5 & a & 72.79 & ab & 125.7 & ab & 144.8 & a & 96.52 & ab \\ 
  PVZ & 124.9 & ab &  &   &  &   & 131.8 & a &  &   & 124.2 & ab &  &   &  &   \\ 
  VVC & 126.6 & ab & 87.63 & ab & 117.8 & b & 185.7 & a & 75.41 & a & 120.6 & b & 137 & ab & 98.38 & a \\ 
   \hline
\end{tabular}

\end{frame}

\begin{frame}
\frametitle{La gestion \insitu}
\framesubtitle{Dans les réseaux de \MSPs~qui cultivent et produisent}

\begin{center}
\includegraphics[width=\textwidth]{abdul_diff_mkr}
\end{center}

HL $\Rightarrow$ différentiation phénotypique mais seulement 2 gènes sous sélection, pas d'association mk-phénotype $\Rightarrow$ mécanismes épigénétiques? 

PI $\Rightarrow$ différentiation phénotypique, nombreux gènes sous sélection, plusieurs gènes associés à la variation phénotypique

\end{frame}


\begin{frame}%<1-2>
\frametitle{La gestion \insitu}
\framesubtitle{Dans les réseaux de \MSPs~qui cultivent et produisent}

\yo{Conlusion de cette étude}:

\begin{itemize}[<+->]
\item La gestion dynamique à la ferme permet de maintenir beaucoup plus de diversité au sein des variétés de pays que la conservation \exsitu

\item Les mélanges de variétés (de pays ou plus récentes) maintenus à la ferme permettent une augmentation de la diversité disponible au sein  des variétés (recombinaison et fécondation)

\item La différenciation pour des caractères adaptatifs entre populations cultivées dans des environnements contrastés est plus importante pour les variétés génétiquement hétérogènes que pour les lignées pures.
Toutefois, une certaine évolution est observée pour ces dernières.

\item La variation épigénétique pourrait expliquer une part de la différenciation entre populations issues de variétés très homogènes génétiquement.

\end{itemize}


\end{frame}


%\begin{frame}
%\frametitle{La gestion \insitu}
%\framesubtitle{Dans les réseaux de \MSPs~qui sélectionnent}
%
%résultats de Julie et Estelle cf FSO
%
%\end{frame}


\begin{frame}
\frametitle{La gestion \insitu}
\framesubtitle{Dans les réseaux de \MSPs~qui cultivent et produisent}

Quelle influence de la structure des réseaux sociaux?

\begin{overprint}
\onslide<1>\centering\includegraphics[width=0.8\textwidth,page=2]{question-cadre}\\
\onslide<2>\centering\includegraphics[width=0.8\textwidth,page=3]{question-cadre}\\
\onslide<3>\centering\includegraphics[width=0.8\textwidth,page=4]{question-cadre}\\
\onslide<4>\centering\includegraphics[width=0.8\textwidth,page=5]{question-cadre}\\
\onslide<5>\centering\includegraphics[width=0.8\textwidth,page=6]{question-cadre}\\
\end{overprint}

\end{frame}


\begin{frame}
\frametitle{La gestion \insitu}
\framesubtitle{Dans les réseaux de \MSPs~qui cultivent et produisent}

\begin{overprint}
\onslide<1>\centering\includegraphics[width=0.9\textwidth,page=1]{mdt-dyn}\\
\onslide<2>\centering\includegraphics[width=0.9\textwidth,page=2]{mdt-dyn}\\
\onslide<3>\centering\includegraphics[width=0.9\textwidth,page=3]{mdt-dyn}\\
\onslide<4>\centering\includegraphics[width=0.9\textwidth,page=4]{mdt-dyn}\\
\onslide<5>\centering\includegraphics[width=0.9\textwidth,page=5]{mdt-dyn}\\
\onslide<6>\centering\includegraphics[width=0.9\textwidth,page=6]{mdt-dyn}\\
\onslide<7>\centering\includegraphics[width=0.9\textwidth,page=7]{mdt-dyn}\\
\onslide<8>\centering\includegraphics[width=0.9\textwidth,page=8]{mdt-dyn}\\
\onslide<9>\centering\includegraphics[width=0.9\textwidth,page=9]{mdt-dyn}\\
\onslide<10>\centering\includegraphics[width=0.9\textwidth,page=10]{mdt-dyn}\\
\onslide<11>\centering\includegraphics[width=0.9\textwidth,page=11]{mdt-dyn}\\
\onslide<12>\centering\includegraphics[width=0.9\textwidth,page=12]{mdt-dyn}\\
\onslide<13>\centering\includegraphics[width=0.9\textwidth,page=16]{mdt-dyn}\\
\end{overprint}

\end{frame}


\begin{frame}
\frametitle{La gestion \insitu}
\framesubtitle{Dans les réseaux de \MSPs~qui cultivent et produisent}


\begin{overprint}
\onslide<1>\centering\includegraphics[width=.9\textwidth,page=1]{mdt-div.pdf}\\
\onslide<2>\centering\includegraphics[width=.9\textwidth,page=2]{mdt-div.pdf}\\
\onslide<3>\centering\includegraphics[width=.9\textwidth,page=3]{mdt-div.pdf}\\
\onslide<4>\centering\includegraphics[width=.9\textwidth,page=4]{mdt-div.pdf}\\
\onslide<5>\centering\includegraphics[width=.9\textwidth,page=5]{mdt-div.pdf}\\
\end{overprint}

\end{frame}

\begin{frame}
\frametitle{La gestion \insitu}
\framesubtitle{Dans les réseaux de \MSPs~qui cultivent et produisent}

\begin{overprint}
%\onslide<1>\centering\includegraphics[width=.9\textwidth,page=1]{mdt-diff.pdf}\\
\onslide<1>\centering\includegraphics[width=.9\textwidth,page=2]{mdt-diff.pdf}\\
\onslide<2>\centering\includegraphics[width=.9\textwidth,page=3]{mdt-diff.pdf}\\
\onslide<3>\centering\includegraphics[width=.9\textwidth,page=4]{mdt-diff.pdf}\\
%\onslide<5>\centering\includegraphics[width=.9\textwidth,page=5]{mdt-diff.pdf}\\
\end{overprint}

\end{frame}



\begin{frame}
\frametitle{La gestion \insitu}
\framesubtitle{Dans les réseaux de \MSPs~qui cultivent et produisent}

\begin{overprint}
%\onslide<1>\centering\includegraphics[width=.9\textwidth,page=1]{mdt-adapt.pdf}\\
%\onslide<2>\centering\includegraphics[width=.9\textwidth,page=2]{mdt-adapt.pdf}\\
%\onslide<3>\centering\includegraphics[width=.9\textwidth,page=3]{mdt-adapt.pdf}\\
%\onslide<4>\centering\includegraphics[width=.9\textwidth,page=4]{mdt-adapt.pdf}\\
\onslide<1>\centering\includegraphics[width=.9\textwidth,page=5]{mdt-adapt.pdf}\\
\end{overprint}

\end{frame}

\begin{frame}
\frametitle{La gestion \insitu}
\framesubtitle{Dans les réseaux de \MSPs~qui cultivent et produisent}

\yo{Conclusion partielles}:

\begin{itemize}
\item Structure génétique compliquée avec des niveaux de diversité très variables entre variétés

\item Différenciation rapide au niveau moléculaire et phénotypique
\begin{itemize}
\item sélection divergente : indique que la gestion à la ferme expose les populations cultivées à des conditions très contrastées
\item grande flexibilité des mélanges pour s’adapter à des conditions contrastées
\end{itemize}

\item La variété la plus diversifiée est omniprésente et évolue rapidement
\begin{itemize}
\item intérêt de la diversité assurant une capacité adaptative aux populations cultivées dans des conditions contrastées
\end{itemize}

\end{itemize}

\end{frame}



\begin{frame}
\frametitle{La gestion \insitu}
\framesubtitle{Dans les réseaux de \MSPs~qui sélectionnent}

Qu'en est il dans un programme de sélection participative? \\

\vfill


\yo{Objectif} : caractériser l'effet de différents facteurs sur la réponse des populations et le maintien de la diversité dans le cadre du programme de sélection décentralisée et participative.

\vfill

\begin{block}{}
Pour répondre à cet objectif, une étude phénotypique et moléculaire est réalisée.
\end{block}

\end{frame}


\begin{frame}
\frametitle{La gestion \insitu}
\framesubtitle{Dans les réseaux de \MSPs~qui sélectionnent. Méthodes}

\begin{columns}

\begin{column}{.6\textwidth}
\begin{itemize}
\item \textbf{Dans les croisements : tester trois facteurs : }

\begin{itemize}
\item<1-> \textit{\textbf{type de germplasm}}

 - parents : 
\begin{itemize}
\item $VP$ : Variétés de Pays
\item $VA$ : Variétés Anciennes
\item $VM$ : Variétés Modernes
\end{itemize}
- croisements :
$VA \times VP$,
$VP \times VP$,
$VM \times VM$,
$VP \times VP$.

\item<2-> \textit{\textbf{sélection massale intra-population}}
\item<8-> \textit{\textbf{fermes}} (pratiques + environnement) où les germplasms ont été cultivés

\end{itemize}
\item<9-> \textbf{Les parents} utilisés dans les croisements %issus de $CLF$ ou du RSP

\end{itemize}

\end{column}


\begin{column}{.45\textwidth}

\begin{center}
\small Histoire du germplasm $C21$  ($VM \times VP$)
 \begin{overprint}
 \onslide<1>\includegraphics[width=\textwidth]{C21_19}
 \onslide<2>\includegraphics[width=\textwidth]{C21_15}
 \onslide<3>\includegraphics[width=\textwidth]{C21_14}
 \onslide<4>\includegraphics[width=\textwidth]{C21_10} 
 \onslide<5>\includegraphics[width=\textwidth]{C21_8} 
 \onslide<6>\includegraphics[width=\textwidth]{C21_6} 
 \onslide<7>\includegraphics[width=\textwidth]{C21_4} 
 \onslide<8->\includegraphics[width=\textwidth]{C21_1}
 \end{overprint}
\end{center}

\end{column}

\end{columns}

\end{frame}



\begin{frame}
\frametitle{La gestion \insitu}
\framesubtitle{Dans les réseaux de \MSPs~qui sélectionnent. Méthodes}

\textbf{Évaluation phénotypique:}
\begin{itemize}
\item 104 pops issues de 25 germplasms + parents % 149 pops en tout
\item semis dans trois fermes : $JSG$, $OLR$ et $JFB$ avec un dispositif type ferme régionale + tout au Moulon en deux répétitions
\item Mêmes mesures que pour le projet de sélection participative + épiaisons au Moulon
\end{itemize}

\vfill

\textbf{Évaluation moléculaire : }
\begin{itemize}
\item 49 pops issues de 12 germplasms + parents % 65 pops en tout
\item entre 15 et 25 individus génotypés par populations
\item 48 marqueurs SNP dans des zones non codantes et réparties sur le génome
%\item 34 marqueurs SNP dans des gènes candidats
\end{itemize}

\end{frame}



\begin{frame}
\frametitle{La gestion \insitu}
\framesubtitle{Dans les réseaux de \MSPs~qui sélectionnent. Résultats}

\begin{columns}

\begin{column}{.6\textwidth}
%\includegraphics[page=4,width=\textwidth]{DAPC_parents}
%\includegraphics[page=1,width=.5\textwidth]{DAPC_parents}
ACP sur les données phénotypiques sur les parents \\
\includegraphics[page=2,width=\textwidth]{ACP_pheno_bis}
\end{column}

\begin{column}{.4\textwidth}

\begin{itemize}

\item \textbf{Analyse phénotypique} \\
- pas de regroupement selon le type.\\
- plus de diversité dans $VP$

\item \textbf{Analyse moléculaire :} \\
Pas de regroupement selon le type.
%Les parents se structurent selon leur germplasm et pas selon leur type.
%\item La diversité au sein des parents est liée aux pratiques de conservation.

\end{itemize}

\end{column}

\end{columns}

\end{frame}


\begin{frame}
\frametitle{La gestion \insitu}
\framesubtitle{Dans les réseaux de \MSPs~qui sélectionnent. Résultats moléculaires}

\begin{itemize}

\item AMOVA sur les marqueurs neutres

\begin{tabular}{p{.2\textwidth}rrrr}
\hline
& type & germplasm & population & résiduelle \\
\hline
\% de variation & 12\% & 27\%*** & 13\%*** & 48\% \\
\hline
\end{tabular}

\begin{itemize}
\item Pas d'effet \textbf{type de germplasm}
\item Effet fort du \textbf{germplam}
\item Effet \textbf{population} => forte différenciation intra-germplasm = divergence
\end{itemize}

\item Forte diversité dans les descendants. % ($He$). % : 46 populations avec des $He$ entre 0,14 et 0,32 et trois populations à 0.
%La diversité a été conservée au cours des générations et les populations conservent un potentiel adaptatif important.

\item La sélection massale n'a pas tendance à diminuer la diversité génétique intra-population.

\end{itemize}

\end{frame}



\begin{frame}
\frametitle{La gestion \insitu}
\framesubtitle{Dans les réseaux de \MSPs~qui sélectionnent. Résultats pĥénotypiques}

\begin{itemize}[<+->]

\item \yo{Effet type :}

\begin{itemize}
\item Pas effet type sauf pour poids épis et courbure
\item Les croisements avec un ou deux $VM$ présentent un potentiel plus important d'amélioration de la production en grains par épi ainsi qu'un croisement $VA \times VP$.
\item Les germplasms de type $VM \times VP$ sont les plus diversifiés.
\end{itemize}


\item \yo{Effet germplasm :}
\begin{itemize}
\item Effet germplasm pour la majorité des caractères mesurés $\rightarrow$ les populations sont diversifiées et constituent une base de sélection pour les paysans.
\end{itemize}

\item \yo{Effet sélection massale}

\begin{itemize}
\item Pas d'effet unidirectionnel de la sélection sauf pour les barbes %(perspective : tester sélection $\times$ 
type de germplasm).
\end{itemize}

\item \yo{Effet ferme (pratiques + environnement)}
\begin{itemize}
\item Tous germplasms confondus, pas d'effet ferme.
\end{itemize}


\end{itemize}


\end{frame}




\begin{frame}
\frametitle{La gestion \insitu}
\framesubtitle{Dans les réseaux de \MSPs~qui sélectionnent. Résultats}

\begin{columns}

\begin{column}{.5\textwidth}

\begin{center}
ACP sur les 48 marqueurs neutres
\end{center}
\includegraphics[width=\textwidth]{ACP_C21_color}
\end{column}

\begin{column}{.5\textwidth}

\begin{itemize}
\item les  \textcolor{bleu}{parents} \textcolor{vert}{séparés},

\item \colorbox{black}{\textcolor{yellow}{un groupe}} avec une sélection : \colorbox{black}{\textcolor{yellow}{$\#Sb$-$select$-$R10\_JFB$}} proche de Pollux : sélection diversifiante,
\item \textcolor{rose}{un groupe} avec les populations issues de chez \textcolor{rose}{$CHD$}, \textcolor{rose}{$FRC$}, \textcolor{rose}{$RAB$} et \textcolor{rose}{$\#b\_JFB$}. % : population issue de $\#b-JFB$ ?
\item \textcolor{marron}{un groupe} avec les populations issues de chez \textcolor{marron}{$BRE$}, \textcolor{marron}{$JSG$}, \textcolor{marron}{$OLR$}.
\item La diversité $He$ variait entre 0,17 et 0,27 avec une valeur de 0 pour \colorbox{black}{\textcolor{yellow}{$\#Sb$-$select$-$R10\_JFB$}}.
\end{itemize}

\end{column}

\end{columns}

\end{frame}


%\item Aucune hétérozygotie nulle à part pour deux populations.

%\item Les parents et les histoires évolutives structurent les populations au sein de chaque germplasm.

%\item Certains germplasm contiennent des populations peu différenciées : un fond génétique structure les populations.
%	\begin{itemize}
%	\item 4 germplasms proches d'un des parents : effet échantillonnage, dérive, sélection, autofécondation
%	\item 3 germplasms distants des deux parents
%	\end{itemize}

%\item Certains germplasm contiennent des populations différenciées.
%	\begin{itemize}
%	\item Pas de structuration claire en groupe pour 5 germplasms
%	\end{itemize}


%\begin{frame}
%{\color{mln-green} Le programme dans la gestion des ressources génétiques}
%{\color{mln-brown} Un germplasm = une métapopulation}
%
%
%\begin{tabular}{
%p{.2\textwidth}
%p{.35\textwidth}
%p{.35\textwidth}
%}
%\hline
%force évolutive & description & analyse \\
%\hline
%
%pression de sélection dans les fermes & 
%conditions pédo-climatiques, pratiques culturales &
%\\
%\hline
%
%sélection intra-population & & \\
%
%\hline
%dérive & 
%effet d'échantillonnage 750 plantes en moyenne par micro-parcelle & 
%\\
%
%mutations & 
%670 000 plantes cultivées dans le réseau en 2012 &
%Génération de mutations qui pourront être sélectionnées par les paysans \\
%
%\end{tabular}
%
%\end{frame}
%
%
%
%\begin{frame}
%{\color{mln-green} Le programme dans la gestion des ressources génétiques}
%{\color{mln-brown} Un germplasm = une métapopulation}
%
%\begin{tabular}{
%p{.15\textwidth}
%p{.3\textwidth}
%p{.5\textwidth}
%}
%\hline
%force évolutive & description & analyse \\
%\hline
%
%migration &
%\begin{itemize}
%\item pollen (micro-parcelle côte à côte)
%\item grains (triage et échanges entre paysans dans le cadre du projet et aussi au delà \citep{thomas_gestion_2011})
%\end{itemize}
%&
%
%\begin{itemize}
%\item Croisement de l'ordre de 5\% \citep{enjalbert_relevance_1998}
%\item $He$ important + hétérozygotie plus forte qu'attendue en génération avancée.
%\item Maintient une certaine diversité qui compense la dérive.
%\item L'effet d'hétérosis apporte un avantage adaptatif ? \citep{enjalbert_inferring_2000}
%\end{itemize}
%\\
%
%\hline
%\end{tabular}
%
%\end{frame}
%

\begin{frame}
\frametitle{La gestion \insitu}
\framesubtitle{Dans les réseaux de \MSPs~qui sélectionnent. Discussion}

\yo{Un germplasm = une métapopulation = un ensemble de populations} \\

Chaque population est soumise aux forces évolutives : 
\begin{itemize}
\item migrations (pollens, grains, échanges),
\item dérive (taille des micro-parcelles), 
\item sélection (environnement, pratique, homme), 
\item mutation.
\end{itemize}

\begin{block}{}
La sélection participative, en favorisant l'utilisation de variétés hétérogènes dans des environnements très contrastés et dans un réseau d'acteurs connectés participe à la gestion des ressources génétiques.%\\
%Ces résultats permettent d'étendre les connaissances sur l'évolution de la diversité au sein des populations en gestion dynamique en station et à la ferme \citep{enjalbert_dynamic_2011} aux populations soumises à la sélection humaine au sein d'un programme de sélection participative.
\end{block}

\end{frame}


