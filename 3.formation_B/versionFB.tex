
%%%%%%%%%%%%%%%%%%%%%%%%%%%%%%%%%%%%%%%%%%%%%%%%%%%%%%%%%%%%%%%%%%%%%%%%%%%%%%%%%%%%%%%%%%%%%%%%
%\newcommand{\version}{1.0.}
%\newcommand{\dateversion}{26 mai 2015}
% Présentation: Version de départ
% Fiche: Titres des différentes parties
%%%%%%%%%%%%%%%%%%%%%%%%%%%%%%%%%%%%%%%%%%%%%%%%%%%%%%%%%%%%%%%%%%%%%%%%%%%%%%%%%%%%%%%%%%%%%%%%

%%%%%%%%%%%%%%%%%%%%%%%%%%%%%%%%%%%%%%%%%%%%%%%%%%%%%%%%%%%%%%%%%%%%%%%%%%%%%%%%%%%%%%%%%%%%%%%%
%\newcommand{\version}{1.0.1.}
%\newcommand{\dateversion}{10 juin 2015}
% Présentation: 
% Mise à jour de la diapo 94 avec plus d'illusatrations
% Suppression du sous titre diapo 41
%%%%%%%%%%%%%%%%%%%%%%%%%%%%%%%%%%%%%%%%%%%%%%%%%%%%%%%%%%%%%%%%%%%%%%%%%%%%%%%%%%%%%%%%%%%%%%%%

%%%%%%%%%%%%%%%%%%%%%%%%%%%%%%%%%%%%%%%%%%%%%%%%%%%%%%%%%%%%%%%%%%%%%%%%%%%%%%%%%%%%%%%%%%%%%%%%
\newcommand{\versionFB}{2}
\newcommand{\dateversionFB}{17 juin 2015}
% Présentation: 
% - Ajouts des forces évolutives
% Mise à jour schéma méthodo
%%%%%%%%%%%%%%%%%%%%%%%%%%%%%%%%%%%%%%%%%%%%%%%%%%%%%%%%%%%%%%%%%%%%%%%%%%%%%%%%%%%%%%%%%%%%%%%%


%%%%%%%%%%%%%%%%%%%%%%%%%%%%%%%%%%%%%%%%%%%%%%%%%%%%%%%%%%%%%%%%%%%%%%%%%%%%%%%%%%%%%%%%%%%%%%%%
% to do list

% Voir à appeler avec un input dans formation 1?  mais avec qql diapos en moins, plus, qui change: à voir ce qui est faisable pour ne pas s'emmerder à refaire des copier coller / mise à jour

% - pour lamarak: exemple d'une plante malade qui reste et n'est plus malade après (demander à isa si ok: non a priori car sur temps court a voir sur la tomate ?!?)

% pratique exsitu dans les CRB (cf Mathieu)
% résultats julie suite à FSO (ACP et on différentie toujours les pops)
% résultats isa et jerome sur GD en station cf enjalbert 2011

% fonction R pour mixture  CCP PS => faire des simuls (cf thèse gaelle) en lien avec taille des pops (f pour partie genet des pops dans formation sélection)

% parler des autres méthodes comme les OGM et la mutagénèse ?

% Mieux coller le nom des auteurs de photos! Faire une fonction pour ça? Idem pour formation 1, ds structure_common

% mettre les liens des sites de julien de cetab des banques de graines
%Pour info , je vous donne le lien vers une association anglaise qui bosse sur les blés anciens et qui possède un site  qui permet de faire des recherches dans les divers conservatoires mondiaux en les regroupant sur un seul moteur de recherche/

%lien: http://www.brockwell-bake.org.uk/wheat/

%mais aussi une bibliographie sur divers sujets liés au blé (sélection,généalogie, gluten, maladies...)

%lien: http://www.brockwell-bake.org.uk/docs/

%Seul problème tout est quasiment en Anglais donc pas très adapté à la diffusion auprès des paysans surtout qu'il y a pas mal d'études scientifiques dont la lecture est assez ardue...


%%%%%%%%%%%%%%%%%%%%%%%%%%%%%%%%%%%%%%%%%%%%%%%%%%%%%%%%%%%%%%%%%%%%%%%%%%%%%%%%%%%%%%%%%%%%%%%%

